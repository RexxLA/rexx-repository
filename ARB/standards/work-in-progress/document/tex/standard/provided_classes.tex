%preprocessed texin
\chapter{Provided classes}\label{provided-classes}

(Informative)

\section{Notation}\label{notation}

The provided classes are defined mainly through code. \#\# The
Collection Classes

\subsection{Collection Class Routines}\label{collection-class-routines}

These routines are used in the definition of the collection classes
::routine CommonxXor /* Returns a new collection that contains all items
from self and the argument except that all indexes that appear in both
collections are removed. \emph{/ /} When the target is a bag, there may
be an index in the bag that is duplicated and the same value as an index
in the argument. Should one copy of the index survive in the bag?
\emph{/ vel if (arg(1)\textasciitilde class==.Set \&
arg(2)\textasciitilde class==.Bag) then v=2 if
(arg(1)\textasciitilde class==.Table \&
arg(2)\textasciitilde class==.Bag) then v=2 if
(arg(1)\textasciitilde class==.Table \&
arg(2)\textasciitilde class==.Relation) then v=2 if
(arg(1)\textasciitilde class==.Directory \&
arg(2)\textasciitilde class==.Bag) then v=2 if
(arg(1)\textasciitilde class==.Directory \&
arg(2)\textasciitilde class==.Relation) then v=2 /} This version it
does: \emph{/ if v=1 then do This = arg(1) /} self of caller */
r=This\textsubscript{class}new ab=MayEnBag (arg (2) )
ss=This\textasciitilde supplier do while ss\textasciitilde available
r{[}ss\textasciitilde index{]} =ss\textasciitilde item
ss\textasciitilde next end cs=ab\textasciitilde supplier do while
cs\textasciitilde available if r\textsubscript{hasindex(cs}index) then
r\textasciitilde remove (cs\textasciitilde index) else
r{[}cs\textasciitilde index{]} =cs\textasciitilde item
cs\textasciitilde next end return r end

/* But following matches practice on Set\textasciitilde XOR(bag) etc. */

This = arg(1) /* self of caller */

r=This\textsubscript{class}new

ab=MayEnBag (arg (2) )

ss=This\textasciitilde supplier

do while ss\textasciitilde available if
\ab\textsubscript{hasindex(ss}index) then r{[}ss\textasciitilde index{]}
=ss\textasciitilde item ss\textasciitilde next end

cs=ab\textasciitilde supplier

do while cs\textasciitilde available if
\This\textsubscript{hasindex(cs}index) then
r{[}cs\textasciitilde index{]} =cs\textasciitilde item es-next end

return r

::routine CommoniIntersect /* Returns a new collection of the same class
as SELF that contains the items from SELF that have indexes also in the
argument. \emph{/ /} Actually an index in SELF can only be `matched'
with one in the argument once. Hence copy and removal. \emph{/ This =
arg(1) /} self of caller */ w= .Bag\textasciitilde new
sc=This\textasciitilde supplier do while sc\textasciitilde available
wlsc\textasciitilde index{]} =sc\textasciitilde index sc-next end
r=This\textsubscript{class}new

cs=MayEnBag (arg(2))\textasciitilde supplier do while
cs\textasciitilde available i=cs\textasciitilde index if
w\textasciitilde hasindex(i) then do r{[}i{]}=This {[}i{]}
w\textasciitilde remove (i) end cs\textasciitilde next end return r

::routine CommonUnion /* Returns a new collection of the same class as
SELF that contains all the items from SELF and items from the argument
that have an index not in the first. \emph{/ /} Best to add them all. By
adding non-receiver first we ensure that receiver takes priority when
same indexes. \emph{/ This = arg(1) /} self of caller */
r=This\textsubscript{class}new cs=MayEnBag
(arg(2))\textasciitilde supplier do while cs\textasciitilde available
r{[}cs\textasciitilde index{]} =cs\textasciitilde item
cs\textasciitilde next end cs=This\textasciitilde supplier do while
cs\textasciitilde available r{[}cs\textasciitilde index{]}
=cs\textasciitilde item cs\textasciitilde next end return r

::routine CommonDifference /* Returns a new collection containing only
those index-item pairs from the SELF whose indexes the other collection
does not contain. \emph{/ This = arg(1) /} self of caller */
r=This\textsubscript{class}new cs=This\textasciitilde supplier do while
cs\textasciitilde available r{[}cs\textasciitilde index{]}
=cs\textasciitilde item es-next end cs=MayEnBag
(arg(2))\textasciitilde supplier do while cs\textasciitilde available
r\textasciitilde remove (cs\textasciitilde index) es-next end return r

::routine MayEnBag

/* For List and Queue the indexes are dropped. */ rearg(1) if r-clags ==
.List \textbar{} r-class == .Queue then r=EnBag(r) return r

::routine EnBag r=.Bag\textasciitilde new
s=arg(1)\textasciitilde supplier do while s\textasciitilde available if
arg(1)\textasciitilde class == .List \textbar{}
arg(1)\textasciitilde class == .Queue then r{[}s\textasciitilde item{]}
=s\textasciitilde item else /* This case is when the receiver is a Bag.
*/ r{[}s\textasciitilde index{]} =s\textasciitilde index
s\textasciitilde-next end return r

\subsection{The collection class}\label{the-collection-class}

::Cclass `Collection'

\subsubsection{INIT}\label{init}

::method init

expose a /* A collection is modelled as using 3 slots in an array for
each element. The first slot holds the item, the second the index, and
the third is used by particular types of collection. This order of slots
is arbitary, chosen to match order of arguments for PUT and
SUPPLIER\textasciitilde NEW. \emph{/ /} The first set of 3 slots is
reserved for other purposes, to avoid having separate variables which
the subclassing would need to access. */

a=.array\textasciitilde new

a{[}1{]} /\emph{ItemsCount}/=0

a{[}2{]}/\emph{Unique}/=0

return self

\subsubsection{EXPOSED}\label{exposed}

::method exposed private expose a

/* This method allows subclasses to get at the implementation of
Collection. */ return a

\subsubsection{FINDINDEX}\label{findindex}

::method findindex private expose a /* Returns array index if the
collection contains any item associated with the index specified or
returns 0 otherwise. \emph{/ do j=4 by 3 to
1+3}a{[}1{]}/\emph{ItemsCount}/ if alj+l{]}==arg(1) then return j end j
return 0

\subsubsection{AT}\label{at}

::method at /* vANY \emph{/ expose a /} Returns the item associated with
the specified index. */ j=self\textasciitilde findindex(arg(1)) if j=0
then return .nil return a{[}j{]}

\subsubsection{{[}{]}}\label{section}

::method `{[}{]}' /* Synonym for the AT method. */ forward message `AT'

\subsubsection{PUT}\label{put}

::method put /* rANY rANy \emph{/ expose a use arg item, index /}
Replaces any existing item associated with the specified index with the
new item. Otherwise adds the item-index pair. \emph{/
j=self\textasciitilde findindex (index) if j\textgreater0 then do
alj{]}=item return end a{[}1{]} /}ItemsCount\emph{/=a {[}1{]}
/}ItemsCount\emph{/+1 j=1+3}al{[}1{]} /\emph{ItemsCount}/ alj{]}=item
a{[}lj+1{]} =index a{[}j+2{]}=0 return /* Error 91 in OOI if context
requiring result. */

\subsubsection{{[}{]}=}\label{section-1}

:smethod `{[}{]}=' /* Synonym for the PUT method. */ forward message
`PUT'

\subsubsection{HASINDEX}\label{hasindex}

::method hasindex /* vANY \emph{/ /} Returns 1 (true) if the collection
contains any item associated with the index specified or returns 0
(false) otherwise. */

return self\textasciitilde findindex (arg(1))\textgreater0

\subsubsection{ITEMS}\label{items}

::method items expose a

/* Returns the number of items in the collection. \emph{/ return
a{[}1{]}/}ItemsCount*/

\subsubsection{REMOVE}\label{remove}

::method remove /* vANY */

expose a /* Returns and removes from a collection the member item with
the specified index. */

j=self\textasciitilde findindex(arg(1))

if j=0 then return .nil

r=al{[}j{]}

self\textasciitilde removeit (j)

return r

\subsubsection{REMOVEIT}\label{removeit}

::method removeit private

expose a

use arg j

/* Remove relevant slots from the array, with compaction. */

do j=j+3 by 3 to 14+3\emph{a{[}1{]}/}ItemsCount*/
alj-3l=saljl;alj-2{]}=al{[}j+1{]};alj-ll=alj+2{]} end j

a{[}1{]} /\emph{ItemsCount}/=a {[}1{]} /\emph{ItemsCount}/-1

return

\subsubsection{MAKEARRYA}\label{makearrya}

::method makearray

expose a /* Returns a single-index array containing the receiver list
items. \emph{/ r= .array\textasciitilde new /} To build result in. */

do j=4 by 3 to 1+3\emph{a{[}1{]}/}ItemsCount*/
r{[}r\textasciitilde dimension(1)+1{]}=a{[}j{]} end j

return r

\subsubsection{MAKEARRAYX}\label{makearrayx}

::method makearrayx private expose a /* Returns a single-index array
containing the receiver index items. \emph{/ r=
.array\textasciitilde new /} To build result in. \emph{/ do j=4 by 3 to
1+3}a{[}1{]}/\emph{ItemsCount}/
r{[}r\textasciitilde dimension(1)+1{]}=a{[}j+1{]} end j return r

\subsubsection{SUPPLIER}\label{supplier}

::method supplier expose a

/* Returns a supplier object for the list. */ return
.supplier\textsubscript{new(self}makearray: .collection,
self\textasciitilde makearrayx)

\subsection{Class list}\label{class-list}

::¢class `List' subclass Collection

\subsubsection{PUT}\label{put-1}

::method put /* vANY rANY */ use arg item, index
a=self\textasciitilde exposed

/* PUT for a List must not be an insertion. */
j=self\textasciitilde findindex (index) if j=0 then call Raise
`Syntax',93.918 alj{]}=item return

\subsubsection{OF}\label{of}

::method of class /* 1 or more oANY Are they omittable? Not in I00
\emph{/ /} Returns a newly created list containing the specified value
objects in the order specified. */ r= self \textasciitilde{} new do j =
1 to arg() r\textasciitilde{} insert (arg(j)) end j return r

\subsubsection{INSERT}\label{insert}

::method insert /* rANY oANY */

use arg item, index

a=self\textasciitilde exposed /* Returns a list-supplied index for a new
item, of specified value, which is added to the list. The new item
follows the existing item with the specified index in the list ordering.
\emph{/ /} Establish the index of what preceeds the new element. \emph{/
/} If there was no index given, the new item becomes the last on list.
\emph{/ /} .nil argument means first */

if arg(2,`E') then p=arg(2)

else p=self\textasciitilde last

/* Convert from list index to underlying array index. */

if p==.nil then j=1

else j=self\textasciitilde findindex(p)

if j=0 then call Raise `Syntax',93.918

j=j+3 /* Where new entry will be. \emph{/ /} Move space to required
place. */

a{[}1{]} /\emph{ItemsCount}/=a {[}1{]} /\emph{ItemsCount}/+1

do k=1+3\emph{a{[}1{]}/}ItemsCount*/ by -3 to j+3

a{[}k{]} =a{[}k-3{]} ;a{[}k+1{]} =a{[}k-2{]} ;a{[}k{]} =a{[}k-3{]} end

/* Insert new element. */

alj{]}=item /* A new, unique, index is needed. \emph{/ /} The basic
requirement is for something unique, so this would be correct:

is.object\textasciitilde new /* a unique object, used as a key (the
index on the list) \emph{/ }

/ /* However, a number can be used. (At risk of the user thinking it is
sensible to do arithmetic on it.) */

a{[}j+1{]}=a{[}2{]}/\emph{Unique}/;a{[}2{]} /\emph{Unique}/=a{[}2{]}
/\emph{Unique}/+1

a{[}j+2{]}=0

return a{[}j+1{]}

\subsubsection{FIRST}\label{first}

::method first a=self\textasciitilde exposed

/* Returns the index of the first item in the list. \emph{/ if
a{[}1{]}/}ItemsCount*/=0 then return .nil

141 return a{[}5{]}

\subsubsection{LAST}\label{last}

::smethod last a=self\textasciitilde exposed

/* Returns the index of the last item in the list. \emph{/ if
a{[}1{]}/}ItemsCount*/=0 then return .nil return
a{[}3\emph{a{[}1{]}/}ItemsCount*/+2{]}

\subsubsection{FIRSTITEM}\label{firstitem}

::method firstitem a=self\textasciitilde exposed

/* Returns the first item in the list. \emph{/ if
a{[}1{]}/}ItemsCount*/=0 then return .nil return a{[}4{]}

\subsubsection{LASTITEM}\label{lastitem}

::method lastitem a=self\textasciitilde exposed

/* Returns the last item in the list. \emph{/ if
a{[}1{]}/}ItemsCount*/=0 then return .nil return
a{[}3\emph{a{[}1{]}/}ItemsCount*/+1{]}

\subsubsection{NEXT}\label{next}

::method next /* vANY \emph{/ a=self\textasciitilde exposed /} Returns
the index of the item that follows the list item having the specified
index. \emph{/ j=self\textasciitilde findindex(arg(1)) if j=0 then call
Raise `Syntax',93.918 j=j+3 if
j\textgreater3}a{[}1{]}/\emph{ItemsCount}/ then return .nil /* Next of
last was requested. */ return a{[}j+1{]}

\subsubsection{PREVIOUS}\label{previous}

::method previous /* vANY */

a=self\textasciitilde exposed /* Returns the index of the item that
precedes the list item having the specified index. */

j=self\textasciitilde findindex(arg(1))

if j=0 then call Raise `Syntax',93.918

j=j-3

if j\textless4 then return .nil /* Previous of first was requested. */

return a{[}j+1{]}

\subsubsection{SECTION}\label{section-2}

::method section /* rANY oWHOLE\textgreater=0 */

a=self\textasciitilde exposed /* Returns a new list containing selected
items from the receiver list. The first item in the new list is the item
corresponding to the index specified, in the receiver list. */

j=self\textasciitilde findindex(arg(1))

if j=0 then call Raise `Syntax',93.918

r= .list\textasciitilde new /* To build result in. \emph{/ if arg(2,`E')
then s = arg(2) else s = self\textasciitilde items; do s
r\textasciitilde insert (a{[}j{]}) j=j+3 if
j\textgreater1+3}a{[}1{]}/\emph{ItemsCount}/ then leave end return r

\subsection{Class queue}\label{class-queue}

::class `Queue' subclass Collection

/* A queue is a sequenced collection with whole-number indexes. The

indexes specify the position of an item relative to the head (first
item) of the queue. Adding or removing an item changes the association
of an index to its queue item. */

\subsubsection{PUSH}\label{push}

::method push /* vANY */

/* Adds the object value to the queue at its head. \emph{/
a=self\textasciitilde exposed a{[}1{]} /}ItemCount\emph{/=a {[}1{]}
/}ItemCount*/+1

/* Slide along to make a space. \emph{/ do
j=1+3}a{[}1{]}/\emph{ItemCount}/ by -3 to 7

aljl=alj-3{]} alj+l{]}=a{[}j-2{]}+1; /* Index changes */ end j
a{[}4{]}=arg(1) a{[}5J=1 return 12.2.4.2 PULL

::method pull /* Returns and removes the item at the head of the queue.
\emph{/ a=self\textasciitilde exposed if a{[}1{]}/}ItemCount\emph{/=0
then return .nil /} Stays empty \emph{/ r=a{[}4{]} a{[}1{]}
/}ItemCount\emph{/=a {[}1{]} /}ItemCount\emph{/-1 do j=4 by 3 to
1+3}a{[}1{]}/\emph{ItemCount}/ aljl=a{[}j+3{]} alj+l{]}=a{[}j+4{]}-1; /*
Index changes */ end j return r

\subsubsection{QUEUE}\label{queue}

::method queue /* vANY \emph{/ /} Adds the object value to the queue at
its tail. \emph{/ a=self\textasciitilde exposed a{[}1{]}
/}ItemCount\emph{/=a {[}1{]} /}ItemCount\emph{/+1 a{[}1+3\emph{a{[}1{]}
/}ItemCount*/{]}=arg(1) a{[}2+3\emph{a{[}1{]}
/}ItemCount*/{]}=a{[}1{]}/}ItemCount*/ return

\subsubsection{PEEK}\label{peek}

::method peek

a=self\textasciitilde exposed /* Returns the item at the head of the
queue. The collection remains unchanged. */

return a{[}4{]}

\subsubsection{REMOVE}\label{remove-1}

::method remove /* rcWHOLE\textgreater O \emph{/ /} Returns and removes
from a collection the member item with the specified index. \emph{/
a=self\textasciitilde exposed if
a{[}1{]}/}ItemCount\emph{/\textless arg(1) then return .nil
r=self\textasciitilde remove: super (arg(1)) /} Reset the indexes.
\emph{/ k=0 do j=4 by 3 to 1+3}a{[}1{]}/\emph{ItemsCount}/ k=k+1
alj+l{]}=k end j return r

\subsection{Class table}\label{class-table}

::Class `Table' subclass Collection

\subsubsection{MAKEARRAY}\label{makearray}

::method makearray /* Returns a single-index array containing the index
objects. \emph{/ /} This is different from Collection MAKEARRAY where
items rather than indexes are in the returned array. */ forward message
`MAKEARRAYX'

\subsubsection{UNION}\label{union}

::method union /* rCOLLECTION */ return CommonUnion (self,arg(1))

\subsubsection{INTERSECTION}\label{intersection}

::method intersection /* rCOLLECTION */ return CommoniIntersect
(self,arg(1))

\subsubsection{XOR}\label{xor}

::method xor /* rCOLLECTION */ return CommonXor (self,arg(1))

\subsubsection{DIFFERENCE}\label{difference}

::method difference /* rCOLLECTION */ return
CommonDifference(self,arg(1))

\subsubsection{SUBSET}\label{subset}

::method subset /* rCOLLECTION */ return
self\textsubscript{difference(arg(1))}items = 0

\subsubsection{Class set}\label{class-set}

::¢class `Set' subclass table

/* A set is a collection that restricts the member items to have a value
that is the same as the index. Any object can be placed in a set. There
can be only

one occurrence of any object in a set. */

\subsubsection{PUT}\label{put-2}

/* Second arg same as first. Committee has dropped second? \emph{/
::method put /} YANY oANY \emph{/ /} Makes the object value a member
item of the collection and associates it with specified index. \emph{/
if arg(2,`E') then if arg(2)==arg(1) then signal error /} 949 */
self\textasciitilde put:super(arg(1),arg(1))

\subsubsection{OF}\label{of-1}

::method of class /* 1 or more rANYy \emph{/ /} Returns a newly created
set containing the specified value objects. */ r=self\textasciitilde new
do j=1 to arg() r\textasciitilde put (arg(j)) end j return r

\subsubsection{UNION}\label{union-1}

::method union /* rCOLLECTION */ return CommonUnion (self,
EnBag(arg(1)))

\subsubsection{INTERSECTION}\label{intersection-1}

::method intersection /* rCOLLECTION */ return CommoniIntersect
(self,EnBag(arg(1)))

\subsubsection{XOR}\label{xor-1}

::method xor /* rCOLLECTION */ return CommonXor (self, EnBag(arg(1)))

\subsubsection{DIFFERENCE}\label{difference-1}

::method difference /* rCOLLECTION */ return CommonDifference (self,
EnBag(arg(1)))

\subsection{Class relation}\label{class-relation}

::Class `Relation' subclass Collection

\subsubsection{PUT}\label{put-3}

::method put /* vANY rANY */

use arg item, index

a=self\textasciitilde exposed /* Makes the object value a member item of
the relation and associates it with the specified index. If the relation
already contains any items with the specified index, this method adds a
new member item value with the same index, without removing any existing
members */

a{[}1{]} /\emph{ItemsCount}/=a {[}1{]} /\emph{ItemsCount}/+1

j=1+3\emph{al{[}1{]} /}ItemsCount*/

alj{]}=item

a{[}lj+1{]} =index

a{[}j+2{]}=0

return /* Error 91 in OOI if context requiring result. */

\subsubsection{ITEMS}\label{items-1}

::method items /* oANY \emph{/ a=self\textasciitilde exposed /} Returns
the number of relation items with the specified index. If you specify no
index, this method returns the total number of items associated with all
indexes in the relation. \emph{/ if \arg(1,`E') then return
a{[}1{]}/}ItemsCount\emph{/ n=0 do j=4 by 3 to
1+3}a{[}1{]}/\emph{ItemsCount}/ if arg(1)==a{[}j+1{]} then n=n+1 end j
return n

\subsubsection{MAKEARRAY}\label{makearray-1}

::method makearray forward message `MAKEARRAYX'

\subsubsection{SUPPLIER}\label{supplier-1}

::method supplier /* oANY \emph{/ a=self\textasciitilde exposed /}
Returns a supplier object for the collection. If an index is specified,
the supplier enumerates all of the items in the relation with the
specified index. \emph{/ m=.array\textasciitilde new /} For the items
\emph{/ r=.array\textasciitilde new /} For the indexes \emph{/ do j=4 by
3 to 1+3}a{[}1{]}/\emph{ItemsCount}/ if arg(1,`E') then if
arg(1)=s=a{[}j+1{]} then iterate n=r\textasciitilde dimension(1)+1

m{[}n{]} =a{[}j{]} r{[}n{]} =a{[}j+1{]}

end j return .supplier\textasciitilde new(m,r) 12.2.7.5 UNION ::method
union /* rCOLLECTION */

/* Union for a relation is just all of both. */
r=self\textsubscript{class}new cs=self\textasciitilde supplier do while
cs\textasciitilde available

r{[}cs\textasciitilde index{]} =cs\textasciitilde item es-next end
cs=MayEnBag (arg(1))\textasciitilde supplier do while
cs\textasciitilde available r{[}cs\textasciitilde index{]}
=cs\textasciitilde item es-next end return r

\subsubsection{INTERSECTION}\label{intersection-2}

::method intersection /* rCOLLECTION */

/* Intersection for a relation requires the items as well as the keys to
match. */

r=self\textsubscript{class}new sc=self\textsubscript{class}new
cs=self\textasciitilde supplier do while cs\textasciitilde available sc
{[}cs\textasciitilde index{]} =cs\textasciitilde item
cs\textasciitilde next end cs=MayEnBag (arg(1))\textasciitilde supplier
do while cs\textasciitilde available if
sc\textsubscript{hasitem(cs}item,cs\textasciitilde index) then r
{[}ecs\textasciitilde index{]} =sc\textsubscript{removeitem(cs}item,
cs\textasciitilde next end return r

cs\textasciitilde index)

\subsubsection{XOR}\label{xor-2}

::method xor /* rCOLLECTION \emph{/ /} Returns a new relation that
contains all items from self and

the argument except that all index-item pairs that appear in both
collections are removed. */

r=self\textsubscript{class}new cs=self\textasciitilde supplier do while
cs\textasciitilde available r{[}cs\textasciitilde index{]}
=cs\textasciitilde item cs\textasciitilde next end cs=MayEnBag
(arg(1))\textasciitilde supplier do while cs\textasciitilde available if
self\textsubscript{hasitem(cs}item,cs\textasciitilde index) then
r\textsubscript{removeitem(cs}item, cs\textasciitilde index) else
r{[}cs\textasciitilde index{]} =cs\textasciitilde item
cs\textasciitilde next end return r

\subsubsection{DIFFERENCE}\label{difference-2}

::method difference /* rCOLLECTION */

/* Returns a new relation containing only those index-item pairs from
the g Y P SELF whose indexes the other collection does not contain. */
r=self\textsubscript{class}new cs=self\textasciitilde supplier

do while cs\textasciitilde available r{[}cs\textasciitilde index{]}
=cs\textasciitilde item cs\textasciitilde next end

cs=MayEnBag (arg(1))\textasciitilde supplier

do while cs\textasciitilde available r\textsubscript{removeitem(cs}item,
cs\textasciitilde index) cs\textasciitilde next end

return r

\subsubsection{SUBSET}\label{subset-1}

::method subset /* rCOLLECTION */ return
self\textsubscript{difference(arg(1))}items = 0

\subsubsection{REMOVEITEM}\label{removeitem}

::method removeitem /* YANY rANY \emph{/ a=self\textasciitilde exposed
/} Returns and removes from a relation the member item value (associated
with the specified index). If value is not a member item associated with
index index, this method returns the NIL object and removes no item.
\emph{/ do j=4 by 3 to 1+3}a{[}1{]}/\emph{ItemsCount}/ if alj{]}==arg(1)
\& a{[}j+1l{]}==arg(2) then do self\textasciitilde removeit (j) return
arg(1) end end j return .nil

\subsubsection{INDEX}\label{index}

::method index /* vANY \emph{/ a=self\textasciitilde exposed /} Returns
the index for the specified item. If there is more than one index
associated with the specified item, the one this method returns is not
defined. \emph{/ do j=4 by 3 to 1+3}a{[}1{]}/\emph{ItemsCount}/ if
arg(1)==a{[}j{]} then return a{[}j+1{]} end j return .nil

\subsubsection{ALLAT}\label{allat}

::method allat /* vANY \emph{/ a=self\textasciitilde exposed /} Returns
a single-index array containing all the items associated with the
specified index. \emph{/ r=.array\textasciitilde new do j=4 by 3 to
1+3}a{[}1{]}/\emph{ItemsCount}/ if arg(1)==a{[}j+1{]} then
r{[}r\textasciitilde dimension(1)+1{]} = aljl end j return r

\subsubsection{HASITEM}\label{hasitem}

::method hasitem /* YANY rANY \emph{/ a=self\textasciitilde exposed /}
Returns 1 (true) if the relation contains the member item value
(associated with specified index). Returns 0 (false) otherwise. \emph{/
do j=4 by 3 to 1+3}a{[}1{]}/\emph{ItemsCount}/ if alj{]}==arg(1) \&
a{[}j+l{]}==arg(2) then return 1 end j return 0

\subsubsection{ALLINDEX}\label{allindex}

::method allindex /* vANY */

a=self\textasciitilde exposed /* Returns a single-index array containing
all indexes for the specified item. \emph{/ r=.array\textasciitilde new
do j=4 by 3 to 1+3}a{[}1{]}/\emph{ItemsCount}/ if alj{]}==arg(1) then do
r{[}r-\textasciitilde dimension (1)+1{]} =a{[}j+1{]} end end j return r

\#\#3 The bag class

::class `Bag' subclass relation

/* A bag is a collection that restricts the member items to having a
value that is the same as the index. Any object can be placed in a bag,
and the same object can be placed in a bag multiple times. */

\subsubsection{OF}\label{of-2}

::method of class /* 1 or more rANYy \emph{/ /} Returns a newly created
bag containing the specified value objects. */ r=self\textasciitilde new
do j=1 to arg() r\textasciitilde put (arg(j)) end j return r

\subsubsection{PUT}\label{put-4}

::method put /* vANY oANY \emph{/ /} Committee does away with second
argument? \emph{/ /} Makes the object value a member item of the
collection and associates it with the specified index. If you specify
index, it must be the same as value. */ if arg(2,`E') then if
arg(2)=s=arg(1) then signal error self\textasciitilde put: super
(arg(1),arg(1))

\subsubsection{UNION}\label{union-2}

::method union /* rCOLLECTION */ return CommonUnion (self,
EnBag(arg(1)))

\subsubsection{INTERSECTION}\label{intersection-3}

::method intersection /* rCOLLECTION */ return CommoniIntersect
(self,EnBag(arg(1)))

\subsubsection{XOR}\label{xor-3}

::method xor /* rCOLLECTION */ return CommonXor (self, EnBag(arg(1)))

\subsubsection{DIFFERENCE}\label{difference-3}

::method difference /* rCOLLECTION */ return CommonDifference (self,
EnBag(arg(1)))

\subsection{The directory class}\label{the-directory-class}

::class `Directory' subclass Collection

\subsubsection{AT}\label{at-1}

::method at /* vANY */ a=self\textasciitilde exposed

/* Returns the item associated with the specified index. */
j=self\textasciitilde findindex(arg(1))

if j=0 then return .nil

/* Run the method if there is one. */ if a{[}j+2{]} then return
self\textasciitilde run(a{[}j{]}) return a{[}j{]}

\subsubsection{PUT}\label{put-5}

::method put /* vANY rANY \emph{/ a=self\textasciitilde exposed /} Makes
the object value a member item of the collection and associates it with
the specified index. */ if
\arg(2)\textasciitilde hasmethod(`MAKESTRING') then call Raise `Syntax',
93.938 self\textasciitilde put:super(arg(1),arg(2)
\textasciitilde makestring) return

\subsubsection{MAKEARRAY}\label{makearray-2}

::method makearray forward message `MAKEARRAYX'

\subsubsection{SUPPLIER}\label{supplier-2}

::method supplier a=self\textasciitilde exposed /* Returns a supplier
object for the directory. \emph{/ /} Check out what happens to the
SETENTRY fields. \emph{/ r=.array\textasciitilde new /} For items
\emph{/ do j=4 by 3 to 1+3}a{[}1{]}/\emph{ItemsCount}/
r{[}r\textasciitilde dimension(1)+1{]}=a{[}j{]}

end j return .supplier\textsubscript{new(r,self}makearray) \#\#\#\#
UNION ::method union /* rCOLLECTION */

return CommonUnion (self,arg(1))

\subsubsection{INTERSECTION}\label{intersection-4}

::method intersection /* rCOLLECTION */ return CommoniIntersect
(self,arg(1))

\subsubsection{XOR}\label{xor-4}

::method xor /* rCOLLECTION */ return CommonXor (self,arg(1))

\subsubsection{DIFFERENCE}\label{difference-4}

::method difference /* rCOLLECTION */ return
CommonDifference(self,arg(1))

\subsubsection{SUBSET}\label{subset-2}

::method subset /* rCOLLECTION */ return
self\textsubscript{difference(arg(1))}items = 0

\subsubsection{SETENTRY}\label{setentry}

::smethod setentry /* rSTRING oANY */

a=self\textasciitilde exposed /* Sets the directory entry with the
specified name (translated to uppercase) to the second argument,
replacing any existing entry or method for the specified name. */

n=translate(arg(1))

j=self\textasciitilde findindex(n)

if j=0 \& \arg(2,`E') then return

if \arg(2,`E') then do /* Removal */

self\textasciitilde removeit (j)

return end

if j=0 then do /* It's new \emph{/
a{[}1{]}/}ItemsCount\emph{/=a{[}1{]}/}ItemsCount\emph{/ +1
j=1+3}al{[}1{]} /\emph{ItemsCount}/ alj+l{]}=n end

alj{]}=arg(2)

a{[}4j+2{]}=0

return

\subsubsection{ENTRY}\label{entry}

::smethod entry /* rSTRING \emph{/ a=self\textasciitilde exposed /}
Returns the directory entry with the specified name (translated to
uppercase). \emph{/ n=translate(arg(1))
j=self\textasciitilde findindex(n) /}if j=0 then signal error according
to online \emph{/ /} Online has something about running UNKNOWN. \emph{/
if j=0 then return .nil /} If there is an entry decide whether to invoke
it. */ if a\textasciitilde hasindex(j) then do if \al[j+2] then return
al{[}j{]} return self\textasciitilde run(al{[}jl{]}) end

\subsubsection{HASENTRY}\label{hasentry}

::method hasentry /* rSTRING \emph{/ /} Returns 1 (true) if the
directory has an entry or a method for the specified name (translated to
uppercase) or 0 (false) otherwise. */

return self\textasciitilde findindex (translate(arg(1)))\textgreater0

\subsubsection{SETMETHOD}\label{setmethod}

::method setmethod /* rSTRING oMETHOD \emph{/
a=self\textasciitilde exposed /} Associates entry with the specified
name (translated to uppercase) with method method. Thus, the language
processor returns the result of running method when you access this
entry. \emph{/ /} (Part of METHOD checking converts string or array to
actual method.) \emph{/ n=translate(arg(1))
j=self\textasciitilde findindex(n) if j=0 \& \arg(2,`E') then return if
\arg(2,`E') then do self\textasciitilde removeit (j) return end if j=0
then do /} It's new \emph{/
a{[}1{]}/}ItemsCount\emph{/=a{[}1{]}/}ItemsCount\emph{/ +1
j=1+3}al{[}1{]} /\emph{ItemsCount}/ alj+l{]}=n end alj{]}=arg(2)
a{[}j+2{]}=1 return

\subsubsection{UNKNOWN}\label{unknown}

::method unknown /* rSTRING rARRAY */

/* Runs either the ENTRY or SETENTRY method, depending on whether the
message name supplied ends with an equal sign. If the message name does
not end with an equal sign, this method runs the ENTRY method, passing
the message name as its argument. */

if right (arg(1),1)=s=`=' then

return self\textasciitilde entry(arg(1)) /* 2?? Not clear whether second
argument is mandatory. */ t=.nil

if arg(2,`E') then t=arg(2) {[}1{]} self\textasciitilde setentry (left
(arg(1),length(arg(1))-1),t) 12.3 The stem class For some reason, the
stem class doesn't have PUT and AT methods, which stops us having a
general rule about {[}{]} synonyms AT, {[}{]}= synonyms PUT. Anyway,
committee doing without this class as such.

Here is temporary stuff showing how to use algebra in the collection
coding.

/* This 1998 version uses Rony's rules for XOR and INTERSECTION based on
UNION and DIFFERENCE */

/* Test Set-Operator-Methods on different collection objects */

/* This top part has some rough parts - not meant for standard. */

/* The dumps put out results sorted, so that comparisons can be made
between implementations that keep collections in different orders. */

/* Invocation example: settest.cmd 1\textgreater{} tmp.res
2\textgreater{} tmp.err */

/* Jnitial verification that new definitions are in effect */ J18list =
.List\textasciitilde new if \J18list\textasciitilde hasmethod(``J18'')
then signal error

/* Input collections used for the tests */

coll.1 = .array\textasciitilde of(1, 2,, 4)

coll.2 = list\textasciitilde of(2, 3, 6)

coll.3 = .queue\textsubscript{newPUSH(2)PUSH(3)}\textasciitilde PUSH(7)

coll.4 = .directory\textsubscript{new}\textasciitilde setentry(1,
``eins'')\textasciitilde\textasciitilde setentry(3, ``drei'')

coll.5 = .bag\textsubscript{newput(2)put(3)put(5)}\textasciitilde put(2)

coll.6 =
.relation\textsubscript{new}\textasciitilde{}``{[}J=''(``zwei'',
2)\st{``{[}{]}=''('``drei'', 3)}``{[}J=''('vier'',
8)\textasciitilde\textasciitilde{}``J=''C''drei'',3) coll.7 =
.set\textasciitilde of(2, 3, 9)

coll.8 = .table\textsubscript{new}\textasciitilde{}``{[}{]}=''(``zwei'',
2)\st{``{[}J=''(``drei'', 3)}``{[}T{]}J=''C''vier'', 10)

coll.0 = 8

message. 1 = ``UNION''

message.2 = ``INTERSECTION'' \_\_/* index the same in both \emph{/
message.3 = ``DIFFERENCE'' /} if index only in Ist collection \emph{/
message.4 = ``XOR'' /} unique index among both collections \emph{/
message.5 = ``SUBSET'' /} target is subset of other collection */

message. = 5

target. = coll. hstart = 4 istart = \textbar{} jstart = 1

output = 1 setOfTargets = .set\textasciitilde new

SAY ``Test Results of Set Operations on Collection Classes -- dated''
date(`U')

SAY

DO h= hstart TO target.0 /* loop over target targetID =
target.h\textsubscript{class}id IF
\setOfTargets\textasciitilde hasindex(targetID) THEN DO SAY SAY
CENTER('' Target:'' targetID ````, 70,''=``)
setOfTargets\textasciitilde put(targetID) output = 1 END

DO i= istart TO coll.0 /* loop over other collections if output then do
output = 0 argumentID = coll.i\textsubscript{class}id SAY SAY CENTER(''
argument:'' argumentID ````, 65,''=``) SAY SAY''INPUT:'' SAY ``contents
of'' pp(targetID) ``target:'' CALL dump\_collection target.h SAY

SAY ``contents of'' pp(argumentID) ``argument:''

CALL dump\_collection colli

SAY

SAY CENTER('' start set operators ``, 65,''-``) end

DO j =jstart TO message.0 /* loop over set operators tmpString
\textbar{} = RIGHT(``h'' pp(h) ``i'' ppG) ``j'' ppG), 65)

*/

*/

*/

tmpString2 = pp(targetID ``\textasciitilde{}'' message.j
\textbar\textbar{} ``('' argumentID ``)'')

SAY OVERLAY ( tmpString2, tmpString1 )

/* set resume parameter in case of error*/

jstart = j+1 IF jstart\textgreater message.0 THEN DO istart = i+{]}

IF istart\textgreater coll.0 THEN DO hstart = h+1 istart = 1

END jstart = 1 output = \textbar{} END /* process method invocation */
IF target.h\textasciitilde hasmethod(message.j) THEN DO tmp =
.message\textasciitilde new(target.h, message.j, ``I'',
coll.i)\textasciitilde send if ``The String
class''=tmp\textsubscript{class}defaultname then do if
datatype(tmp,``B'') then do

if tmp then SAY'' Result is TRUE'' else SAY'' Result is FALSE'' end end
else CALL dump\_collection tmp END ELSE

SAY pp(targetID) ``does not have method \textasciitilde{}''
pp(message.j)

SAY LEFT(````, 40,''-``) END jstart = 1 END jstart = 1 istart = 1 output
= 1 END

RETURN

dump\_collection:procedure USE ARG collection k =
.array\textasciitilde new i= .array\textasciitilde new tmpSupp =
collection\textasciitilde supplier DO WHILE
tmpSupp\textasciitilde AVAILABLE
k{[}k\textasciitilde dimension(1)+1{]}=tmpSupp\textasciitilde INDEX
i\{i\textsubscript{dimension(1)+1{]}=tmpSupp}ITEM
tmpSupp\textasciitilde NEXT END do until hope hope=1 do j=1 to
k\textasciitilde dimension(1)-1 if
k{[}j{]}\textsubscript{string\textgreater k{[}j+1{]}}string \textbar,
(k{[}j{]}\textsubscript{string=k{[}j+1{]}}string \&
i{[}j{]}\textsubscript{string\textless i{[}j+1{]}}string) then do

t=k{[}j{]}sk(jJ=kUj+ 1 )sk{[}j+1{]}=t t=i{[}j)siG {]}=if+ Usifj+t{]}=t
hope=0 end end end if O=collection\textasciitilde items then say'' The
result is empty!'' else do j=1 to k\textasciitilde dimension(1) SAY ''
'' ``index'' pp(k{[}j{]}) ``item'' ppd{[}j)

end RETURN /* Auxiliary routines \emph{/ pp: RETURN ``{[}''
\textbar\textbar{} ARG(1)\textasciitilde string \textbar\textbar{} ``J''
{[}}

/* X3J18 Rexx Language Standard proposal for the Set-like operations on
the Collection classes */

/* Tn the same way that the first standard uses BIFs which are defined
using other BIFs and ultimately the arithmetic and character operators,
the second standard can define classes using other classes and some
fundamental basis.

This program gives the definition of some other classes, in a form which
(when thoroughly developed) might be part of the second standard. It
also has a testing mechanism, which will not be part of a standard.

This particular program is implementing collections on top of array. */

/* The class Collection is not one builtin, but is used to simplify the
definition. */

::class `Collection'

:imethod init expose a /* A collection is modelled as using 3 slots in
an array for each element. The first slot holds the item, the second the
index, and the third is used by particular types of collection. This
order of slots is arbitary, chosen to match order of arguments for PUT
and SUPPLIER\textasciitilde NEW. \emph{/ /} The first set of 3 slots is
reserved for other purposes, to avoid having separate variables which
the subclassing would need to access. \emph{/
a=.alray\textasciitilde new a{[}1{]}/}ItemsCount\emph{/=0
a{[}2{]}/}Unique*/=0

return self

::method exposed private expose a

/* This method allows subclasses to get at the implementation of
Collection. */ retum a

::method findindex private expose a /* Returns array index if the
collection contains any item associated with the index specified or
returns 0 otherwise. \emph{/ do j=4 by 3 to
143}a{[}1{]}/\emph{ItemsCount}/ if a{[}j+1{]}==arg(1) then return j end
j return 0

::method at /* rANY \emph{/ expose a /} Returns the item associated with
the specified index. */ j=self\textasciitilde findindex(arg(1)) if j=0
then return .nil return a{[}j{]}

::method `{[}{]}' /* Synonym for the AT method. */ forward message `AT'

::method put /* rANY rANY \emph{/ expose a use arg item, index /}
Replaces any existing item associated with the specified index with the
new item. Otherwise adds the item-index pair. \emph{/
j=self\textasciitilde findindex(index) if j\textgreater0 then do
a{[}j{]}=item returm end a{[}1{]}/}ItemsCount\emph{/=a{[}1
{]}/}ItemsCount\emph{/+1 j=lt3}a{[}1{]}/\emph{ItemsCount}/ a{[}j{]}=item
a{[}j+1{]}=index a{[}j+2{]}=0 return /* Error 91 in OOI if context
requiring result. */

:method `{[}{]}=' /* Synonym for the PUT method. */

forward message `PUT'

z:method hasindex /* rANY \emph{/ /} Returns \textbar{} (true) if the
collection contains any item associated with the index specified or
returns O (false) otherwise. */

return self\textasciitilde findindex(arg(1))\textgreater0

::method items expose a

/* Returns the number of items in the collection. \emph{/ return
a{[}1{]}/}ItemsCount*/

zimethod remove /* rANY */

expose a /* Returns and removes from a collection the member item with
the specified index. */

j=self\textasciitilde findindex(arg(1))

if j=0 then return .nil

r=alj{]}

self\textasciitilde removeit(j)

returm r

::method removeit private

expose a

use arg j

/* Remove relevant slots from the array, with compaction. */

do j=j+3 by 3 to 14+3\emph{a{[}1{]}/}ItemsCount*/
a{[}j-3{]}=alj{]};alj-2{]}=alj+1{]};a{[}j- 1 {]}=alj+2{]} end j

a{[}1{]}/\emph{ItemsCount}/=a{[} 1 {]}/\emph{ItemsCount}/-1

return

::method makearray expose a /* Returns a single-index array containing
the receiver list items. \emph{/ r=.array\textasciitilde new = /} To
build result in. \emph{/ do j=4 by 3 to 143}a{[}1{]}/\emph{ItemsCount}/
r{[}r\textasciitilde dimension(1)+1{]}=a{[}j{]} end j returm r

::method makearrayx private expose a

/* Returns a single-index array containing the receiver index items.
\emph{/ r=.array\textasciitilde new = /} To build result in. \emph{/ do
j=4 by 3 to 143}a{[}1{]}/\emph{ItemsCount}/

r{[}r\textasciitilde dimension(1)+1{]}=a{[}j+1{]} end j return r

:imethod supplier expose a /* Returns a supplier object for the list. */
return
.supplier\textsubscript{new(self}makearray:.collection,self\textasciitilde makearrayx)

::class `List' subclass Collection

zimethod J18 = /* Here to demonstrate .LIST is replaced */ return

/* List and Queue are special because there is an order to their
elements. */

::method put /* rANY rANY */ use arg item, index
a=self\textasciitilde exposed

/* PUT for a List must not be an insertion. */
j=self\textasciitilde findindex(index) if j=0 then call Raise
`Syntax',93.918 a{[}j{]}=item retum

zimethod of class /* 1 or more oANY Are they omittable? Not in IOO
\emph{/ /} Returns a newly created list containing the specified value
objects in the order specified. */ r=self \textasciitilde{} new do j = 1
to argQ) r \textasciitilde{} insert(arg(j)) end j return r

zimethod insert /* rANY oANY */

use arg item, index

a=self\textasciitilde exposed /* Returns a list-supplied index for a new
item, of specified value, which is added to the list. The new item
follows the existing item with the specified index in the list ordering.
\emph{/ /} Establish the index of what preceeds the new element. \emph{/
/} Tf there was no index given, the new item becomes the last on list.
\emph{/ /} mil argument means first */

if arg(2,`E') then p=arg(2)

else p=self\textasciitilde last

/* Convert from list index to underlying array index. */

if p==.nil then j=1 else j=self\textasciitilde findindex(p)

if j=0 then call Raise `Syntax',93.918

j=J+3 /* Where new entry will be. \emph{/ /} Move space to required
place. */

a{[}1{]}/\emph{ItemsCount}/=a{[}1 {]}/\emph{ItemsCount}/+1

do k=1+3\emph{a{[}1{]}/}ItemsCount*/ by -3 to j+3

a{[}k{]}=a{[}k-3{]};a{[}k+1 {]}=a{[}k-2{]};a{[}k{]}=a{[}k-3{]} end

/* Insert new element. */

a{[}j{]}=item /* A new, unique, index is needed. \emph{/ /} The basic
requirement is for something unique, so this would be correct:

i=.object\textasciitilde new /* a unique object, used as a key (the
index on the list) \emph{/ }/ /* However, a number can be used. (At risk
of the user thinking it is sensible to do arithmetic on it.) */

a{[}j+1{]}=a{[}2{]}/\emph{Unique}/;a{[}2{]}/\emph{Unique}/=a{[}2
{]}/\emph{Unique}/+1

a{[}j+2{]}=0

return a{[}j+1{]}

:smethod first a=self\textasciitilde exposed

/* Returns the index of the first item in the list. \emph{/ if
a{[}1{]}/}ItemsCount*/=0 then return .nil return a{[}5{]}

::method last a=self\textasciitilde exposed

/* Returns the index of the last item in the list. \emph{/ if
a{[}1{]}/}ItemsCount*/=0 then return .nil return
a{[}3\emph{a{[}1{]}/}ItemsCount*/+2{]}\textbar{}

::method firstitem a=self\textasciitilde exposed

/* Returns the first item in the list. \emph{/ if
a{[}1{]}/}ItemsCount*/=0 then return .nil return a{[}4{]}

::method lastitem a=self\textasciitilde exposed

/* Returns the last item in the list. \emph{/ if
a{[}1{]}/}ItemsCount\emph{/=0 then return .nil return
a{[}3}a{[}1{]}/\emph{ItemsCount}/+1 \textbar{}

::method next P* rANY \emph{/ a=self\textasciitilde exposed /} Returns
the index of the item that follows the list item having the specified
index. */

j=self\textasciitilde findindex(arg(1))

if j=0 then call Raise `Syntax',93.918

jait3

if j\textgreater3\emph{a{[}1{]}/}ItemsCount\emph{/ then return .nil /}
Next of last was requested. */

return a{[}j+1{]}

zimethod previous /* rANY */

a=self\textasciitilde exposed /* Returns the index of the item that
precedes the list item having the specified index. */

j=self\textasciitilde findindex(arg(1))

if j=0 then call Raise `Syntax',93.918

ja-3

if j\textless4 then return .nil /* Previous of first was requested. */

return a{[}j+1{]}

zimethod section /* rANY OoWHOLE\textgreater=0 \emph{/
a=self\textasciitilde exposed /} Returns a new list containing selected
items from the receiver list. The first item in the new list is the item
corresponding to the index specified, in the receiver list. \emph{/
j=self\textasciitilde findindex(arg(1)) if j=0 then call Raise
`Syntax',93.918 r=.list\textasciitilde new /} To build result in.
\emph{/ if arg(2,`E') then s = arg(2) else s =
self\textasciitilde items; do s r\textasciitilde insert(a{[}j{]}) jait3
if j\textgreater1+3}a{[}1{]}/\emph{ItemsCount}/ then leave end retum r

::class `Queue' subclass Collection /* A queue is a sequenced collection
with whole-number indexes. The indexes specify the position of an item
relative to the head (first item) of

the queue. Adding or removing an item changes the association of an
index to its queue item. */

::method push /* rANY */

/* Adds the object value to the queue at its head. \emph{/
a=self\textasciitilde exposed a{[}1{]}/}ItemCount\emph{/=a{[}1
\textbar/}ItemCount*/+1

/* Slide along to make a space. \emph{/ do
j=1+3}a{[}1{]}/\emph{ItemCount}/ by -3 to 7

aljl=alj-3{]} a{[}j+1{]}=a{[}j-2{]}+1; /* Index changes */ end j
a{[}4{]}=arg(1) a{[}5{]}=1 return ::method pull

/* Returns and removes the item at the head of the queue. \emph{/
a=self\textasciitilde exposed if a{[}1{]}/}ItemCount\emph{/=0 then
return .nil /} Stays empty \emph{/ r=a{[}4{]}
a{[}1{]}/}ItemCount\emph{/=a{[}1{]}/}ItemCount\emph{/- 1 do j=4 by 3 to
14+3}a{[}1{]}/\emph{ItemCount}/ alj\}=alj+3{]} a{[}j+1{]}=al{[}j+4{]}-1;
/* Index changes */ end j returm r

zimethod queue =/* rANY */

/* Adds the object value to the queue at its tail. \emph{/
a=self\textasciitilde exposed a{[}1{]}/}ItemCount\emph{/=a{[}1
\textbar/}ItemCount\emph{/+1
a{[}\{1+3}a{[}1{]}/\emph{ItemCount}/\textbar=arg(1)
a{[}2+3\emph{a{[}1{]}/}ItemCount\emph{/\textbar=al{[} 1 {]}/}ItemCount*/
return

:imethod peek

a=self\textasciitilde exposed /* Returns the item at the head of the
queue. The collection remains unchanged. */

return a{[}4{]}

zimethod remove /* rWHOLE\textgreater O \emph{/ /} Returns and removes
from a collection the member item with the specified index. \emph{/
a=self\textasciitilde exposed if
a{[}1{]}/}ItemCount\emph{/\textless arg(1) then return .nil
r=self\textasciitilde remove:super(arg(1)) /} Reset the indexes. \emph{/
k=0 do j=4 by 3 to 143}a{[}1{]}/\emph{ItemsCount}/ k=k+1 a\{j+1J=k

end j return r

::class ``Table' subclass Collection

::method makearray /* Returns a single-index array containing the index
objects. \emph{/ /} This is different from Collection MAKEARRAY where
items rather than indexes are in the returned array. */ forward message
`MAKEARRAYX'

::method union /* rCOLLECTION */ return CommonUnion(self,arg(1))

::method intersection /* rCOLLECTION \emph{/ /} Returns a new collection
of the same class as SELF that contains the items from SELF that have
indexes also in the argument. \emph{/ /} Actually an index in SELF can
only be `matched' with one in the argument once. */

return self\textsubscript{difference(self}difference(arg(1)))

::method xor /* rCOLLECTION \emph{/ /} Returns a new relation that
contains all items from self and the argument except that all index-item
pairs that appear in both collections are removed. */ return
CommonXor(self,arg(1))

::method difference /* rCOLLECTION */ return
CommonDifference(self,arg(1))

::method subset = /* rCOLLECTION */ return
self\textsubscript{difference(arg(1))}items = 0

::class `Set' subclass table

/* A set is a collection that restricts the member items to have a value
that is the same as the index. Any object can be placed in a set. There
can be only one occurrence of any object in a set. */

/* Second arg same as first. Committee has dropped second? \emph{/
:rmethod put /} rANY oANY \emph{/ /} Makes the object value a member
item of the collection and associates it with specified index. \emph{/
if arg(2,`E') then if arg(2)==arg(1) then signal error /} 949 */
self\textasciitilde put:super(arg(1),arg(1))

zimethod of class /* 1 or more rANY \emph{/ /} Returns a newly created
set containing the specified value objects. */ r=self\textasciitilde new
do j=1 to arg() r\textasciitilde put(arg(j)) end j retum r

::method union /* rCOLLECTION */ return CommonUnion(self,EnBag(arg(1)))

::method intersection /* rCOLLECTION */ return
self\textsubscript{difference(self}difference(arg(1)))

::method xor /* rCOLLECTION */ return CommonXor(self,EnBag(arg(1)))

::method difference /* rCOLLECTION */ return
CommonDifference(self,EnBag(arg(1)))

::class `Relation' subclass Collection

::method put /* rANY rANY */

use arg item, index

a=self\textasciitilde exposed /* Makes the object value a member item of
the relation and associates it with the specified index. If the relation
already contains any items with the specified index, this method adds a
new member item value with the same index, without removing any existing
members */

a{[}1{]}/\emph{ItemsCount}/=a{[}1 {]}/\emph{ItemsCount}/+1

j=lt3\emph{a{[}1{]}/}ItemsCount*/

a{[}j{]}=item

a{[}j+1{]}=index

a{[}j+2{]}=0

return /* Error 91 in OOI if context requiring result. */

:imethod items /* oANY \emph{/ a=self\textasciitilde exposed /} Returns
the number of relation items with the specified index. If you specify

no index, this method returns the total number of items associated with
all indexes in the relation. */

if \arg(1,'E') then return a{[}1{]}/\emph{ItemsCount}/

n=0

do j=4 by 3 to 143\emph{a{[}1{]}/}ItemsCount*/

if arg(1)==a{[}j+1{]} then n=n+1 end j retum n

::method makearray forward message `MAKEARRAYX'

zimethod supplier /* oANY \emph{/ a=self\textasciitilde exposed /}
Returns a supplier object for the collection. If an index is specified,
the supplier enumerates all of the items in the relation with the
specified index. \emph{/ m=.array\textasciitilde new /} For the items
\emph{/ r=.array\textasciitilde new /} For the indexes \emph{/ do j=4 by
3 to 143}a{[}1{]}/\emph{ItemsCount}/ if arg(1,`E') then if
arg(1)==a{[}j+1{]} then iterate n=r\textasciitilde dimension(1)+1
m{[}n{]}=alj{]} r{[}n{]}=a{[}j+1{]} end j return
.supplier\textasciitilde new(m,r)

::method union /* rCOLLECTION \emph{/ /} Union for a relation is just
all of both. */ r=self\textsubscript{class}new
cs=self\textasciitilde supplier do while cs\textasciitilde available
r{[}cs\textasciitilde index{]}=cs\textasciitilde item
cs\textasciitilde next end cs=MayEnBag(arg(1))\textasciitilde supplier
do while cs\textasciitilde available
r{[}cs\textasciitilde index{]}=cs\textasciitilde item
cs\textasciitilde next end return r

::method intersection /* rCOLLECTION */ return
self\textsubscript{difference(self}difference(arg(1)))

::method xor /* rCOLLECTION */ return CommonXor(self,arg(1))

zimethod difference /* rCOLLECTION \emph{/ /} Returns a new relation
containing only those index-item pairs from the

163 SELF whose indexes the other collection does not contain. */
r=self\textsubscript{class}new cs=self\textasciitilde supplier do while
cs\textasciitilde available
r{[}cs\textasciitilde index{]}=cs\textasciitilde item
cs\textasciitilde next end cs=MayEnBag(arg(1))\textasciitilde supplier
do while cs\textasciitilde available
r\textsubscript{removeitem(cs}item,cs\textasciitilde index)
cs\textasciitilde next end return r

::method subset = /* rCOLLECTION */ return
self\textsubscript{difference(arg(1))}items = 0

zimethod removeitem /* rANY rANY \emph{/ a=self\textasciitilde exposed
/} Returns and removes from a relation the member item value (associated
with the specified index). If value is not a member item associated with
index index, this method returns the NIL object and removes no item.
\emph{/ do j=4 by 3 to 143}a{[}1{]}/\emph{ItemsCount}/ if a\{jJ==arg(1)
\& a{[}j+1{]}==arg(2) then do self\textasciitilde removeit(j) return
arg(1) end end j return .nil

:rmethod index /* rANY \emph{/ a=self\textasciitilde exposed /} Returns
the index for the specified item. If there is more than one index
associated with the specified item, the one this method returns is not
defined. \emph{/ do j=4 by 3 to 143}a{[}1{]}/\emph{ItemsCount}/ if
arg(1)==a{[}j{]} then return a{[}j+1{]} end j return .nil

::method allat /* rANY */

a=self\textasciitilde exposed /* Returns a single-index array containing
all the items associated with the specified index. */

r=.array\textasciitilde new

do j=4 by 3 to 143\emph{a{[}1{]}/}ItemsCount*/

if arg(1)==a{[}j+1{]} then r{[}r\textasciitilde dimension(1)+1{]} =
a{[}j{]} end j return r

::method hasitem /* rANY rANY \emph{/ a=self\textasciitilde exposed /}
Returns 1 (true) if the relation contains the member item value
(associated with specified index). Returns 0 (false) otherwise. \emph{/
do j=4 by 3 to 143}a{[}1{]}/\emph{ItemsCount}/ if a\{jJ==arg(1) \&
a{[}j+1{]}==arg(2) then return \textbar{} end j return 0

z:method allindex /* rANY \emph{/ a=self\textasciitilde exposed /}
Returns a single-index array containing all indexes for the specified
item. \emph{/ r=.array\textasciitilde new do j=4 by 3 to
143}a{[}1{]}/\emph{ItemsCount}/ if a{[}j{]}==arg(1) then do
r{[}r\textasciitilde dimension(1)+1{]}=a{[}j+1{]} end end j returm r

::class `Bag' subclass relation

/* A bag is a collection that restricts the member items to having a
value that is the same as the index. Any object can be placed in a bag,
and the same object can be placed in a bag multiple times. */

zimethod of class /* 1 or more rANY \emph{/ /} Returns a newly created
bag containing the specified value objects. */ r=self\textasciitilde new
do j=1 to arg() r\textasciitilde put(arg(j)) end j retum r

:rmethod put /* rANY oANY \emph{/ /} Committee does away with second
argument? \emph{/ /} Makes the object value a member item of the
collection and associates it with the specified index. If you specify
index, it must be the same as value. */ if arg(2,`E') then if
arg(2)==arg(1) then signal error
self\textasciitilde put:super(arg(1),arg(1))

/* Bag may be a subclass of relation but many methods have different
semantics. */

::method union /* rCOLLECTION */ return CommonUnion(self,EnBag(arg(1)))

::method intersection /* rCOLLECTION */ return
self\textsubscript{difference(self}difference(arg(1)))

::method xor /* rCOLLECTION */ return CommonXor(self,EnBag(arg(1)))

::method difference /* rCOLLECTION */ return
CommonDifference(self,EnBag(arg(1)))

::class `Directory' subclass Collection

/* Later we take three array elements for each element in the directory,
one for the item, one to contain the index, one to say if the item is a
method

to be run or not. */

::method at /* rANY \emph{/ a=self\textasciitilde exposed /} Returns the
item associated with the specified index. \emph{/
j=self\textasciitilde findindex(arg(1)) if j=0 then return .nil /} Run
the method if there is one. */ if a{[}j+2{]} then return
self\textasciitilde run(a{[}j{]}) return a{[}j{]}

::method put /* rANY rANY \emph{/ a=self\textasciitilde exposed /} Makes
the object value a member item of the collection and associates it with
the specified index. */ if
\arg(2)\textasciitilde hasmethod(MAKESTRING'\,') then call Raise
`Syntax', 93.938 self\textsubscript{put:super(arg(1),arg(2)}makestring)
return

::method makearray forward message `MAKEARRAYX'

:imethod supplier a=self\textasciitilde exposed

/* Returns a supplier object for the directory. */

/* Check out what happens to the SETENTRY fields. \emph{/
r=.array\textasciitilde new /} For items */

do j=4 by 3 to 143\emph{a{[}1{]}/}ItemsCount*/
r{[}r\textasciitilde dimension(1)+1{]}=a{[}j{]} end j

return .supplier\textsubscript{new(r,self}makearray)

::method union /* rCOLLECTION */ return CommonUnion(self,arg(1))

::method intersection /* rCOLLECTION */ return
self\textsubscript{difference(self}difference(arg(1)))

::method xor /* rCOLLECTION */ return CommonXor(self,arg(1))

::method difference /* rCOLLECTION */ return
CommonDifference(self,arg(1))

::method subset = /* rCOLLECTION */ return
self\textsubscript{difference(arg(1))}items = 0

z:method setentry /* rSTRING oANY \emph{/ a=self\textasciitilde exposed
/} Sets the directory entry with the specified name (translated to
uppercase) to the second argument, replacing any existing entry or
method for the specified name. \emph{/ n=translate(arg(1))
j=self\textasciitilde findindex(n) if j=0 \& \arg(2,`E') then return if
\arg(2,`E') then do /} Removal \emph{/ self\textasciitilde removeit(j)
returm end if j=0 then do /} It's new \emph{/
a{[}1{]}/}ItemsCount\emph{/=a{[}1{]}/}ItemsCount\emph{/ +1
j=lt3}a{[}1{]}/\emph{ItemsCount}/ a{[}j+1{]}J=n end a{[}j{]}=arg(2)
a{[}j+2{]}=0

returm

::method entry /* rSTRING */

a=self\textasciitilde exposed /* Returns the directory entry with the
specified name (translated to uppercase). */

n=translate(arg(1))

j=self\textasciitilde findindex(n) /\emph{if j=0 then signal error
according to online }/ /* Online has something about running UNKNOWN.
\emph{/ if j=0 then return .nil /} Tf there is an entry decide whether
to invoke it. */ if a\textasciitilde hasindex(j) then do if \a[j+2] then
return a{[}j{]} return self\textasciitilde run(a{[}j{]}) end

z:method hasentry /* rSTRING \emph{/ /} Returns 1\textbar{} (true) if
the directory has an entry or a method for the specified name
(translated to uppercase) or O (false) otherwise. */

return self\textasciitilde findindex(translate(arg(1)))\textgreater0

z:method setmethod /* rSTRING oMETHOD \emph{/
a=self\textasciitilde exposed /} Associates entry with the specified
name (translated to uppercase) with method method. Thus, the language
processor returns the result of running method when you access this
entry. \emph{/ /} (Part of METHOD checking converts string or array to
actual method.) \emph{/ n=translate(arg(1))
j=self\textasciitilde findindex(n) if j=0 \& \arg(2,`E') then return if
\arg(2,'E') then do self\textasciitilde removeit(j) returm end if j=0
then do /} It's new \emph{/
a{[}1{]}/}ItemsCount\emph{/=a{[}1{]}/}ItemsCount\emph{/ +1
j=lt3}a{[}1{]}/\emph{ItemsCount}/ a{[}j+1{]}J=n end a{[}j{]}=arg(2)
a\{j+2{]}=1

returm

:imethod unknown =/* rSTRING rARRAY */

/* Runs either the ENTRY or SETENTRY method, depending on whether the
message name supplied ends with an equal sign. If the message name does
not end with an equal sign, this method runs the ENTRY method, passing
the message name as its argument. */ if right(arg(1),1)==`=``' then

return self\textasciitilde entry(arg(1)) /* 22 Not clear whether second
argument is mandatory. */ t=.nil

if arg(2,`E') then t=arg(2){[}1{]}
self\textasciitilde setentry(left(arg(1),length(arg(1))-1),t)

routine CommonXor /* Returns a new collection that contains all items
from self and the argument except that all indexes that appear in both
collections are removed. \emph{/ /} When the target is a bag, there may
be an index in the bag that is duplicated and the same value as an index
in the argument. Should one copy of the index survive in the bag? */
Ihs=arg(1)\textasciitilde difference(arg(2))
rhs=Cast(arg(1),MayEnBag(arg(2)))\textasciitilde difference(arg(1))
return lhs\textasciitilde union(rhs)

nroutine CommonUnion /* Returns a new collection of the same class as
SELF that contains all the items from SELF and items from the argument
that have an index not in the first. \emph{/ /} Best to add them all. By
adding non-receiver first we ensure that receiver takes priority when
same indexes. \emph{/ This = arg(1) /} self of caller */
r=This\textsubscript{class}new
cs=MayEnBag(arg(2))\textasciitilde supplier do while
cs\textasciitilde available
r{[}cs\textasciitilde index{]}=cs\textasciitilde item
cs\textasciitilde next end cs=This\textasciitilde supplier do while
cs\textasciitilde available
r{[}cs\textasciitilde index{]}=cs\textasciitilde item
cs\textasciitilde next end return r

nroutine CommonDifference /* Returns a new collection containing only
those index-item pairs from the SELF whose indexes the other collection
does not contain. \emph{/ This = arg(1) /} self of caller */
r=This\textsubscript{class}new cs=This\textasciitilde supplier do while
cs\textasciitilde available
r{[}cs\textasciitilde index{]}=cs\textasciitilde item
cs\textasciitilde next end cs=MayEnBag(arg(2))\textasciitilde supplier

do while cs\textasciitilde available r\textsubscript{remove(cs}index)
cs\textasciitilde next end

return r

routine MayEnBag

/* For List and Queue the indexes are dropped. */ r=arg(1) if
r\textasciitilde class == .List \textbar{} r\textasciitilde class ==
.Queue then r=EnBag(r) retum r

routine EnBag r=.Bag\textasciitilde new s=arg(1)\textasciitilde supplier
do while s\textasciitilde available if arg(1)\textasciitilde class ==
.List \textbar{} arg(1)\textasciitilde class == .Queue then
1{[}s\textasciitilde item{]}=s\textasciitilde item else /* This case is
when the receiver is a Bag. */
1{[}s\textasciitilde index{]}=s\textasciitilde index
s\textasciitilde next end return r

/* This Cast routine commented away, since replaced by Oct 98 Rony
version. routine Cast public use arg Target, Other TmpColl =
Target\textsubscript{class}new /* Create an instance of type Target
\emph{/ TmpSupp = Other\textasciitilde supplier /} Get supplier from
Other */ signal on syntax do while TmpSupp\textasciitilde available
TmpColl{[}TmpSupp\textasciitilde index{]} = TmpSupp\textasciitilde item
TmpSupp\textasciitilde next end return TmpColl

/* Tf syntax error 93.949, then target is an index-only collection like
a set.*/ syntax:

if condition( ``O'' )\textasciitilde code = ``93.949'' then signal
IndexOnly

raise propagate /* Unhandled syntax error, raise in caller */

IndexOnly: /* This for index-only collections. */ do while
TmpSupp\textasciitilde available
TmpColl{[}TmpSupp\textasciitilde index{]} = TmpSupp\textasciitilde index
TmpSupp\textasciitilde next

end return TmpColl End commented away */

/* 98-09-24, ---ref; CAST2.CMD return a collection of type ``target''
which collected all item/index pairs of the argument ``other'' */

2: ROUTINE cast PUBLIC USE ARG target, other

SIGNAL ON SYNTAX IF ~other \textasciitilde{} HASMETHOD( ``SUPPLIER'' )
THEN RAISE SYNTAX 98.907 ARRAY (``COLLECTION
(i.e.~argument2=`other'-object must have a `SUPPLIER'-method)'' )

tmpColl = target \textasciitilde{} CLASS \textasciitilde{} NEW /* create
a an instance of type target \emph{/ tmpSupp = other\textasciitilde{}
SUPPLIER /} get supplier from other */

/* is index of ``other'' usable ? */ bIndexUsable = other
\textasciitilde{} HASMETHOD( ``UNION'' )

IF .Debug = .true THEN IF ~bIndexUsable THEN SAY'' /// index of `other'
not usable for setlike-operations''

/* possible syntax-error, if index and item must have the same value,
e.g.~for sets/bags \emph{/ SIGNAL ON SYNTAX NAME INDEX ONLY target
\textasciitilde{} CLASS \textasciitilde{} NEW \textasciitilde{} PUT(
1,2) /} test, if target-type is index-only */

SIGNAL ON SYNTAX DO WHILE tmpSupp \textasciitilde{} AVAILABLE IF
bIndexUsable THEN tmpColl{[} tmpSupp \textasciitilde{} INDEX {]} =
tmpSupp \textasciitilde{} ITEM ELSE tmpColl{[} tmpSupp \textasciitilde{}
ITEM {]} = tmpSupp \textasciitilde{} ITEM tmpSupp \textasciitilde{} NEXT
END RETURN tmpColl

INDEX\_ONLY : /* this is for index-only collections (e.g.~sets, bags) */
SIGNAL ON SYNTAX IF .Debug = .true THEN SAY'' \textbackslash{}`target'
is an index-only collection (index==item)'' DO WHILE tmpSupp
\textasciitilde{} AVAILABLE

IF bIndexUsable THEN tmpColl{[} tmpSupp \textasciitilde{} INDEX {]} =
tmpSupp \textasciitilde{} INDEX ELSE tmpColl{[} tmpSupp
\textasciitilde{} ITEM {]} = tmpSupp \textasciitilde{} ITEM tmpSupp
\textasciitilde{} NEXT END RETURN tmpColl

SYNTAX: RAISE PROPAGATE /* raise error in caller */

\section{The stream class}\label{the-stream-class}

The stream class provides input/output on external streams. ::class
stream

::method init /* rString \emph{/ Initializes a stream object for a
stream named name, but does not open the stream. ::smethod query /}
keywords */

There is also QUERY as command with method COMMAND.

Used with options, the QUERY method returns specific information about a
stream. ::method charin

::smethod charout

::method chars

::method linein

::method lineout

::method lines

::method qualify

::method command /* rString */

Returns a string after performing the specified stream command. ::method
open

There is also OPEN as command with method COMMAND.

Opens the stream to which you send the message and returns ``READY:''.

Committee dropping OPEN POSITION QUERY SEEK as methods in favour of
command use. ::method state

Returns a string that indicates the current state of the specified
stream. ::method say

::method uninit

::method position /* Ugh \emph{/ POSITION is a synonym for SEEK.
::method seek /} Ugh */

Sets the read or write position a specified number (offset) within a
persistent stream.

::method flush

Returns ``READY:''. Forces any data currently buffered for writing to be
written to the stream receiving the message.

There is also FLUSH as command with method COMMAND.

Committee dropping FLUSH.

::method close

Closes the stream that receives the message.

There is also CLOSE as command with method COMMAND.

Semantics are `seen by other thread'. ::method string

::method makearray /* rCHARLINE */ Returns a fixed array that contains
the data from the stream in line or character format, starting from the

current read position. ::method supplier

Returns a supplier object for the stream. ::method description

::smethod arrayin /* rCHARLINE */

Mixed case value works on OOI.

Committee dropping Arrayin \& Arrayout. Arrayin == MakeArray

Returns a fixed array that contains the data from the stream in line or
character format, starting from the

current read position. ::smethod arrayout /* rARRAY rCHARLINE */

Returns a stream object that contains the data from array.

\section{The alarm class}\label{the-alarm-class}

::class alarm

::method init /* Time, Msg */ Sets up an alarm for a future time atime.
::method cancel

Cancels the pending alarm request represented by the receiver. This
method takes no action if the specified time has already been reached.

\section{The monitor class}\label{the-monitor-class}

The Monitor class forwards messages to a destination object.

-local {[}`OUTPUT'{]} = .monitor\textasciitilde new(.output)

::class monitor

\subsection{INIT}\label{init-1}

Initializes the newly created monitor object.

::method init /* oDESTINATION */ expose Destination Destination =
.queue\textasciitilde new if arg(1,`E') then
Destination\textasciitilde push (arg(1)) return

\subsection{CURRENT}\label{current}

Returns the current destination object.

::smethod current expose Destination return Destination {[}1{]}

\subsection{DESTINATION}\label{destination}

Returns a new destination object.

::method destination /* oDESTINATION */ expose Destination if arg(1,`E')
then Destination\textasciitilde push (arg(1)) else
Destination\textasciitilde pull return Destination {[}1{]}

\subsection{UNKNOWN}\label{unknown-1}

Reissues or forwards to the current monitor destination all unknown
messages sent to a monitor object

::method unknown expose Destination

Extra parens needed here in original OREXX syntax

forward to destination{[}1{]} message arg(1) arguments arg(2) return
