%preprocessed texin
\chapter{Provided classes}\label{provided-classes}

(Informative)

\section{Notation}\label{notation}

The provided classes are defined mainly through code. \#\# The
Collection Classes

\subsection{Collection Class Routines}\label{collection-class-routines}

These routines are used in the definition of the collection classes

\lstinputlisting[language=rexx,label=CommonXor.rexx,caption=CommonXor.rexx]{CommonXor.rexx}

\subsection{The collection class}\label{the-collection-class}

\lstinputlisting[language=rexx,label=collectionclass.rexx,caption=collectionclass.rexx]{collectionclass.rexx}

\subsubsection{INIT}\label{init}

\lstinputlisting[language=rexx,label=collectioninitmethod.rexx,caption=collectioninitmethod.rexx]{collectioninitmethod.rexx}

\subsubsection{EXPOSED}\label{exposed}

\lstinputlisting[language=rexx,label=collectionexposedmethod.rexx,caption=collectionexposedmethod.rexx]{collectionexposedmethod.rexx}

\subsubsection{FINDINDEX}\label{findindex}

\lstinputlisting[language=rexx,label=collectionfindindexmethod.rexx,caption=collectionfindindexmethod.rexx]{collectionfindindexmethod.rexx}

\subsubsection{AT}\label{at}

\lstinputlisting[language=rexx,label=collectionatmethod.rexx,caption=collectionatmethod.rexx]{collectionatmethod.rexx}

\subsubsection{{[}{]}}\label{section}

\lstinputlisting[language=rexx,label=collectionatalt.rexx,caption=collectionatalt.rexx]{collectionatalt.rexx}

\subsubsection{PUT}\label{put}

\lstinputlisting[language=rexx,label=collectionputmethod.rexx,caption=collectionputmethod.rexx]{collectionputmethod.rexx}

\subsubsection{{[}{]}=}\label{section-1}

\lstinputlisting[language=rexx,label=collectionputalt.rexx,caption=collectionputalt.rexx]{collectionputalt.rexx}

\subsubsection{HASINDEX}\label{hasindex}

\lstinputlisting[language=rexx,label=collectionhasindexmethod.rexx,caption=collectionhasindexmethod.rexx]{collectionhasindexmethod.rexx}

\subsubsection{ITEMS}\label{items}

\lstinputlisting[language=rexx,label=collectionsitemsmethod.rexx,caption=collectionsitemsmethod.rexx]{collectionsitemsmethod.rexx}

\subsubsection{REMOVE}\label{remove}

\lstinputlisting[language=rexx,label=collectionsremovemethod.rexx,caption=collectionsremovemethod.rexx]{collectionsremovemethod.rexx}

\subsubsection{REMOVEIT}\label{removeit}

\lstinputlisting[language=rexx,label=collectionsremoveitmethod.rexx,caption=collectionsremoveitmethod.rexx]{collectionsremoveitmethod.rexx}

\subsubsection{MAKEARRAY}\label{makearray}

\lstinputlisting[language=rexx,label=collectionsmakearraymethod.rexx,caption=collectionsmakearraymethod.rexx]{collectionsmakearraymethod.rexx}

\subsubsection{MAKEARRAYX}\label{makearrayx}

\lstinputlisting[language=rexx,label=collectionsmakearrayxmethod.rexx,caption=collectionsmakearrayxmethod.rexx]{collectionsmakearrayxmethod.rexx}

\subsubsection{SUPPLIER}\label{supplier}

\lstinputlisting[language=rexx,label=collectionssuppliermethod.rexx,caption=collectionssuppliermethod.rexx]{collectionssuppliermethod.rexx}

\subsection{Class list}\label{class-list}

\lstinputlisting[language=rexx,label=classlist.rexx,caption=classlist.rexx]{classlist.rexx}

\subsubsection{PUT}\label{put-1}

\lstinputlisting[language=rexx,label=collectionsputmethod.rexx,caption=collectionsputmethod.rexx]{collectionsputmethod.rexx}

\subsubsection{OF}\label{of}

\lstinputlisting[language=rexx,label=collectionsofmethod.rexx,caption=collectionsofmethod.rexx]{collectionsofmethod.rexx}

\subsubsection{INSERT}\label{insert}

\lstinputlisting[language=rexx,label=collectionsinsertmethod.rexx,caption=collectionsinsertmethod.rexx]{collectionsinsertmethod.rexx}

\subsubsection{FIRST}\label{first}

\lstinputlisting[language=rexx,label=collectionsfirstmethod.rexx,caption=collectionsfirstmethod.rexx]{collectionsfirstmethod.rexx}

\subsubsection{LAST}\label{last}

\lstinputlisting[language=rexx,label=collectionslastmethod.rexx,caption=collectionslastmethod.rexx]{collectionslastmethod.rexx}

\subsubsection{FIRSTITEM}\label{firstitem}

\lstinputlisting[language=rexx,label=collectionsfirstitemmethod.rexx,caption=collectionsfirstitemmethod.rexx]{collectionsfirstitemmethod.rexx}

\subsubsection{LASTITEM}\label{lastitem}

\lstinputlisting[language=rexx,label=collectionslasttitemmethod.rexx,caption=collectionslasttitemmethod.rexx]{collectionslasttitemmethod.rexx}

\subsubsection{NEXT}\label{next}

\lstinputlisting[language=rexx,label=collectionsnexttitemmethod.rexx,caption=collectionsnexttitemmethod.rexx]{collectionsnexttitemmethod.rexx}

\subsubsection{PREVIOUS}\label{previous}

\lstinputlisting[language=rexx,label=collectionsprevioustitemmethod.rexx,caption=collectionsprevioustitemmethod.rexx]{collectionsprevioustitemmethod.rexx}

\subsubsection{SECTION}\label{section-2}

\lstinputlisting[language=rexx ,label=collectionssectionmethod.rexx,caption=collectionssectionmethod.rexx]{collectionssectionmethod.rexx}

\subsection{Class queue}\label{class-queue}

\lstinputlisting[language=rexx,label=collectionsqueueclass.rexx,caption=collectionsqueueclass.rexx]{collectionsqueueclass.rexx}

\subsubsection{PUSH}\label{push}

\lstinputlisting[language=rexx,label=queuepushmethod.rexx,caption=queuepushmethod.rexx]{queuepushmethod.rexx}

\subsubsection{PULL}\label{pull}

\lstinputlisting[language=rexx,label=collectionspullmethod.rexx,caption=collectionspullmethod.rexx]{collectionspullmethod.rexx}

\subsubsection{QUEUE}\label{queue}

\lstinputlisting[language=rexx,label=queuequeuemethod.rexx,caption=queuequeuemethod.rexx]{queuequeuemethod.rexx}

\subsubsection{PEEK}\label{peek}

\lstinputlisting[language=rexx,label=queuepeekmethod.rexx,caption=queuepeekmethod.rexx]{queuepeekmethod.rexx}

\subsubsection{REMOVE}\label{remove-1}

\lstinputlisting[language=rexx,label=queueremovemethod.rexx,caption=queueremovemethod.rexx]{queueremovemethod.rexx}

\subsection{Class table}\label{class-table}

\lstinputlisting[language=rexx,label=tableclass.rexx,caption=tableclass.rexx]{tableclass.rexx}

\subsubsection{MAKEARRAY}\label{makearray-1}

\lstinputlisting[language=rexx,label=tablemakearray.rexx,caption=tablemakearray.rexx]{tablemakearray.rexx}

\subsubsection{UNION}\label{union}

\lstinputlisting[language=rexx,label=tableunionmethod.rexx,caption=tableunionmethod.rexx]{tableunionmethod.rexx}

\subsubsection{INTERSECTION}\label{intersection}

\lstinputlisting[language=rexx,label=tableintersectionmethod.rexx,caption=tableintersectionmethod.rexx]{tableintersectionmethod.rexx}

\subsubsection{XOR}\label{xor}

\lstinputlisting[language=rexx,label=tablexormethod.rexx,caption=tablexormethod.rexx]{tablexormethod.rexx}

\subsubsection{DIFFERENCE}\label{difference}

\lstinputlisting[language=rexx,label=tabledifferencemethod.rexx,caption=tabledifferencemethod.rexx]{tabledifferencemethod.rexx}

\subsubsection{SUBSET}\label{subset}

\lstinputlisting[language=rexx,label=tablesubsetmethod.rexx,caption=tablesubsetmethod.rexx]{tablesubsetmethod.rexx}

\subsubsection{Class set}\label{class-set}

\lstinputlisting[language=rexx,label=classset.rexx,caption=classset.rexx]{classset.rexx}

\subsubsection{PUT}\label{put-2}

\lstinputlisting[language=rexx,label=setputmethod.rexx,caption=setputmethod.rexx]{setputmethod.rexx}

\subsubsection{OF}\label{of-1}

\lstinputlisting[language=rexx,label=setofmethod.rexx,caption=setofmethod.rexx]{setofmethod.rexx}

\subsubsection{UNION}\label{union-1}

\lstinputlisting[language=rexx,label=setunion.rexx,caption=setunion.rexx]{setunion.rexx}

\subsubsection{INTERSECTION}\label{intersection-1}

\lstinputlisting[language=rexx,label=setintersectionmethod.rexx,caption=setintersectionmethod.rexx]{setintersectionmethod.rexx}

\subsubsection{XOR}\label{xor-1}

\lstinputlisting[language=rexx,label=setxormethod.rexx,caption=setxormethod.rexx]{setxormethod.rexx}

\subsubsection{DIFFERENCE}\label{difference-1}

\lstinputlisting[language=rexx,label=setdifferencemethod.rexx,caption=setdifferencemethod.rexx]{setdifferencemethod.rexx}

\subsection{Class relation}\label{class-relation}

\lstinputlisting[language=rexx,label=relationclass.rexx,caption=relationclass.rexx]{relationclass.rexx}

\subsubsection{PUT}\label{put-3}

\lstinputlisting[language=rexx,label=relputmethod.rexx,caption=relputmethod.rexx]{relputmethod.rexx}

\subsubsection{ITEMS}\label{items-1}

\lstinputlisting[language=rexx,label=relitemsmethod.rexx,caption=relitemsmethod.rexx]{relitemsmethod.rexx}

\subsubsection{MAKEARRAY}\label{makearray-2}

\lstinputlisting[language=rexx,label=relmakearray method,caption=relmakearray method]{relmakearray method}

\subsubsection{SUPPLIER}\label{supplier-1}

\lstinputlisting[language=rexx,label=relsuppliermethod.rexx,caption=relsuppliermethod.rexx]{relsuppliermethod.rexx}

\subsubsection{UNION}\label{union-2}

\lstinputlisting[language=rexx,label=relunionmethod.rexx,caption=relunionmethod.rexx]{relunionmethod.rexx}

\subsubsection{INTERSECTION}\label{intersection-2}

\lstinputlisting[language=rexx,label=relintersectionmethod.rexx,caption=relintersectionmethod.rexx]{relintersectionmethod.rexx}

\subsubsection{XOR}\label{xor-2}

\lstinputlisting[language=rexx,label=relxormethod.rexx,caption=relxormethod.rexx]{relxormethod.rexx}

\subsubsection{DIFFERENCE}\label{difference-2}

\lstinputlisting[language=rexx,label=reldifferencemethod.rexx,caption=reldifferencemethod.rexx]{reldifferencemethod.rexx}

\subsubsection{SUBSET}\label{subset-1}

\lstinputlisting[language=rexx,label=relsubsetmethod.rexx,caption=relsubsetmethod.rexx]{relsubsetmethod.rexx}

\subsubsection{REMOVEITEM}\label{removeitem}

\lstinputlisting[language=rexx,label=relremoveitemmethod.rexx,caption=relremoveitemmethod.rexx]{relremoveitemmethod.rexx}

\subsubsection{INDEX}\label{index}

\lstinputlisting[language=rexx,label=relindexmethod.rexx,caption=relindexmethod.rexx]{relindexmethod.rexx}

\subsubsection{ALLAT}\label{allat}

\lstinputlisting[language=rexx,label=relAllAtmethod.rexx,caption=relAllAtmethod.rexx]{relAllAtmethod.rexx}

\subsubsection{HASITEM}\label{hasitem}

\lstinputlisting[language=rexx,label=relhasitemmethod.rexx,caption=relhasitemmethod.rexx]{relhasitemmethod.rexx}

\subsubsection{ALLINDEX}\label{allindex}

\lstinputlisting[language=rexx,label=relallindexmethod.rexx,caption=relallindexmethod.rexx]{relallindexmethod.rexx}

\lstinputlisting[language=rexx,label=bagclass.rexx,caption=bagclass.rexx]{bagclass.rexx}

\subsubsection{OF}\label{of-2}

\lstinputlisting[language=rexx,label=bagofmethod.rexx,caption=bagofmethod.rexx]{bagofmethod.rexx}

\subsubsection{PUT}\label{put-4}

\lstinputlisting[language=rexx,label=bagputmethod.rexx,caption=bagputmethod.rexx]{bagputmethod.rexx}

\subsubsection{UNION}\label{union-3}

\lstinputlisting[language=rexx,label=bagunionmethod.rexx,caption=bagunionmethod.rexx]{bagunionmethod.rexx}

\subsubsection{INTERSECTION}\label{intersection-3}

\lstinputlisting[language=rexx,label=bagintersectionmethod.rexx,caption=bagintersectionmethod.rexx]{bagintersectionmethod.rexx}

\subsubsection{XOR}\label{xor-3}

\lstinputlisting[language=rexx,label=bagxormethod.rexx,caption=bagxormethod.rexx]{bagxormethod.rexx}

\subsubsection{DIFFERENCE}\label{difference-3}

\lstinputlisting[language=rexx,label=bagdifferencemethod.rexx,caption=bagdifferencemethod.rexx]{bagdifferencemethod.rexx}

\subsection{The directory class}\label{the-directory-class}

\lstinputlisting[language=rexx,label=directoryclass.rexx,caption=directoryclass.rexx]{directoryclass.rexx}

\subsubsection{AT}\label{at-1}

\lstinputlisting[language=rexx,label=diratmethod.rexx,caption=diratmethod.rexx]{diratmethod.rexx}

\subsubsection{PUT}\label{put-5}

\lstinputlisting[language=rexx,label=dirputmethod.rexx,caption=dirputmethod.rexx]{dirputmethod.rexx}

\subsubsection{MAKEARRAY}\label{makearray-3}

\lstinputlisting[language=rexx,label=diremakearray.rexx,caption=diremakearray.rexx]{diremakearray.rexx}

\subsubsection{SUPPLIER}\label{supplier-2}

\lstinputlisting[language=rexx,label=dirsuppliermethod.rexx,caption=dirsuppliermethod.rexx]{dirsuppliermethod.rexx}

\subsubsection{UNION}\label{union-4}

\lstinputlisting[language=rexx,label=dirunionmethod.rexx,caption=dirunionmethod.rexx]{dirunionmethod.rexx}

\subsubsection{INTERSECTION}\label{intersection-4}

\lstinputlisting[language=rexx,label=dirintersectionmethod.rexx,caption=dirintersectionmethod.rexx]{dirintersectionmethod.rexx}

\subsubsection{XOR}\label{xor-4}

\lstinputlisting[language=rexx,label=dirxormethod.rexx,caption=dirxormethod.rexx]{dirxormethod.rexx}

\subsubsection{DIFFERENCE}\label{difference-4}

\lstinputlisting[language=rexx,label=dirdifferencemethod.rexx,caption=dirdifferencemethod.rexx]{dirdifferencemethod.rexx}

\subsubsection{SUBSET}\label{subset-2}

\lstinputlisting[language=rexx,label=dirsubsetmethod.rexx,caption=dirsubsetmethod.rexx]{dirsubsetmethod.rexx}

\subsubsection{SETENTRY}\label{setentry}

\lstinputlisting[language=rexx,label=dirsetentrymethod.rexx,caption=dirsetentrymethod.rexx]{dirsetentrymethod.rexx}

\subsubsection{ENTRY}\label{entry}

\lstinputlisting[language=rexx,label=direntrymethod.rexx,caption=direntrymethod.rexx]{direntrymethod.rexx}

\subsubsection{HASENTRY}\label{hasentry}

\lstinputlisting[language=rexx,label=dirhasentrymethod.rexx,caption=dirhasentrymethod.rexx]{dirhasentrymethod.rexx}

\subsubsection{SETMETHOD}\label{setmethod}

\lstinputlisting[language=rexx,label=dirsetmethod.rexxx,caption=dirsetmethod.rexxx]{dirsetmethod.rexxx}

\subsubsection{UNKNOWN}\label{unknown}

\lstinputlisting[language=rexx,label=dirunknownmethod.rexx,caption=dirunknownmethod.rexx]{dirunknownmethod.rexx}

\subsection{The stem class}\label{the-stem-class}

\emph{For some reason, the stem class doesn't have PUT and AT methods,
which stops us having a general rule about {[}{]} synonyms AT, {[}{]}=
synonyms PUT.}

\emph{Anyway, committee doing without this class as such.}

\emph{Here is temporary stuff showing how to use algebra in the
collection coding.}

\lstinputlisting[language=rexx,label=stemclass.rexx,caption=stemclass.rexx]{stemclass.rexx}

{[}The following code looks like a partial duplicate of the previous.
Formatting pending -- JMB{]}

\lstinputlisting[language=rexx,label=setlikeoperations.rexx,caption=setlikeoperations.rexx]{setlikeoperations.rexx}

\section{The stream class}\label{the-stream-class}

The stream class provides input/output on external streams.

\lstinputlisting[language=rexx,label=streamclass.rexx,caption=streamclass.rexx]{streamclass.rexx}

Initializes a stream object for a stream named name, but does not open
the stream.

\lstinputlisting[language=rexx,label=streamquerymethod.rexx,caption=streamquerymethod.rexx]{streamquerymethod.rexx}

\emph{There is also QUERY as command with method COMMAND.}

Used with options, the QUERY method returns specific information about a
stream.

\lstinputlisting[language=rexx,label=streamiomethods.rexx,caption=streamiomethods.rexx]{streamiomethods.rexx}

Returns a string after performing the specified stream command.

\lstinputlisting[language=rexx,label=steamopenmethod.rexx,caption=steamopenmethod.rexx]{steamopenmethod.rexx}

\emph{There is also OPEN as command with method COMMAND.}

Opens the stream to which you send the message and returns ``READY:''.

\emph{Committee dropping OPEN POSITION QUERY SEEK as methods in favour
of command use.}

\lstinputlisting[language=rexx,label=streamstatemethod.rexx,caption=streamstatemethod.rexx]{streamstatemethod.rexx}

Returns a string that indicates the current state of the specified
stream.

\lstinputlisting[language=rexx,label=streamvariousmethods.rexx,caption=streamvariousmethods.rexx]{streamvariousmethods.rexx}

POSITION is a synonym for SEEK.

\lstinputlisting[language=rexx,label=streamseekmethod.rexx,caption=streamseekmethod.rexx]{streamseekmethod.rexx}

Sets the read or write position a specified number (offset) within a
persistent stream.

\lstinputlisting[language=rexx,label=streamflushmethod.rexx,caption=streamflushmethod.rexx]{streamflushmethod.rexx}

Returns ``READY:''. Forces any data currently buffered for writing to be
written to the stream receiving the message.

\emph{There is also FLUSH as command with method COMMAND.}

\emph{Committee dropping FLUSH.}

\lstinputlisting[language=rexx,label=streamclosemethod.rexx,caption=streamclosemethod.rexx]{streamclosemethod.rexx}

Closes the stream that receives the message.

\emph{There is also CLOSE as command with method COMMAND.}

\emph{Semantics are `seen by other thread'.}

\lstinputlisting[language=rexx,label=streamstringmethod.rexx,caption=streamstringmethod.rexx]{streamstringmethod.rexx}

Returns a fixed array that contains the data from the stream in line or
character format, starting from the current read position.

\lstinputlisting[language=rexx,label=streamsuppliermethod.rexx,caption=streamsuppliermethod.rexx]{streamsuppliermethod.rexx}

Returns a supplier object for the stream.

\lstinputlisting[language=rexx,label=streamdescriptionmethod.rexx,caption=streamdescriptionmethod.rexx]{streamdescriptionmethod.rexx}

\lstinputlisting[language=rexx,label=streamaaryinmethod.rexx,caption=streamaaryinmethod.rexx]{streamaaryinmethod.rexx}

\emph{Mixed case value works on OOI.}

\emph{Committee dropping Arrayin \& Arrayout. Arrayin == MakeArray}

Returns a fixed array that contains the data from the stream in line or
character format, starting from the current read position.

\lstinputlisting[language=rexx,label=streamarrayoutmethod.rexx,caption=streamarrayoutmethod.rexx]{streamarrayoutmethod.rexx}

Returns a stream object that contains the data from array.

\section{The alarm class}\label{the-alarm-class}

\lstinputlisting[language=rexx,label=alarmclass,caption=alarmclass]{alarmclass}

Sets up an alarm for a future time atime.

\lstinputlisting[language=rexx,label=alarmcancelmethod.rexx,caption=alarmcancelmethod.rexx]{alarmcancelmethod.rexx}

Cancels the pending alarm request represented by the receiver. This
method takes no action if the specified time has already been reached.

\section{The monitor class}\label{the-monitor-class}

The Monitor class forwards messages to a destination object.

\lstinputlisting[language=rexx,label=monitorclass.rexx,caption=monitorclass.rexx]{monitorclass.rexx}

\subsection{INIT}\label{init-1}

Initializes the newly created monitor object.

\lstinputlisting[language=rexx,label=monitorinitmethod.rexx,caption=monitorinitmethod.rexx]{monitorinitmethod.rexx}

\subsection{CURRENT}\label{current}

Returns the current destination object.

\lstinputlisting[language=rexx,label=monitorcurrentmethod.rexx,caption=monitorcurrentmethod.rexx]{monitorcurrentmethod.rexx}

\subsection{DESTINATION}\label{destination}

Returns a new destination object.

\lstinputlisting[language=rexx,label=monitordestinationmethod.rexx,caption=monitordestinationmethod.rexx]{monitordestinationmethod.rexx}

\subsection{UNKNOWN}\label{unknown-1}

Reissues or forwards to the current monitor destination all unknown
messages sent to a monitor object

\lstinputlisting[language=rexx,label=monitorunknownmethod.rexx,caption=monitorunknownmethod.rexx]{monitorunknownmethod.rexx}

\emph{Extra parens needed here in original OREXX syntax}
\lstinputlisting[language=rexx,label=monitorforwardsyntax.rexx,caption=monitorforwardsyntax.rexx]{monitorforwardsyntax.rexx}
