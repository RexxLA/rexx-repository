%preprocessed texin
\chapter{Foreword}\label{foreword}

\section{Purpose}\label{purpose}

This standard provides an unambiguous definition of the programming
language Rexx. Its purpose is to facilitate portability of Rexx programs
for use on a wide variety of computer systems. History The computer
programming language Rexx was designed by Mike Cowlishaw to satisfy the
following principal aims:

\begin{itemize}
\item
  to provide a highly readable command programming language for the
  benefit of programmers and program readers, users and maintainers;
\item
  to incorporate within this language program design features such as
  natural data typing and control structures which would contribute to
  rapid, efficient and accurate program development;
\item
  to define a language whose implementations could be both reliable and
  efficient on a wide variety of computing platforms.
\end{itemize}

In November, 1990, X3 announced the formation of a new technical
committee, X3J18, to develop an American National Standard for Rexx.
This standard was published as ANSI X3.274-1996.

The popularity of ``Object Oriented'' programming, and the need for Rexx
to work with objects created in various ways, led to Rexx extensions and
to a second X3J18 project which produced this standard. (Ed - hopefully)

\section{Committee lists}\label{committee-lists}

(Here)

This standard was prepared by the Technical Development Committee for
Rexx, X3J18. There are annexes in this standard; they are informative
and are not considered part of this standard.

Suggestions for improvement of this standard will be welcome. They
should be sent to the

Information Technology Industry Council, 1250 Eye Street, NW, Washington
DC 20005-3922.

This standard was processed and approved for submittal to ANSI by the
Accredited Standards Committee on Information Processing Systems, NCITS.
Committee approval of this standard does not necessarily imply that all
committee members voted for its approval. At the time it approved this
standard, the NCITS Committee had the following members:

To be inserted The people who contributed to Technical Committee J18 on
Rexx, which developed this standard, include:
