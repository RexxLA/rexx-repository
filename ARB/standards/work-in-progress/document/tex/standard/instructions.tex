%preprocessed texin
\chapter{Instructions}\label{instructions}

This clause describes the execution of instructions, and how the
sequence of execution can vary from the normal execution in order of
appearance in the program.

Execution of the program begins with its first clause.

If we left Routine initialization to here we can leave method
initialization. \#\# Method initialization

There is a pool for local variables.

call Config ObjectNew

\#Po00ol = \#Outcome

Set self and super

\section{Routine initialization}\label{routine-initialization}

If the routine is invoked as a function, \#lsFunction.\#NewLevel shall
be set to `1', otherwise to `0O'; this affects the processing of a
subsequent RETURN instruction.

\#AllowProcedure.\#NewLevel = `1'

Many of the initial values for a new invocation are inherited from the
caller's values. \#Digits.\#NewLevel = \#Digits.\#Level

\#Form.\#NewLevel = \#Form.\#Level

\#Fuzz.\#NewLevel = \#Fuzz.\#Level

\#StartTime.\#NewLevel = \#StartTime.\#Level

\#Tracing.\#NewLevel = \#Tracing.\#Level \#Interactive.\#NewLevel =
\#Interactive.\#Level

call EnvAssign ACTIVE, \#NewLevel, ACTIVE, \#Level call EnvAssign
ALTERNATE, \#NewLevel, ALTERNATE, \#Level

do t=1 to 7 Condition = word(`SYNTAX HALT ERROR FAILURE NOTREADY NOVALUE
LOSTDIGITS',t) \#Enabling.Condition.\#NewLevel =
\#Enabling.Condition.\#Level

\#Instruction.Condition.\#NewLevel = \#Instruction.Condition.\#Level
\#TrapName.Condition.\#NewLevel = \#TrapName.Condition.\#Level
\#EventLevel.Condition.\#NewLevel = \#EventLevel.Condition.\#Level end t

If this invocation is not caused by a condition occurring, see nnn, the
state variables for the CONDITION

built-in function are copied. \#Condition.\#NewLevel =
\#Condition.\#Level \#ConditionDescription.\#NewLevel =
\#ConditionDescription.\#Level \#ConditionExtra.\#NewLevel =
\#ConditionExtra.\#Level \#ConditionInstruction.\#NewLevel =
\#ConditionInstruction.\#Level Execution of the initialized routine
continues at the new level of invocation. \#Level = \#NewLevel
\#NewLevel = \#Level + 1

\section{Clause initialization}\label{clause-initialization}

The clause is traced before execution: if pos(\#Tracing.\#Level, `AIR')
\textgreater{} 0 then call \#TraceSource

The time of the first use of DATE or TIME will be retained throughout
the clause.

\#ClauseTime.\#Level = '\,' The state variable \#LineNumber is set to
the line number of the clause, see nnn. A clause other than a null
clause or label or procedure instruction sets:

\#AllowProcedure.\#Level = `0' /* See message 17.1 */

\section{Clause termination}\label{clause-termination}

if \#InhibitTrace \textgreater{} 0 then \#InhibitTrace = \#InhibitTrace
- 1 Polling for a HALT condition occurs:

\#Response = Config Halt Query ()

if \#0utcome == `Yes' then do call Config Halt Reset

call \#Raise `HALT', substr(\#Response,2) /* May return */ end

At the end of each clause there is a check for conditions which occurred
and were delayed. It is acted on

if this is the clause in which the condition arose. do t=1 to 4
\#Condition=WORD(`HALT FAILURE ERROR NOTREADY',t) /* HALT can be
established during HALT handling. */ do while
\#PendingNow.\#Condition.\#Level \#PendingNow.\#Condition.\#Level = `0'
call \#Raise end end

Interactive tracing may be turned on via the configuration. Only a
change in the setting is significant. call Config Trace Query

if \#AtPause = 0 \& \#Outcome == `Yes' \& \#Trace QueryPrior == `No'
then do /* External request for Trace `?R!' */ \#Interactive.\#Level =
`1' \#Tracing.\#Level = `R' end

\#TraceQueryPrior = \#Outcome

Tracing just not the same with NetRexx.

When tracing interactively, pauses occur after the execution of each
clause except for CALL, DO the second or subsequent time through the
loop, END, ELSE, EXIT, ITERATE, LEAVE, OTHERWISE, RETURN, SIGNAL, THEN
and null clauses.

If the character `=' is entered in response to a pause, the prior clause
is re-executed.

Anything else entered will be treated as a string of one or more clauses
and executed by the language processor. The same rules apply to the
contents of the string executed by interactive trace as do for strings
executed by the INTERPRET instruction. If the execution of the string
generates a syntax error, the standard message is displayed but no
condition is raised. All condition traps are disabled during execution
of the string. During execution of the string, no tracing takes place
other than error or failure return codes from commands. The special
variable RC is not set by commands executed within the string, nor is
.RC.

If a TRACE instruction is executed within the string, the language
processor immediately alters the trace setting according to the TRACE
instruction encountered and leaves this pause point. If no TRACE
instruction is executed within the string, the language processor simply
pauses again at the same point in the program.

At a pause point: if \#AtPause = 0 \& \#Interactive.\#Level \&
\#InhibitTrace = 0 then do if \#InhibitPauses \textgreater{} 0 then
\#InhibitPauses = \#InhibitPauses-1 else do \#TraceInstruction = `0' do
forever

call Config Trace Query

if \#Outcome == `No' \& \#Trace QueryPrior == `Yes' then do /* External
request to stop tracing. */ \#Trace\_QueryPrior=\#Outcome
\#Interactive.\#Level = `0' \#Tracing.\#Level = `N' leave end

if \#Outcome == `Yes' \& \#Trace QueryPrior == `No' then do /* External
request for Trace `?R!' */ \#Trace QueryPrior = \#Outcome
\#Interactive.\#Level = `1' \#Tracing.\#Level = `R' leave end

if \#Interactive.\#Level \textbar{} \#TraceInstruction then leave

/* Accept input for immediate execution. */

call Config Trace Input

if length(\#0utcome) = 0 \textbar{} \#0Outcome == `=' then leave
\#AtPause = \#Level

interpret \#Outcome

\#AtPause = 0 end /* forever loop \emph{/ if \#Outcome == `=' then call
\#Retry /} With no return */ end end

\section{Instruction}\label{instruction}

\subsection{ADDRESS}\label{address}

For a definition of the syntax of this instruction, see nnn.

An external environment to which commands can be submitted is identified
by an environment name. Environment names are specified in the ADDRESS
instruction to identify the environment to which a command should be
sent.

I/O can be redirected when submitting commands to an external
environment. The submitted command's input stream can be taken from an
existing stream or from a set of compound variables with a common stem.
In the latter case (that is, when a stem is specified as the source for
the commands input stream) whole number tails are used to order input
for presentation to the submitted command. Stem.0 must contain a whole
number indicating the number of compound variables to be presented, and
stem. 1 through stem.n (where n=stem.0) are the compound variables to be
presented to the submitted command.

Similarly, the submitted command's output stream can be directed to a
stream, or to a set of compound variables with a given stem. In the
latter case (i.e., when a stem is specified as the destination) compound
variables will be created to hold the standard output, using whole
number tails as described above. Output redirection can specify a
REPLACE or APPEND option, which controls positioning prior to the
command's execution. REPLACE is the default.

I/O redirection can be persistently associated with an environment name.
The term ``environment'' is used to refer to an environment name
together with the I/O redirections.

At any given time, there will be two environments, the active
environment and the alternate environment. When an ADDRESS instruction
specifies a command to the environment, any specified I/O redirection
applies to that command's execution only, providing a third environment
for the duration of the instruction. When an ADDRESS command does not
contain a command, that ADDRESS command creates a new active
environment, which includes the specified I/O redirection.

The redirections specified on the ADDRESS instruction may not be
possible. If the configuration is aware that the command processor named
does not perform I/O in a manner compatible with the request, the value
of \#Env\_Type. may be set to `UNUSED' as an alternative to `STEM' and
`STREAM' where those values are assigned in the following code.

In the following code the particular use of \#Contains(address,
expression) refers to an expression immediately contained in the
address.

Addrinstr: /* If ADDRESS keyword alone, environments are swapped.
\emph{/ if \#Contains (address, taken constant), \& \#Contains
(address,valueexp), \& \#Contains (address, `WITH') then do call
EnvAssign TRANSIENT, \#Level, ACTIVE, \#Level call EnvAssign ACTIVE,
\#Level, ALTERNATE, \#Level call EnvAssign ALTERNATE, \#Level,
TRANSIENT, \#Level return end /} The environment name will be explicitly
specified. */ if \#Contains(address,taken constant) then Name =
\#Instance(address, taken \_ constant) else Name = \#Evaluate(valueexp,
expression) if length(Name) \textgreater{} \#LimitEnvironmentName then
call \#Raise `SYNTAX', 29.1, Name

if \#Contains(address,expression) then do /* The command is evaluated
(but not issued) at this point. */ Command = \#Evaluate (address,
expression)

if \#Tracing.\#Level == `C' \textbar{} \#Tracing.\#Level == `A' then do
call \#Trace '\textgreater\textgreater\textgreater! end

end

call AddressSetup /* Note what is specified on the ADDRESS instruction.
\emph{/ /} If there is no command, the persistent environment is being
set. */ if \#Contains(address,expression) then do

call EnvAssign ACTIVE, \#Level, TRANSIENT, \#Level

return

end

call CommandIssue Command /* See nnn \emph{/ return /} From Addrinstr */

AddressSetup: /* Note what is specified on the ADDRESS instruction, into
the TRANSIENT environment. \emph{/ EnvTail = `TRANSIENT.'\#Level /}
Initialize with defaults. */ \#Env\_Name.EnvTail = '\,'

\#Env\_ Type.I.EnvTail = `NORMAL' \#Env\_ Type.O.EnvTail = `NORMAL'
\#Env\_ Type.E.EnvTail = `NORMAL'

\#Env\_Resource.I.EnvTail = '\,' \#Env\_Resource.O.EnvTail = `!
\#Env\_Resource.E.EnvTail ='' /* APPEND / REPLACE does not apply to
input. */

\#Env\_Position.I.EnvTail = `INPUT' \#Env\_Position.O.EnvTail =
`REPLACE' \#Env\_Position.E.EnvTail = `REPLACE'

/* If anything follows ADDRESS, it will include the command processor
name.*/ \#Env\_Name.EnvTail = Name

/* Connections may be explicitly specified. */ if \#Contains (address,
connection) then do

if \#Contains(connection,input) then do /* input redirection */ if
\#Contains (resourcei, `STREAM') then do \#Env\_Type.I.EnvTail =
`STREAM' \#Env\_Resource.1.EnvTail=\#Evaluate(resourcei, VAR\_SYMBOL)
end if \#Contains (resourcei, `STEM') then do \#Env\_Type.I.EnvTail =
`STEM'

Temp=\#Instance (resourcei,VAR\_SYMBOL) if \#Parses(Temp, stem /* See
nnn \emph{/) then call \#Raise `SYNTAX', 53.3, Temp
\#Env\_Resource.I.EnvTail=Temp end end /} Input */

if \#Contains(connection,output) then /* output redirection */ call
NoteTarget O

if \#Contains(connection,error) then /* error redirection \emph{/ /} The
prose on the description of \#Contains specifies that the relevant
resourceo is used in NoteTarget. */ call NoteTarget E

end /* Connection */

return /* from AddressSetup */

NoteTarget:

/* Note the characteristics of an output resource. */

arg Which /* O or E */

if \#Contains (resourceo, `STREAM') then do
\#Env\_Type.Which.EnvTail=`STREAM'
\#Env\_Resource.Which.EnvTail=\#Evaluate(resourceo, VAR\_SYMBOL) end

if \#Contains(resourceo,`STEM') then do \#Env\_Type.Which.EnvTail=`STEM'
Temp=\#Instance (resourceo, VAR\_SYMBOL) if \#Parses(Temp, stem /* See
nnn */) then

call \#Raise `SYNTAX', 53.3, Temp

\#Env\_Resource.Which.EnvTail=Temp end

if \#Contains (resourceo,append) then
\#Env\_Position.Which.EnvTail=`APPEND' return /* From NoteTarget */

EnvAssign: /* Copy the values that name an environment and describe its
redirections. */ arg Lhs, LhsLevel, Rhs, RhsLevel
\#Env\_Name.Lhs.LhsLevel = \#Env\_Name.Rhs.RhsLevel \#Env\_
Type.I.Lhs.LhsLevel = \#Env\_Type.I.Rhs.RhsLevel \#Env\_
Resource.I.Lhs.LhsLevel = \#Env\_Resource.I.Rhs.RhsLevel
\#Env\_Position.I.Lhs.LhsLevel = \#Env\_Position.I.Rhs.RhsLevel \#Env\_
Type.O.Lhs.LhsLevel = \#Env\_Type.O.Rhs.RhsLevel
\#Env\_Resource.O.Lhs.LhsLevel = \#Env\_Resource.O.Rhs.RhsLevel
\#Env\_Position.O.Lhs.LhsLevel = \#Env\_Position.O.Rhs.RhsLevel \#Env\_
Type.E.Lhs.LhsLevel = \#Env\_Type.E.Rhs.RhsLevel
\#Env\_Resource.E.Lhs.LhsLevel \#Env\_Resource.E.Rhs.RhsLevel
\#Env\_Position.E.Lhs.LhsLevel \#Env\_Position.E.Rhs.RhsLevel return

\subsection{ARG}\label{arg}

For a definition of the syntax of this instruction, see nnn.

The ARG instruction is a shorter form of the equivalent instruction:
PARSE UPPER ARG template list

\subsection{Assignment}\label{assignment}

Assignment can occur as the result of executing a clause containing an
assignment (see nnn and nnn), or as a result of executing the VALUE
built-in function, or as part of the execution of a PARSE instruction.
Assignment involves an expression and a VAR\_SYMBOL. The value of the
expression is determined; see nnn.

If the VAR\_SYMBOL does not contain a period, or contains only one
period as its last character, the

value is associated with the VAR\_SYMBOL: call Var Set \#Pool,VAR
SYMBOL, `0',Value

Otherwise, a name is derived, see nnn. The value is associated with the
derived name: call Var Set \#Pool,Derived Name,`1',Value

\subsection{CALL}\label{call}

For a definition of the syntax of this instruction, see nnn.

The CALL instruction is used to invoke a routine, or is used to control
trapping of conditions. lf a vrefis specified that value is the name of
the routine to invoke:

if \#Contains (call, vref) then Name = \#Evaluate(vref, var\_symbol)

If a taken\_constant is specified, that name is used. if \#Contains
(call, taken constant) then Name = \#Instance(call, taken constant)

The name is used to invoke a routine, see nnn. If that routine does not
return a result the RESULT and

-RESULT variables become uninitialized: call Var Drop \#Pool, `RESULT',
`0! call Var Drop \#ReservedPool,'.RESULT', `0'

If the routine does return a result that value is assigned to RESULT and
.RESULT. See nnn for an exception to assigning results.

If the routine returns a result and the trace setting is `R' or `I' then
a trace with that result and a tag
'\textgreater\textgreater\textgreater'' shall be produced, associated
with the call instruction.

If a callon\_spec is specified: If \#Contains(call,callon spec) then do
Condition = \#Instance(callon\_spec,callable condition)

\#Instruction.Condition.\#Level = `CALL' If \#Contains(callon spec,
`OFF') then \#Enabling.Condition.\#Level = `OFF' else
\#Enabling.Condition.\#Level = `ON'

/* Note whether NAME supplied. */ If Contains (callon spec,taken
constant) then Name = \#Instance (callable condition, taken\_constant)

else

Name = Condition \#TrapName.Condition.\#Level = Name end

\subsection{Command to the
configuration}\label{command-to-the-configuration}

For a definition of the syntax of a command, see nnn. A command that is
not part of an ADDRESS instruction is processed in the ACTIVE
environment.

Command = \#Evaluate(command, expression)

if \#Tracing.\#Level == `C' \textbar{} \#Tracing.\#Level == `A' then
call \#Trace `\textgreater\textgreater\textgreater!'

call EnvAssign TRANSIENT, \#Level, ACTIVE, \#Level

call CommandiIssue Command

Commandlssue is also used to describe the ADDRESS instruction:

CommandIssue: parse arg Cmd /* Issues the command, requested environment
is TRANSIENT \emph{/ /} This description does not require the command
processor to understand stems, so it uses an altered environment. */
call EnvAssign PASSED, \#Level, TRANSIENT, \#Level EnvTail =
`TRANSIENT.'\#Level

/* Note the command input. \emph{/ if \#Env\_Type.I.EnvTail = `STEM'
then do /} Check reasonableness of the stem. \emph{/ Stem =
\#Env\_Resource.I.EnvTail Lines = value(Stem'0') if \datatype(Lines,`W')
then call \#Raise `SYNTAX',54.1,Stem'0', Lines if Lines\textless0 then
call \#Raise `SYNTAX',54.1,Stem'0', Lines /} Use a stream for the stem
*/ \#Env\_ Type.I.PASSED.\#Level = `STREAM' call Config Stream Unique
InputStream = \#Outcome \#Env\_Resource.1I.PASSED.\#Level = InputStream
call charout InputStream , vl do j = 1 to Lines call lineout
InputStream, value(Stem \textbar\textbar{} j) end j call lineout
InputStream end

/* Note the command output. */

if \#Env\_Type.O.EnvTail = `STEM' then do Stem =
\#Env\_Resource.O.EnvTail if \#Env\_Position.O.EnvTail == `APPEND' then
do

/* Check that Stem.0 will accept incrementing. \emph{/ Lines=value
(Stem'0'); if \datatype(Lines,`W') then call \#Raise
`SYNTAX',54.1,Stem'0', Lines if Lines\textless0 then call \#Raise
`SYNTAX',54.1,Stem'0', Lines end else call value Stem'0',O /} Use a
stream for the stem */ \#Env\_Type.O.PASSED.\#Level = `STREAM' call
Config Stream Unique \#Env\_Resource.O.PASSED.\#Level = \#Outcome end

/* Note the command error stream. */

if \#Env\_Type.E.EnvTail = `STEM' then do Stem =
\#Env\_Resource.E.EnvTail if \#Env\_Position.E.EnvTail == `APPEND' then
do

/* Check that Stem.0 will accept incrementing. */ Lines=value (Stem'0');
if \datatype(Lines,`W') then call \#Raise `SYNTAX',54.1,Stem'0', Lines
if Lines\textless0 then call \#Raise `SYNTAX',54.1,Stem'0', Lines

end else call value Stem'0',0O /* Use a stream for the stem */ \#Env\_
Type.E.PASSED.\#Level = `STREAM' call Config Stream Unique
\#Env\_Resource.E.PASSED.\#Level = \#Outcome end

\#API Enabled = `1'

call Var\_Reset \#Pool

/* Specifying PASSED here implies all the

components of that environment. */

\#Response = Config Command (PASSED, Cmd) \#Indicator = left
(\#Response,1) Description = substr (\#Response, 2)

\#API Enabled = `0'

/* Recognize success and failure. */ if \#AtPause = 0 then do

call value `RC', \#RC

call var Set 0, `.RC', 0, \#RC end

select when \#Indicator==`N' then Temp=0

when \#Indicator==`F' then Temp=-1 /* Failure \emph{/ when
\#Indicator==`E' then Temp=1 /} Error \emph{/ end call Var Set 0, `.RS',
0, Temp /} Process the output \emph{/ if \#Env\_Type.O.EnvTail=`STEM'
then do /} get output into stem. \emph{/ Stem =
\#Env\_Resource.0O.EnvTail OutputStream =
\#Env\_Resource.0O.PASSED.\#Level do while lines (OutputStream)
\textgreater{} 0 call value Stem'0O',value(Stem'0')4+1 call value
Stem\textbar{} \textbar value(Stem'0'),linein (OutputStream) end end /}
Stemmed Output \emph{/ if \#Env\_Type.E.EnvTail=`STEM' then do /} get
error output into stem. */ Stem = \#Env\_Resource.E.EnvTail OutputStream
= \#Env\_Resource.E.PASSED.\#Level do while lines (OutputStream)
\textgreater{} 0 call value Stem'0O',value(Stem'0')4+1 call value
Stem\textbar{} \textbar value(Stem'0'),linein (OutputStream)

end end /* Stemmed Error output */ if \#Indicator == `N' \&
pos(\#Tracing.\#Level, `CAIR') \textgreater{} 0 then call \#Trace `+++'
if (\#Indicator == `N' \& \#Tracing.\#Level==`E'), \textbar{}
(\#Indicator==`F' \& (\#Tracing.\#Level==`F' \textbar{}
\#Tracing.\#Level==`N')) then do

call \#Trace `\textgreater\textgreater\textgreater!' call \#Trace `+++!'

end \#Condition=`FAILURE' if \#Indicator=`F' \&
\#Enabling.\#Condition.\#Level == `OFF' then call \#Raise `FAILURE' ,
Cmd else if \#Indicator=`E' \textbar{} \#Indicator=`F' then call \#Raise
`ERROR', Cmd

return /* From CommandIssue */

The configuration may choose to perform the test for message 54.1 before
or after issuing the command.

\subsection{DO}\label{do}

For a definition of the syntax of this instruction, see nnn.

The DO instructions is used to group instructions together and
optionally to execute them repeatedly. Executing a do\_simple has the
same effect as executing a nop, except in its trace output. Executing
the do\_ending associated with a do\_simple has the same effect as
executing a nop, except in its trace output.

A do\_instruction that does not contain a do\_simple is equivalent,
except for trace output, to a sequence of instructions in the following
order.

\#Loop = \#Loop+1 \#Iterate.\#Loop = \#Clause (IterateLabel)
\#Once.\#Loop = \#Clause (OnceLabel) \#Leave.\#Loop = \#Clause
(LeaveLabel) if \#Contains (do specification,assignment) then
\#Identity.\#Loop = \#Instance(assignment, VAR SYMBOL) if \#Contains (do
specification, repexpr) then if \datatype(repexpr,`W') then call \#Raise
`SYNTAX', 26.2,repexpr else do \#Repeat.\#Loop = repexpr+0 if
\#Repeat.\#Loop\textless0 then call \#Raise
`SYNTAX',26.2,\#Repeat.\#Loop end if \#Contains (do
specification,assignment) then do \#StartValue.\#Loop = \#Evaluate
(assignment, expression) if datatype (\#StartValue.\#Loop) == `NUM' then
call \#Raise `SYNTAX', 41.6, \#StartValue.\#Loop \#StartValue.\#Loop =
\#StartValue.\#Loop + 0 if \#Contains (do specification,byexpr) then
\#By.\#Loop = 1 end

The following three assignments are made in the order in which `TO'``,
`BY' and `FOR' appear in docount; see nnn.

if \#Contains (do specification, toexpr) then do if datatype(toexpr) ==
`NUM' then call \#Raise `SYNTAX', 41.4, toexpr \#To.\#LOop = toexpr+0 if
\#Contains (do specification, byexpr) then do if datatype (byexpr)
==`NUM' then call \#Raise `SYNTAX', 41.5, byexpr \#By.\#Loop = byexpr+0
if \#Contains (do specification, forexpr) then do if \datatype(forexpr,
`W') then call \#Raise `SYNTAX', 26.3, forexpr \#For.\#Loop = forexpr+0
if \#For.\#Loop \textless0 then call \#Raise `SYNTAX', 26.3,
\#For.\#Loop end if \#Contains (do specification,assignment) then do
call value \#Identity.\#Loop, \#StartValue.\#Loop end if \#Contains (do
specification, `OVER') then do Value = \#Evaluate(dorep, expression)
\#OverArray.\#Loop = Value \textasciitilde{} makearray

\#Repeat.\#Loop = \#OverArray\textasciitilde items /* Count this
downwards as if repexpr. */ \#Iidentity.\#Loop = \#Instance(dorep, VAR
SYMBOL) end

call \#Goto \#Once.\#Loop /* to OnceLabel */

IterateLabel:

if \#Contains (do specification, untilexpr) then do Value =
\#Evaluate(untilexp, expression)

if Value == `1' then leave if Value == `0' then call \#Raise `SYNTAX',
34.4, Value end

if \#Contains (do specification, assignment) then do t = value
(\#Identity. \#Loop)

if \#Indicator == `D' then call \#Raise `NOVALUE', \#Identity.\#Loop
call value \#Identity.\#Loop, t + \#By.\#Loop end

OnceLabel:

if \#Contains (do specification, toexpr) then do if
\#By.\#Loop\textgreater=0 then do if value(\#Identity.\#Loop)
\textgreater{} \#To.\#Loop then leave end else do if
value(\#Identity.\#Loop) \textless{} \#To.\#Loop then leave end end

if \#Contains(dorep, repexpr) \textbar{} \#Contains(dorep, `OVER') then
do if \#Repeat.\#Loop = 0 then leave \#Repeat.\#Loop = \#Repeat.\#Loop-1
if \#Contains(dorep, `OVER') then call value \#Identity.\#Loop,
\#OverArray{[}\#OverArray\textasciitilde items - \#Repeat.\#Loop{]} end
if \#Contains (do specification, forexpr) then do if \#For.\#Loop = 0
then leave \#For.\#Loop = \#For.\#Loop - 1 end if \#Contains (do
specification, whileexpr) then do Value = \#Evaluate(whileexp,
expression)

if Value == `0' then leave if Value == `1' then call \#Raise `SYNTAX',
34.3, Value end \#Execute (do instruction, instruction list) TraceOfEnd:
call \#Goto \#Iterate.\#Loop /* to IterateLabel */ LeaveLabel:

\#Loop = \#Loop - 1

\subsection{DO loop tracing}\label{do-loop-tracing}

When clauses are being traced by \#TraceSource, due to
pos(\#Tracing.\#Level, `AIR') \textgreater{} 0, the DO instruction shall
be traced when it is encountered and again each time the IterateLabel
(see nnn) is encountered. The END instruction shall be traced when the
TraceOfEnd label is encountered.

When expressions or intermediates are being traced they shall be traced
in the order specified by nnn. Hence, in the absence of conditions
arising, those executed prior to the first execution of OnceLabel shall
be shown once per execution of the DO instruction; others shall be shown
depending on the outcome of the tests.

The code in the DO description: t = value (\#Identity. \#Loop) if
\#Indicator == `D' then call \#Raise `NOVALUE', \#Identity.\#Loop call
value \#Identity.\#Loop, t + \#By.\#Loop

represents updating the control variable of the loop. That assignment is
subject to tracing, and other expressions involving state variables are
not. When tracing intermediates, the BY value will have a tag of
`\textgreater+\textgreater{}',

\subsection{DROP}\label{drop}

For a definition of the syntax of this instruction, see nnn.

The DROP instruction restores variables to an uninitialized state.

The words of the variable\_list are processed from left to right.

A word which is a VAR\_SYMBOL, not contained in parentheses, specifies a
variable to be dropped. If VAR\_SYMBOL does not contain a period, or has
only a single period as its last character, the variable

associated with VAR\_SYMBOL by the variable pool is dropped:

\#Response = Var Drop (\#Pool,VAR\_ SYMBOL, `0')

If VAR\_SYMBOL has a period other than as the last character, the
variable associated with VAR\_SYMBOL by the variable pool is dropped by:

\#Response = Var Drop (\#Pool,VAR SYMBOL, `1')

If the word of the variable\_list is a VAR\_SYMBOL enclosed in
parentheses then the value of the

VAR\_SYMBOL is processed. The value is considered in uppercase: \#Value
= Config Upper (\#Value)

Each word in that value found by the WORD built-in function, from left
to right, is subjected to this process:

If the word does not have the syntax of VAR\_SYMBOL a condition is
raised: call \#Raise `SYNTAX', 20.1, word

Otherwise the VAR\_SYMBOL indicated by the word is dropped, as if that
VAR\_SYMBOL were a word of the variable\_list.

\subsection{EXIT}\label{exit}

For a definition of the syntax of this instruction, see nnn.

The EXIT instruction is used to unconditionally complete execution of a
program.

Any expression is evaluated:

if \#Contains(exit, expression) then Value = \#Evaluate(exit,
expression) \#Level = 1

\#Pool = \#Pooll

The opportunity is provided for a final trap. \#API Enabled = `1' call
Var\_Reset \#Pool call Config Termination \#API Enabled = `0'

The processing of the program is complete. See nnn for what API Start
returns as the result.

If the normal sequence of execution ``falls through'' the end of the
program; that is, would execute a further statement if one were appended
to the program, then the program is terminated in the same manner as an
EXIT instruction with no argument.

\subsection{EXPOSE}\label{expose}

The expose instruction identifies variables that are not local to the
method.

We need a check that this starts method; similarities with PROCEDURE.

For a definition of the syntax of this instruction, see nnn.

It is used at the start of a method, after method initialization, to
make variables in the receiver's pool

accessible: if \#AllowExpose then call \#Raise `SYNTAX', 17.2

The words of the variable\_list are processed from left to right.

A word which is a VAR\_SYMBOL, not contained in parentheses, specifies a
variable to be made accessible. If VAR\_SYMBOL does not contain a
period, or has only a single period as its last character, the variable
associated with VAR\_SYMBOL by the variable pool (as a non-tailed name)
is given the

attribute `exposed'. call Var\_ Expose \#Pool, VAR SYMBOL, `0'

If VAR\_SYMBOL has a period other than as last character, the variable
associated with VAR\_SYMBOL

in the variable pool ( by the name derived from VAR\_SYMBOL, see nnn) is
given the attribute `exposed'. call Var\_ Expose \#Pool, Derived Name,
`1'

If the word from the variable\_list is a VAR\_SYMBOL enclosed in
parentheses then the VAR\_SYMBOL is exposed, as if that VAR\_SYMBOL was
a word in the variable\_list. The value of the VAR\_SYMBOL is

processed. The value is considered in uppercase: \#Value = Config Upper
(\#Value)

Each word in that value found by the WORD built-in function, from left
to right, is subjected to this process:

If the word does not have the syntax of VAR\_SYMBOL a condition is
raised: call \#Raise `SYNTAX', 20.1, word

Otherwise the VAR\_SYMBOL indicated by the word is exposed, as if that
VAR\_SYMBOL were a word of the variable\_list.

\subsection{FORWARD}\label{forward}

For a definition of the syntax of this instruction, see nnn.

The FORWARD instruction is used to send a message based on the current
message. if \#Contains (forward, `ARRAY') \& \#Contains(forward,
`ARGUMENTS') then call \#Raise `SYNTAX', nn.n

\subsection{GUARD}\label{guard}

For a definition of the syntax of this instruction, see nnn.

The GUARD instruction is used to conditionally delay the execution of a
method. do forever Value = \#Evaluate( guard, expression)

if Value == `1' then leave

if Value == `0' then call \#Raise `SYNTAX', 34.7, Value Drop exclusive
access and wait for change

end \#\#\# IF

For a definition of the syntax of this instruction, see nnn.

The IF instruction is used to conditionally execute an instruction, or
to select between two alternatives. The expression is evaluated. If the
value is neither `0' nor `1' error 34.1 occurs. If the value is `1', the
instruction in the then is executed. If the value is `0' and e/se is
specified, the instruction in the else is executed.

In the former case, if tracing clauses, the clause consisting of the
THEN keyword shall be traced in addition to the instructions.

In the latter case, if tracing clauses, the clause consisting of the
ELSE keyword shall be traced in addition to the instructions.

\subsection{INTERPRET}\label{interpret}

For a definition of the syntax of this instruction, see nnn.

The INTERPRET instruction is used to execute instructions that have been
built dynamically by evaluating an expression.

The expression is evaluated.

The HALT condition is tested for, and may be raised, in the same way it
is tested at clause termination, see nnn.

The process of syntactic recognition described in clause 6 is applied,
with Config\_SourceChar obtaining its results from the characters of the
value, in left-to-right order, without producing any EOL or EOS events.
When the characters are exhausted, the event EOL occurs, followed by the
event EOS.

If that recognition would produce any message then the interpret raises
the corresponding `SYNTAX' condition.

If the program recognized contains any LABELs then the interpret raises
a condition: call \#Raise `SYNTAX',47.1,Label

where Label is the first LABEL in the program.

Otherwise the instruction\_list in the program is executed.

\subsection{ITERATE}\label{iterate}

For a definition of the syntax of this instruction, see nnn.

The ITERATE instruction is used to alter the flow of control within a
repetitive DO. For a definition of the nesting correction, see nnn.

\#Loop = \#Loop - NestingCorrection call \#Goto \#Iterate.\#Loop

\subsection{Execution of labels}\label{execution-of-labels}

The execution of a label has no effect, other than clause termination
activity and any tracing. if \#Tracing.\#Level==`L' then call
\#TraceSource

\subsection{LEAVE}\label{leave}

For a definition of the syntax of this instruction, see nnn.

The LEAVE instruction is used to immediately exit one or more repetitive
DOs. For a definition of the nesting correction, see nnn.

\#Loop = \#Loop - NestingCorrection call \#Goto \#Leave.\#Loop

\subsection{Message term}\label{message-term}

We can do this by reference to method invokation, just as we do CALL by
reference to invoking a function.

\subsection{LOOP}\label{loop}

Shares most of it's definition with repetitive DO.

\subsection{NOP}\label{nop}

For a definition of the syntax of this instruction, see nnn. The NOP
instruction has no effect other than the effects associated with all
instructions.

\subsection{NUMERIC}\label{numeric}

For a definition of the syntax of this instruction, see nnn. The NUMERIC
instruction is used to change the way in which arithmetic operations are
carried out.

\subsubsection{NUMERIC DIGITS}\label{numeric-digits}

For a definition of the syntax of this instruction, see nnn.

NUMERIC DIGITS controls the precision under which arithmetic operations
and arithmetic built-in functions will be evaluated.

if \#Contains(numericdigits, expression) then

Value = \#Evaluate(numericdigits, expression) else Value = 9

if \datatype(Value,`W') then

call \#Raise `SYNTAX',26.5,Value Value = Value + 0 if
Value\textless=\#Fuzz.\#Level then

call \#Raise `SYNTAX',33.1,Value if Value\textgreater\#Limit Digits then

call \#Raise `SYNTAX',33.2,Value \#Digits.\#Level = Value

\subsubsection{NUMERIC FORM}\label{numeric-form}

For a definition of the syntax of this instruction, see nnn.

NUMERIC FORM controls which form of exponential notation is to be used
for the results of operations and arithmetic built-in functions.

The value of form is either taken directly from the SCIENTIFIC or
ENGINEERING keywords, or by

evaluating valueexp . if \#Contains (numeric,numericsuffix) then

Value = `SCIENTIFIC' else if \#Contains (numericformsuffix,
`SCIENTIFIC') then Value = `SCIENTIFIC' else if \#Contains
(numericformsuffix, `ENGINEERING') then Value = `ENGINEERING' else do
Value = \#Evaluate (numericformsuffix,valueexp) Value = translate (left
(Value,1)) select when Value == `S' then Value = `SCIENTIFIC' when Value
== `E' then Value = `ENGINEERING' otherwise call \#Raise
`SYNTAX',33.3,Value end end \#Form.\#Level = Value

\subsubsection{NUMERIC FUZZ}\label{numeric-fuzz}

For a definition of the syntax of this instruction, see nnn. NUMERIC
FUZZ controls how many digits, at full precision, will be ignored during
a numeric comparison.

If \#Contains (numericfuzz,expression) then Value = \#Evaluate
(numericfuzz,expression) else Value = 0 If \datatype(Value,`W') then
call \#Raise `SYNTAX',26.6,Value Value = Value+0 If Value \textless{} 0
then call \#Raise `SYNTAX',26.6,Value If Value \textgreater=
\#Digits.\#Level then call \#Raise `SYNTAX',33.1,\#Digits.\#Level,Value
\#Fuzz.\#Level = Value

\subsubsection{OPTIONS}\label{options}

For a definition of the syntax of this instruction, see nnn.

The OPTIONS instruction is used to pass special requests to the language
processor.

The expression is evaluated and the value is passed to the language
processor. The language processor treats the value as a series of blank
delimited words. Any words in the value that are not recognized by

the language processor are ignored without producing an error. call
Config Options (Expression)

\subsubsection{PARSE}\label{parse}

For a definition of the syntax of this instruction, see nnn.

The PARSE instruction is used to assign data from various sources to
variables.

The purpose of the PARSE instruction is to select substrings of the
parse\_fype under control of the template\_list. \textbar f the
template\_list is omitted, or a template in the list is omitted, then a
template which is the null string is implied.

Processing for the PARSE instruction begins by constructing a value, the
source to be parsed.

ArgNum = 0 select when \#Contains (parse type, `ARG') then do ArgNum = 1
ToParse = \#Arg.\#Level.ArgNum end when \#Contains (parse type,
`LINEIN') then ToParse = linein('`) when \#Contains (parse type, 'PULL')
then do /* Acquire from external queue or default input. */ \#Response =
Config Pull() if left(\#Response, 1) == `F' then call Config Default
Input ToParse = \#Outcome end when \#Contains (parse type, `SOURCE')
then ToParse = \#Configuration \#HowInvoked \#Source when \#Contains
(parse type, `VALUE') then if \#Contains(parse value, expression) then
ToParse = '\,' else ToParse = \#Evaluate(parse value, expression) when
\#Contains (parse type, `VAR') then ToParse = \#Evaluate (parse var,
VAR\_SYMBOL) when \#Contains (parse type, `VERSION') then ToParse =
\#Version end Uppering = \#Contains(parse, `UPPER') The first template
is associated with this source. If there are further templates, they are
matched against null strings unless `ARG' is specified, when they are
matched against further arguments. The parsing process is defined by the
following routine, ParseData. The template\_list is accessed by
ParseData as a stemmed variable. This variable Template. has elements
which are null strings except

for any elements with tails 1,2,3,\ldots{} corresponding to the tokens
of the template\_list from left to right.

ParseData: /* Targets will be flagged as the template is examined.
\emph{/ Target.=`0' /} Token is a cursor on the components of the
template, moved by FindNextBreak. \emph{/ Token = 1 /} Tok ig a cursor
on the components of the template moved through the target variables by
routine WordParse. */ Tok = 1

do forever /* Until commas dealt with. \emph{/ /} BreakStart and
BreakEnd indicate the position in the source string where there is a
break that divides the source. When the break is a pattern they are the
start of the pattern and the position just beyond it. */ BreakStart =
BreakEnd = 1 SourceEnd = length(ToParse) + 1 If Uppering then ToParse =
translate (ToParse)

do while Template.Tok == '\,' \& Template.Tok == `,'

/* Isolate the data to be processed on this iteration. \emph{/ call
FindNextBreak /} Also marks targets. */

/* Results have been set in DataStart which indicates the start of the
isolated data and BreakStart and BreakEnd which are ready for the next
iteration. Tok has not changed. */

/* If a positional takes the break leftwards from the end of the
previous selection, the source selected is the rest of the string, */

if BreakEnd \textless= DataStart then DataEnd = SourceEnd

else DataEnd = BreakStart

/* Isolated data, to be assigned from: */

Data=substr (ToParse,DataStart, DataEnd-DataStart) call WordParse /*
Does the assignments. */

end /* while \emph{/ if Template.Tok == `,' then leave /} Continue with
next source. */ Token=Token+1 Tok=Token if ArgNum
\textless\textgreater{} 0 then do ArgNum = ArgNum+1 ToParse =
\#Arg.ArgNum end else ToParse='\,' end

return /* from ParseData */

FindNextBreak: do while Template.Token == '\,' \& Template.Token == `,'

Type=left (Template.Token,1) /* The source data to be processed next
will normally start at the end of the break that ended the previous
piece. (However, the relative positionals alter this.) */ DataStart =
BreakEnd select

when Type=`\,``' \textbar{} Type=''\,''' \textbar{} Type=`(' then do if
Type=`(' then do /* A parenthesis introduces a pattern which is not a
constant. */ Token = Token+1 Pattern = value(Template.Token)

if \#Indicator == `D' then call \#Raise `NOVALUE', Template.Token Token
= Token+1 end

else

/* The following removes the outer quotes from the literal pattern
\emph{/ interpret ``Pattern=''Template.Token Token = Token+1 /} Is that
pattern in the remaining source? \emph{/ PatternPos=pos (Pattern,
ToParse,DataStart) if PatternPos\textgreater0 then do /} Selected source
runs up to the pattern. \emph{/ BreakStart=PatternPos
BreakEnd=PatternPos+length (Pattern) return end leave /} The rest of the
source is selected. */ end

when datatype(Template.Token,`W') \textbar{} pos(Type,`+-=')
\textgreater{} 0 then do /* A positional specifies where the relevant
piece of the subject ends. \emph{/ if pos (Type, `+-=') = 0 then do /}
Whole number positional \emph{/ BreakStart = Template.Token Token =
Token+1 end else do /} Other forms of positional. \emph{/
Direction=Template.Token Token = Token + 1 /} For a relative positional,
the position is relative to the start of the previous trigger, and the
source segment starts there. \emph{/ if Direction == `=' then DataStart
= BreakStart /} The adjustment can be given as a number or a variable in
parentheses. */ if Template.Token =`(' then do Token=Token + 1
BreakStart = value(Template. Token) if \#Indicator == `D' then call
\#Raise `NOVALUE', Template.Token Token=Token + 1 end else BreakStart =
Template.Token

if

if

el

\datatype (BreakStart,`W')

then call \#Raise `SYNTAX', 26.4,BreakStart Token = Token+1

Direction=`+'

then BreakStart=DataStart+BreakStart se if Direction=`-'

then BreakStart=DataStart-BreakStart

end /* Adjustment should remain within the ToParse \emph{/ BreakStart =
max(1, BreakStart) BreakStart = min(SourceEnd, BreakStart) BreakEnd =
BreakStart /} No actual literal marks the boundary. */ return

end

when Template.Token == `.' \& pos(Type, `0123456789.')\textgreater0 then
/* A number that isn't a whole number. */

call

\#Raise `SYNTAX', 26.4, Template.Token

/* Raise will not return */

otherwise do /* It is a target, not a pattern */ Target.Token=`1' Token
= Token+1

end

end /*

select */

end /* while \emph{/ /} When no more explicit breaks, break is at the
end of the source. */ DataStart=BreakEnd BreakStart=SourceEnd
BreakEnd=SourceEnd

return

WordParse: /* The names in the template are assigned blank-delimited
values from the source string. */

/* From FindNextBreak */

do while Target.Tok /* Until no more targets for this data. */

/* Last target gets all the residue of the Data.

Next Tok

= Tok + 1

if \Target.NextTok then do call Assign (Data)

leave end /* Not 1 Data = s if Data else do Word =

ast target; assign a word. */ trip (Data, `L') == '\,' then call
Assign('\,')

word (Data,1)

call Assign Word

Data =

substr(Data,length(Word) + 1)

*/

/* The word terminator is not part of the residual data: */ if Data ==
'\,' then Data = substr (Data, 2)

end

Tok = Tok + 1

*/

end Tok=Token /* Next time start on new part of template. \emph{/ return
Assign: if Template.Tok==`.' then Tag=`\textgreater.\textgreater{}' else
do Tag=`\textgreater=\textgreater{}' call value Template.Tok,arg(1) end
/} Arg(1) is an implied argument of the tracing. if \#Tracing.\#Level ==
`R' \textbar{} \#Tracing.\#Level == `I'

return

\subsection{PROCEDURE}\label{procedure}

For a definition of the syntax of this instruction, see nnn.

then call \#Trace Tag The PROCEDURE instruction is used within an
internal routine to protect all the existing variables by making them
unknown to following instructions. Selected variables may be exposed.

It is used at the start of a routine, after routine initialization:

if \#AllowProcedure.\#Level then call \#Raise `SYNTAX', 17.1

\#AllowProcedure.\#Level = 0

/* It introduces a new variable pool: */

call \#Config ObjectNew

call var\_set (\#Outcome,`\#UPPER', `0', \#Pool) /* Previous \#Pool is
upper from the new \#Pool. */

\#Pool=\#O0Outcome

IsProcedure.\#Level=`1'

call Var\_Empty \#Pool

If there is a variable\_list, it provides access to a previous variable
pool.

The words of the variable\_list are processed from left to right.

A word which is a VAR\_SYMBOL, not contained in parentheses, specifies a
variable to be made accessible. If VAR\_SYMBOL does not contain a
period, or has only a single period as its last character, the variable
associated with VAR\_SYMBOL by the variable pool (as a non-tailed name)
is given the

attribute `exposed'. call Var\_ Expose \#Pool, VAR SYMBOL, `0'

If VAR\_SYMBOL has a period other than as last character, the variable
associated with VAR\_SYMBOL

in the variable pool ( by the name derived from VAR\_SYMBOL, see nnn) is
given the attribute `exposed'. call Var\_ Expose \#Pool, Derived Name,
`1'

If the word from the variable\_list is a VAR\_SYMBOL enclosed in
parentheses then the VAR\_SYMBOL is exposed, as if that VAR\_SYMBOL was
a word in the variable\_list. Tne value of the VAR\_SYMBOL is

processed. The value is considered in uppercase: \#Value = Config Upper
(\#Value)

Each word in that value found by the WORD built-in function, from left
to right, is subjected to this process:

If the word does not have the syntax of VAR\_SYMBOL a condition is
raised: call \#Raise `SYNTAX', 20.1, word

Otherwise the VAR\_SYMBOL indicated by the word is exposed, as if that
VAR\_SYMBOL were a word of the variable\_list.

\subsection{PULL}\label{pull}

For a definition of the syntax of this instruction, see nnn.

A PULL instruction is a shorter form of the equivalent instruction:
PARSE UPPER PULL template list

\subsection{PUSH}\label{push}

For a definition of the syntax of this instruction, see nnn. The PUSH
instruction is used to place a value on top of the stack.

If \#Contains(push,expression) then Value = \#Evaluate (push,
expression) else Value = '\,' call Config Push Value

\subsection{QUEUE}\label{queue}

For a definition of the syntax of this instruction, see nnn. The QUEUE
instruction is used to place a value on the bottom of the stack.

If \#Contains (queue,expression) then Value = \#Evaluate (queue,
expression) else Value = '\,' call Config Queue Value

\subsection{RAISE}\label{raise}

The RAISE instruction returns from the current method or routine and
raises a condition.

\subsection{REPLY}\label{reply}

The REPLY instruction is used to allow both the invoker of a method, and
the replying method, to continue executing.

Must set up for error of expression on subsequent RETURN. \#\#\# RETURN
For a definition of the syntax of this instruction, see nnn. The RETURN
instruction is used to return control and possibly a result from a
program or internal routine to the point of its invocation. The RETURN
keyword may be followed by an optional expression, which will be
evaluated and returned as a result to the caller of the routine. Any
expression is evaluated: if \#Contains(return,expression) then \#Outcome
= \#Evaluate(return, expression)

else if \#IsFunction.\#Level then call \#Raise `SYNTAX', 45.1,
\#Name.\#Level

At this point the clause termination occurs and then the following:

If the routine started with a PROCEDURE instruction then the associated
pool is taken out of use: if \#IsProcedure.\#Level then \#Pool = \#Upper
A RETURN instruction which is interactively entered at a pause point
leaves the pause point. if \#Level = \#AtPause then \#AtPause = 0 The
activity at this level is complete: \#Level = \#Level-1 \#NewLevel =
\#Level+1 If \#Level is not zero, the processing of the RETURN
instruction and the invocation is complete. Otherwise processing of the
program is completed:

The opportunity is provided for a final trap. \#API Enabled = `1'

call Var\_Reset \#Pool

call Config Termination

\#API Enabled = `0'

The processing of the program is complete. See nnn for what API Start
returns as the result. \#\#\# SAY

For a definition of the syntax of this instruction, see nnn.

The SAY instruction is used to write a line to the default output
stream.

If \#Contains(say,expression) then Value = Evaluate (say, expression)
else Value = '\,' call Config Default Output Value

\subsection{SELECT}\label{select}

For a definition of the syntax of this instruction, see nnn.

The SELECT instruction is used to conditionally execute one of several
alternative instructions. When tracing, the clause containing the
keyword SELECT is traced at this point.

The \#Contains(select\_body, when) test in the following description
refers to the items of the optional when repetition in order:

LineNum = \#LineNumber Ending = \#Clause (EndLabel) Value=\#Evaluate
(select body, expression) /* In the required WHEN \emph{/ if Value ==
`1' \& Value == `0' then call \#Raise `SYNTAX',34.2,Value If Value==`1'
then call \#Execute when, instruction else do do while \#Contains
(select body, when) Value = \#Evaluate (when, expression) If Value==`1'
then do call \#Execute when, instruction call \#Goto Ending end if Value
== `0' then call \#Raise `SYNTAX', 34.2, Value end /} Of each when */

If \#Contains(select body, `OTHERWISE') then call \#Raise `SYNTAX', 7.3,
LineNum If \#Contains (select body, instruction list) then call
\#Execute select body, instruction list end EndLabel:

When tracing, the clause containing the END keyword is traced at this
point.

\subsection{SIGNAL}\label{signal}

For a definition of the syntax of this instruction, see nnn.

The SIGNAL instruction is used to cause a change in the flow of control
or is used with the ON and OFF keywords to control the trapping of
conditions.

If \#Contains (signal,signal spec) then do Condition = \#Instance(signal
spec,condition)

\#Instruction.Condition.\#Level = `SIGNAL' If \#Contains (signal spec,
`OFF') then \#Enabling.Condition.\#Level = `OFF' else
\#Enabling.Condition.\#Level = `ON'

If Contains (signal spec,taken constant) then Name = \#Instance
(condition, taken constant)

else

Name = Condition \#TrapName.Condition.\#Level = Name end

If there was a signal\_spec this complete the processing of the signal
instruction. Otherwise: if \#Contains (signal, valueexp)

then Name \#Evaluate(valueexp, expression)

else Name \#Instance(signal,taken constant)

The Name matches the first LABEL in the program which has that value.
The comparison is made with the `==``' operator.

If no label matches then a condition is raised:

call \#Raise `SYNTAX',16.1, Name

If the name is a trace-only label then a condition is raised: call
\#Raise `SYNTAX', 16.2, Name

If the name matches a label, execution continues at that label after
these settings: \#Loop.\#Level = 0

/* A SIGNAL interactively entered leaves the pause point. */

if \#Level = \#AtPause then \#AtPause = 0

\subsection{TRACE}\label{trace}

For a definition of the syntax of this instruction, see nnn.

The TRACE instruction is used to control the trace setting which in turn
controls the tracing of execution of the program.

The TRACE instruction is ignored if it occurs within the program (as
opposed to source obtained by

Config\_Trace\_Input) and interactive trace is requested
(\#Interactive.\#Level = `1'). Otherwise: \#TraceInstruction = `1' value
= '\,' if \#Contains(trace, valueexp) then Value = \#Evaluate(valueexp,
expression) if \#Contains (trace, taken constant) then Value =
\#Instance (trace, taken constant) if datatype(Value) == `NUM' \&
\datatype(Value,`W') then call \#Raise `SYNTAX', 26.7, Value if
datatype(Value,`W') then do /* Numbers are used for skipping. \emph{/ if
Value\textgreater=0 then \#InhibitPauses = Value else \#InhibitTrace =
-Value end else do if length(Value) = 0 then do \#Interactive.\#Level =
`0' Value = `N' end /} Each question mark toggles the interacting. */ do
while left(Value,1)==`?' \#Interactive.\#Level = \#Interactive.\#Level
Value = substr(Value,2) end if length(Value) = 0 then do Value =
translate( left(Value,1) ) if verify(Value, `ACEFILNOR') \textgreater{}
0 then call \#Raise `SYNTAX', 24.1, Value if Value==`0O' then
\#Interactive.\#Level=`0' end \#Tracing.\#Level = Value end

\subsection{Trace output}\label{trace-output}

If \#NoSource is `1' there is no trace output.

The routines \#TraceSource and \#Trace specify the output that results
from the trace settings. That output is presented to the configuration
by Config\_Trace\_Output as lines. Each line has a clause identifier at
the left, followed by a blank, followed by a three character tag,
followed by a blank, followed by the trace data.

The width of the clause identifier shall be large enough to hold the
line number of the last line in the program, and no larger. The clause
identifier is the source program line number, or all blank if the line
number is the same as the previous line number indicated and no
execution with trace Off has occurred since. The line number is
right-aligned with leading zeros replaced by blank characters.

When input at a pause is being executed (\#AtPause = 0 ), \#Trace does
nothing when the tag is not `+++'.

When input at a pause is being executed, \#TraceSource does nothing. If
\#InhibitTrace is greater than zero, \#TraceSource does nothing except
decrement \#InhibitTrace. Otherwise, unless the current clause is a null
clause, \#TraceSource outputs all lines of the source program which
contain any part of the current clause, with any characters in those
lines which are not part of the current clause and not
other\_blank\_characters replaced by blank characters. The possible
replacement of other\_blank\_characters is defined by the configuration.
The tag is '\emph{-}'', or if the line is not the first line of the
clause. ™,*'. \#Trace output also has a clause identifier and has a tag
which is the argument to the \#Trace invocation. The data is truncated,
if necessary, to \#Limit\_TraceData characters. The data is enclosed by
quotation marks and the quoted data preceded by two blanks. If the data
is truncated, the trailing quote has the three characters `\ldots{}'
appended. \_ when \#Tracing.\#Level is `C' or `E' or `F' or `N' or `A'
and the tag is `\textgreater\textgreater\textgreater{}' then the data is
the value of the command passed to the environment; \_ when the tag is
`+++' then the data is the four characters `RC concatenated with the
character ``\,``; \_ when \#Tracing.\#Level is 'l' or `R' the data is
the most recently evaluated value. Trace output can also appear as the
result of a'SYNTAX' condition occurring, irrespective of the trace
setting. If a'SYNTAX' condition occurs and it is not trapped by SIGNAL
ON SYNTAX, then the clause in error shall be traced, along with a
traceback. A traceback is a display of each active CALL and INTERPRET
instruction, and function invocation, displayed in reverse order of
execution, each with a tag of `+4+'. \#\#\# USE For a definition of the
syntax of this instruction, see nnn. The USE instruction assigns the
values of arguments to variables. Better not say copies since COPY
method has different semantics. The optional VAR\_SYMBOL positions,
positions 1, 2, \ldots, of the instruction are considered from left to
right. If the position has a VAR\_SYMBOL then its value is assigned to:

if \#ArgExists.Position then call Value VAR\_SYMBOL, \#Arg.Position else

Messy because VALUE bif won't DROP and var\_drop needs to know if
compound.

\section{Conditions and Messages}\label{conditions-and-messages}

When an error occurs during execution of a program, an error number and
message are associated with it. The error number has two parts, the
error code and the error subcode. These are the integer and decimal
parts of the error number. Subcodes beginning or ending in zero are not
used.

Error codes in the range 1 to 90 and error subcodes up to .9 are
reserved for errors described here and for future extensions of this
standard.

concatenated with \#RC

Error number 3 is available to report error conditions occuring during
the initialization phase; error number 2 is available to report error
conditions during the termination phase. These are error conditions
recognized by the language processor, but the circumstances of their
detection is outside of the scope of this standard.

The ERRORTEXT built-in function returns the text as initialized in nnn
when called with the `Standard' option. When the `Standard' option is
omitted, implementation-dependent text may be returned.

When messages are issued any message inserts are replaced by actual
values.

The notation for detection of a condition is:

call \#Raise Condition, Arg2, Arg3, Arg4, Arg5, Arg6é

Some of the arguments may be omitted. In the case of condition `SYNTAX'
the arguments are the message number and the inserts for the message. In
other cases the argument is a further description of the condition.

The action of the program as a result of a condition is dependent on any
signa/\_spec and callon\_spec in the program.

\subsection{Raising of conditions}\label{raising-of-conditions}

The routine \#Raise corresponds to raising a condition. In the following
definition, the instructions containing SIGNAL VALUE and INTERPRET
denote transfers of control in the program being processed. The
instruction EXIT denotes termination. If not at an interactive pause,
this will be termination of the program, see nnn, and there will be
output by Config\_Trace\_Output of the message (with prefix \_ see nnn)
and tracing (see nnn). If at an interactive pause (\#AtPause = 0), this
will be termination of the interpretation of the interactive input;
there will be output by Config\_Trace\_Output of the message (without
traceback) before continuing. The description of the continuation is in
nnn after the ``interpret \#Outcome'' instruction.

The instruction ``interpret `CALL' \#TrapName.\#Condition.\#Level''
below does not set the variables RESULT and .RESULT; any result returned
is discarded.

\#Raise: /* If there is no argument, this is an action which has been
delayed from the time the condition occurred until an appropriate clause
boundary. */ if \arg(1,`E') then do Description = \#PendingDescription.
\#Condition. \#Level Extra = \#PendingExtra.\#Condition. \#Level

end else do \#Condition = arg(1) if \#Condition == `SYNTAX' then do

Description = arg(2) Extra = arg(3) end else do Description =
\#Message(arg(2),arg(3),arg(4),arg(5)) call Var Set \#ReservedPool,
`.MN', 0, arg(2) Extra = '! end end

/* The events for disabled conditions are ignored or cause termination.
*/

if \#Enabling.\#Condition.\#Level == `OFF' \textbar{} \#AtPause = 0 then
do if \#Condition == `SYNTAX' \& \#Condition == `HALT' then return /* To
after use of \#Raise. \emph{/ if \#Condition == `HALT' then Description
= \#Message(4.1, Description) exit /} Terminate with Description as the
message. */ end

/* SIGNAL actions occur as soon as the condition is raised. */

if \#Instruction.\#Condition.\#Level == ```SIGNAL' then do
\#ConditionDescription.\#Level = Description \#ConditionExtra.\#Level =
Extra \#ConditionInstruction.\#Level = `SIGNAL'
\#Enabling.\#Condition.\#Level = `OFF' signal value
\#TrapName.\#Condition.\#Level end

/* All CALL actions are initially delayed until a clause boundary. */

if arg(1,`E') then do /* Events within the handler are not stacked up,
except for one extra HALT while a first is being handled. \emph{/
EventLevel = \#Level if \#Enabling.\#Condition.\#Level == `DELAYED' then
do if \#Condition == `HALT' then return EventLevel =
\#EventLevel.\#Condition. \#Level if \#PendingNow.\#Condition.EventLevel
then return /} Setup a HALT to come after the one being handled. \emph{/
end /} Record a delayed event. \emph{/
\#PendingNow.\#Condition.EventLevel = `1'
\#PendingDescription.\#Condition.EventLevel = Description
\#PendingExtra.\#Condition.EventLevel = Extra
\#Enabling.\#Condition.EventLevel = `DELAYED' return end /} Here for
CALL action after delay. \emph{/ /} Values for the CONDITION built-in
function. \emph{/ \#Condition.\#NewLevel = \#Condition
\#ConditionDescription.\#NewLevel = \#PendingDescription. \#Condition.
\#Level \#ConditionExtra.\#NewLevel = \#PendingExtra.\#Condition.
\#Level \#ConditionInstruction.\#NewLevel = `CALL' interpret `CALL'
\#TrapName.\#Condition.\#Level \#Enabling.\#Condition.\#Level = `ON'
return /} To clause termination */

\subsection{Messages during execution}\label{messages-during-execution}

The state function \#Message corresponds to constructing a message.

This definition is for the message text in nnn. Translations in which
the message inserts are in a different order are permitted.

In addition to the result defined below, the values of MsgNumber and
\#LineNumber shall be shown when a message is output. Also there shall
be an indication of whether the error occurred in code executed at an
interactive pause, see nnn.

Messages are shown by writing them to the default error stream.

\#Message: MsgNumber = arg(1) if \#NoSource then MsgNumber = MsgNumber
\% 1 /* And hence no inserts */ Text = \#ErrorText.MsgNumber Expanded =
'\,'

do Index = 2 parse var Text Begin `\textless{}' Insert `\textgreater{}'
+1 Text

if Insert = '\,' then leave Insert = arg(Index) if length(Insert)
\textgreater{} \#Limit MessageInsert then Insert = left(Insert,\#Limit
MessageInsert)`\ldots{}'

Expanded = Expanded \textbar\textbar{} Begin \textbar\textbar{} Insert

end

Expanded = Expanded \textbar\textbar{} Begin

say Expanded return
