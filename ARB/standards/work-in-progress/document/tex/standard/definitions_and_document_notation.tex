\hypertarget{definitions-and-document-notation}{%
\chapter{Definitions and document
notation}\label{definitions-and-document-notation}}

Lots more for NetRexx

\hypertarget{definitions}{%
\section{Definitions}\label{definitions}}

\hypertarget{application-programming-interface}{%
\subsection{application programming
interface:}\label{application-programming-interface}}

A set of functions which allow access to some Rexx facilities from
non-Rexx programs.

\hypertarget{arguments}{%
\subsection{arguments:}\label{arguments}}

The expressions (separated by commas) between the parentheses of a
function call or following the name on a CALL instruction. Also the
corresponding values which may be accessed by a function or routine,
however invoked.

\hypertarget{built-in-function}{%
\subsection{built-in function:}\label{built-in-function}}

A function (which may be called as a subroutine) that is defined in
section nnn of this standard and can be used directly from a program.

\hypertarget{character-string}{%
\subsection{character string:}\label{character-string}}

A sequence of zero or more characters.

\hypertarget{clause}{%
\subsection{clause:}\label{clause}}

A section of the program, ended by a semicolon. The semicolon may be
implied by the end of a line or by some other constructs.

\hypertarget{coded}{%
\subsection{coded:}\label{coded}}

A coded string is a string which is not necessarily comprised of
characters. Coded strings can occur as arguments to a program, results
of external routines and commands, and the results of some built-in
functions, such as D2C.

\hypertarget{command}{%
\subsection{command:}\label{command}}

A clause consisting of just an expression is an instruction known as a
command. The expression is evaluated and the result is passed as a
command string to some external environment. \#\#\# condition:

A specific event, or state, which can be trapped by CALL ON or SIGNAL
ON.

\hypertarget{configuration}{%
\subsection{configuration:}\label{configuration}}

Any data-processing system, operating system and software used to
operate a language processor.

\hypertarget{conforming-language-processor}{%
\subsection{conforming language
processor:}\label{conforming-language-processor}}

A language processor which obeys all the provisions of this standard.

\hypertarget{construct}{%
\subsection{construct:}\label{construct}}

A named syntax grouping, for example ``expression'', ``do\_
specification''.

\hypertarget{default-error-stream}{%
\subsection{default error stream:}\label{default-error-stream}}

An output stream, determined by the configuration, on which error
messages are written.

\hypertarget{default-input-stream}{%
\subsection{default input stream:}\label{default-input-stream}}

An input stream having a name which is the null string. The use of this
stream may be implied.

\hypertarget{default-output-stream}{%
\subsection{default output stream:}\label{default-output-stream}}

An output stream having a name which is the null string. The use of this
stream may be implied.

\hypertarget{direct-symbol}{%
\subsection{direct symbol:}\label{direct-symbol}}

A symbol which, without any modification, names a variable in a variable
pool.

\hypertarget{directive}{%
\subsection{directive:}\label{directive}}

Clauses which begin with two colons are directives. Directives are not
executable, they indicate the structure of the program. Directives may
also be written with the two colons implied.

\hypertarget{dropped}{%
\subsection{dropped:}\label{dropped}}

A symbol which is in an unitialized state, as opposed to having had a
value assigned to it, is described as dropped. The names in a variable
pool have an attribute of `dropped' or `not-dropped'.

\hypertarget{encoding}{%
\subsection{encoding:}\label{encoding}}

The relation between a character string anda corresponding number. The
encoding of character strings is determined by the configuration.

\hypertarget{end-of-line}{%
\subsection{end-of-line:}\label{end-of-line}}

An event that occurs during the scanning of a source program. Normally
the end-of-lines will relate to the lines shown if the configuration
lists the program. They may, or may not, correspond to characters in the
source program.

\hypertarget{environment}{%
\subsection{environment:}\label{environment}}

The context in which a command may be executed. This is comprised of the
environment name, details of the resource that will provide input to the
command, and details of the resources that will receive output of the
command.

\hypertarget{environment-name}{%
\subsection{environment name:}\label{environment-name}}

The name of an external procedure or process that can execute commands.
Commands are sent to the current named environment, initially selected
externally but then alterable by using the ADDRESS instruction.

\hypertarget{error-number}{%
\subsection{error number:}\label{error-number}}

A number which identifies a particular situation which has occurred
during processing. The message prose associated with such a number is
defined by this standard.

\hypertarget{exposed}{%
\subsection{exposed:}\label{exposed}}

Normally, a symbol refers to a variable in the most recently established
variable pool. When this is not the case the variable is referred to as
an exposed variable.

\hypertarget{expression}{%
\subsection{expression:}\label{expression}}

The most general of the constructs which can be evaluated to produce a
single string value.

\hypertarget{external-data-queue}{%
\subsection{external data queue:}\label{external-data-queue}}

A queue of strings that is external to REXX programs in that other
programs may have access

to the queue whenever REXX relinquishes control to some other program.

\hypertarget{external-routine}{%
\subsection{external routine:}\label{external-routine}}

A function or subroutine that is neither built-in nor in the same
program as the CALL instruction or function call that invokes it.

\hypertarget{external-variable-pool}{%
\subsection{external variable pool:}\label{external-variable-pool}}

A named variable pool supplied by the configuration which can be
accessed by the VALUE built-in function.

\hypertarget{function}{%
\subsection{function:}\label{function}}

Some processing which can be invoked by name and will produce a result.
This term is used for both Rexx functions (See nnn) and functions
provided by the configuration (see n).

\hypertarget{identifier}{%
\subsection{identifier:}\label{identifier}}

The name of a construct.

\hypertarget{implicit-variable}{%
\subsection{implicit variable:}\label{implicit-variable}}

A tailed variable which is in a variable pool solely as a result of an
operation on its stem. The names in a variable pool have an attribute of
`implicit' or `not-implicit'.

\hypertarget{instruction}{%
\subsection{instruction:}\label{instruction}}

One or more clauses that describe some course of action to be taken by
the language processor.

\hypertarget{internal-routine}{%
\subsection{internal routine:}\label{internal-routine}}

A function or subroutine that is in the same program as the CALL
instruction or function call that invokes it.

\hypertarget{keyword}{%
\subsection{keyword:}\label{keyword}}

This standard specifies special meaning for some tokens which consist of
letters and have particular spellings, when used in particular contexts.
Such tokens, in these contexts, are keywords.

\hypertarget{label}{%
\subsection{label:}\label{label}}

A clause that consists of a single symbol or a literal followed by a
colon.

\hypertarget{language-processor}{%
\subsection{language processor:}\label{language-processor}}

Compiler, translator or interpreter working in combination with a
configuration.

\hypertarget{notation-function}{%
\subsection{notation function:}\label{notation-function}}

A function with the sole purpose of providing a notation for describing
semantics, within this standard. No Rexx program can invoke a notation
function.

\hypertarget{null-clause}{%
\subsection{null clause:}\label{null-clause}}

A clause which has no tokens.

\hypertarget{null-string}{%
\subsection{null string:}\label{null-string}}

A character string with no characters, that is, a string of length zero.

\hypertarget{production}{%
\subsection{production:}\label{production}}

The definition of a construct, in Backus-Naur form.

\hypertarget{return-code}{%
\subsection{return code:}\label{return-code}}

A string that conveys some information about the command that has been
executed. Return codes usually indicate the success or failure of the
command but can also be used to represent other information.

\hypertarget{routine}{%
\subsection{routine:}\label{routine}}

Some processing which can be invoked by name.

\hypertarget{state-variable}{%
\subsection{state variable:}\label{state-variable}}

A component of the state of progress in processing a program, described
in this standard by a named variable. No Rexx program can directly
access a state variable.

\hypertarget{stem}{%
\subsection{stem:}\label{stem}}

If a symbol naming a variable contains a period which is not the first
character, the part of the symbol up to and including the first period
is the stem.

\hypertarget{stream}{%
\subsection{stream:}\label{stream}}

Named streams are used as the sources of input and the targets of
output. The total semantics of such a stream are not defined in this
standard and will depend on the configuration. A stream may be a
permanent file in the configuration or may be something else, for
example the input from a keyboard.

\hypertarget{string}{%
\subsection{string:}\label{string}}

For many operations the unit of data is a string. It may, or may not, be
comprised of a sequence of characters which can be accessed
individually.

\hypertarget{subcode}{%
\subsection{subcode:}\label{subcode}}

The decimal part of an error number.

\hypertarget{subroutine}{%
\subsection{subroutine:}\label{subroutine}}

An internal, built-in, or external routine that may or may not return a
result string and is invoked by the CALL instruction. If it returns a
result string the subroutine can also be invoked by a function call, in
which case it is being called as a function. \#\#\# symbol:

A sequence of characters used as a name, see nnn. Symbols are used to
name variables, functions, etc.

\hypertarget{tailed-name}{%
\subsection{tailed name:}\label{tailed-name}}

The names in a variable pool have an attribute of `tailed' or
`non-tailed'. Otherwise identical names are distinct if their attributes
differ. Tailed names are normally the result of replacements in the tail
of a symbol, the part that follows a stem.

\hypertarget{token}{%
\subsection{token:}\label{token}}

The unit of low-level syntax from which high-level constructs are built.
Tokens are literal strings, symbols, operators, or special characters.

\hypertarget{trace}{%
\subsection{trace:}\label{trace}}

A description of some or all of the clauses of a program, produced as
each is executed.

\hypertarget{trap}{%
\subsection{trap:}\label{trap}}

A function provided by the user which replaces or augments some normal
function of the language processor.

\hypertarget{variable-pool}{%
\subsection{variable pool:}\label{variable-pool}}

A collection of the names of variables and their associated values.

\hypertarget{document-notation}{%
\section{Document notation}\label{document-notation}}

\hypertarget{rexx-code}{%
\subsection{Rexx Code}\label{rexx-code}}

Some Rexx code is used in this standard. This code shall be assumed to
have its private set of variables. Variables used in this code are not
directly accessible by the program to be processed. Comments in the code
are not part of the provisions of this standard.

\hypertarget{italics}{%
\subsection{Italics}\label{italics}}

Throughout this standard, except in Rexx code, references to the
constructs defined in section nnn are italicized.
