%preprocessed texin
\chapter{Rationale}\label{rationale}

This annex explains some of the decisions made by the committee that
drafted this standard, and assists in the understanding of this
document. Some of the statements made here are opinions rather than
facts. These should be interpreted as if prefixed by ``In the opinion of
the X3J18 committee\ldots{}''.

The language described in this standard is, almost entirely, a
compatible extension of the language described by the third reference of
Annex C, which we call ``Classic Rexx''.

The extension allows programs to be written in a less monolithic
fashion; ``Directives'' are introduced to allow one file to contain
several executable units and to allow a program to be written as several
files. The functional extension centers on the addition of objects.
Unlike the individual strings which are the data of Classic Rexx, an
object may be composite. The use of identifiers to reference objects is
an indirect reference, that is two identifiers may refer to the same
object. Classic Rexx avoided any aliasing, even to the extent having
by-reference parameters, to promote simple error free programming. In
the years since Rexx originated the problems tackled by programmers have
become more complex and data structures larger, so that the benefit of
simplicity is outweighed by the power of assignment semantics that are
not simply copying all the data.

Even with the addition of references Rexx remains a typeless language,
in the sense that the programmer need not consider underlying hardware
formats such as LONG or FLOAT representations. Object Rexx does have
classes, which are the hardware independent analogy to types. The class
of an object corresponds to the operations that can be performed upon
it.
