%preprocessed texin
\chapter{Built-in functions}\label{built-in-functions}

\section{Notation}\label{notation}

The built-in functions are defined mainly through code. The code refers
to state variables. This is solely a notation used in this standard.

The code refers to functions with names that start with `Config\_';
these are the functions described in section nnn.

The code is specified as an external routine that produces a result from
the values \#Bif (which is the name of the built-in function),
\#Bif\_Arg.0 (the number of arguments), \#Bif\_Arg.i and
\#Bif\_ArgExists.i (which are the argument data.)

The value of \#Level is the value for the clause which invoked the
built-in function.

The code either returns the result of the built-in or exits with an
indication of a condition that the invocation of the built-in raises.

The code below uses built-in functions. Such a use invokes another use
of this code with a new value of \#Level. On these invocations, the
CheckArgs function is not relevant.

Numeric settings as follows are used in the code. When an argument is
being checked as a number by `NUM' or `WHOLENUM' the settings are those
current in the caller. When an argument is being checked as an integer
by an item containing `WHOLE' the settings are those for the particular
built-in function. Elsewhere the settings have sufficient numeric digits
to avoid values which would require exponential notation.

\section{Routines used by built-in
functions}\label{routines-used-by-built-in-functions}

The routine CheckArgs is concerned with checking the arguments to the
built-in. The routines Time2Date and Leap are for date calculations.
ReRadix is used for radix conversion. The routine Raise raises a
condition and does not return.

\subsection{Argument checking}\label{argument-checking}

\lstinputlisting[language=rexx,label=argumentchecking.rexx,caption=argumentchecking.rexx]{argumentchecking.rexx}

10.2.2 Date calculations

\lstinputlisting[language=rexx,label=datecalculations.rexx,caption=datecalculations.rexx]{datecalculations.rexx}

10.2.1. Radix conversion

\lstinputlisting[language=rexx,label=reradix.rexx,caption=reradix.rexx]{reradix.rexx}

\subsection{Raising the SYNTAX
condition}\label{raising-the-syntax-condition}

\lstinputlisting[language=rexx,label=raise.rexx,caption=raise.rexx]{raise.rexx}

\section{Character built-in
functions}\label{character-built-in-functions}

These functions process characters or words in strings. Character
positions are numbered from one at the left. Words are delimited by
blanks and their equivalents, word positions are counted from one at the
left.

\subsection{ABBREV}\label{abbrev}

\lstinputlisting[language=rexx,label=abbrev.rexx,caption=abbrev.rexx]{abbrev.rexx}

\subsection{CENTER}\label{center}

CENTER returns a string with the first argument centered in it. The
length of the result is the second argument and the third argument
specifies the character to be used for padding.

\lstinputlisting[language=rexx,label=center.rexx,caption=center.rexx]{center.rexx}

\subsection{CENTRE}\label{centre}

This is an alternative spelling for the CENTER built-in function.

\subsection{CHANGESTR}\label{changestr}

CHANGESTR replaces all occurrences of the first argument within the
second argument, replacing them with the third argument.

\lstinputlisting[language=rexx,label=changestr.rexx,caption=changestr.rexx]{changestr.rexx}

\subsection{COMPARE}\label{compare}

COMPARE returns `0' if the first and second arguments have the same
value. Otherwise, the result is the position of the first character that
is not the same in both strings.

\lstinputlisting[language=rexx,label=compare.rexx,caption=compare.rexx]{compare.rexx}

\subsection{COPIES}\label{copies}

COPIES returns concatenated copies of the first argument. The second
argument is the number of copies.

\lstinputlisting[language=rexx,label=copies.rexx,caption=copies.rexx]{copies.rexx}

\subsection{COUNTSTR}\label{countstr}

COUNTSTR counts the appearances of the first argument in the second
argument.

\lstinputlisting[language=rexx,label=counstr.rexx,caption=counstr.rexx]{counstr.rexx}

\subsection{DATATYPE}\label{datatype}

DATATYPE tests for characteristics of the first argument. The second
argument specifies the particular test.

\lstinputlisting[language=rexx,label=datatype.rexx,caption=datatype.rexx]{datatype.rexx}

\subsection{DELSTR}\label{delstr}

DELSTR deletes the sub-string of the first argument which begins at the
position given by the second argument. The third argument is the length
of the deletion.

\lstinputlisting[language=rexx,label=delstr.rexx,caption=delstr.rexx]{delstr.rexx}

\subsection{DELWORD}\label{delword}

DELWORD deletes words from the first argument. The second argument
specifies position of the first word to be deleted and the third
argument specifies the number of words.

\lstinputlisting[language=rexx,label=delword.rexx,caption=delword.rexx]{delword.rexx}

\subsection{INSERT}\label{insert}

Output \textbar\textbar{} substr(String, BeginRight)

INSERT insets the first argument into the second. The third argument
gives the position of the character before the insert and the fourth
gives the length of the insert. The fifth is the padding character.

\lstinputlisting[language=rexx,label=insert.rexx,caption=insert.rexx]{insert.rexx}

\subsection{LASTPOS}\label{lastpos}

/* To left of insert /* New string inserted \emph{/ /} To right of
insert

\lstinputlisting[language=rexx,label=lastpos.rexx,caption=lastpos.rexx]{lastpos.rexx}

\subsection{LEFT}\label{left}

LEFT returns characters that are on the left of the first argument.

length of the result and the third is the padding character.

\lstinputlisting[language=rexx,label=left.rexx,caption=left.rexx]{left.rexx}

\subsection{OVERLAY}\label{overlay}

OVERLAY overlays the first argument onto the second. The third argument
is the starting position of the overlay. The fourth argument is the
length of the overlay and the fifth is the padding character.

\lstinputlisting[language=rexx,label=overlay.rexx,caption=overlay.rexx]{overlay.rexx}

\subsection{REVERSE}\label{reverse}

REVERSE returns its argument, swapped end for end.

\lstinputlisting[language=rexx,label=reverse,caption=reverse]{reverse}

\subsection{RIGHT}\label{right}

RIGHT returns characters that are on the right of the first argument.
The second argument specifies the

length of the result and the third is the padding character.

\lstinputlisting[language=rexx,label=right.rexx,caption=right.rexx]{right.rexx}

\subsection{SPACE}\label{space}

SPACE formats the blank-delimited words in the first argument with pad
characters between each word. The second argument is the number of pad
characters between each word and the third is the pad character.

\lstinputlisting[language=rexx,label=space.rexx,caption=space.rexx]{space.rexx}

\subsection{STRIP}\label{strip}

STRIP removes characters from its first argument. The second argument
specifies whether the deletions are leading characters, trailing
characters or both. Each character deleted is equal to the third
argument, or equivalent to a blank if the third argument is omitted.

\lstinputlisting[language=rexx,label=strip.rexx,caption=strip.rexx]{strip.rexx}

\subsection{SUBSTR}\label{substr}

SUBSTR returns a sub-string of the first argument. The second argument
specifies the position of the first character and the third specifies
the length of the sub-string. The fourth argument is the padding
character.

\lstinputlisting[language=rexx,label=substr.rexx,caption=substr.rexx]{substr.rexx}

\subsection{SUBWORD}\label{subword}

SUBWORD returns a sub-string of the first argument, comprised of words.
The second argument is the position in the first argument of the first
word of the sub-string. The third argument is the number of words in the
sub-string.

\lstinputlisting[language=rexx,label=subword.rexx,caption=subword.rexx]{subword.rexx}

\subsection{TRANSLATE}\label{translate}

TRANSLATE returns the characters of its first argument with each
character either unchanged or translated to another character.

\lstinputlisting[language=rexx,label=translate,caption=translate]{translate}

\subsection{VERIFY}\label{verify}

VERIFY checks that its first argument contains only characters that are
in the second argument, or that it contains no characters from the
second argument; the third argument specifies which check is made. The
result is `0', or the position of the character that failed
verification. The fourth argument is a starting position for the check.

\lstinputlisting[language=rexx,label=verify.rexx,caption=verify.rexx]{verify.rexx}

\subsection{WORD}\label{word}

WORD returns the word from the first argument at the position given by
the second argument.

\lstinputlisting[language=rexx,label=word.rexx,caption=word.rexx]{word.rexx}

\subsection{WORDINDEX}\label{wordindex}

WORDINDEX returns the character position in the first argument of a word
in the first argument. The second argument is the word position of that
word.

\lstinputlisting[language=rexx,label=wordindex.rexx,caption=wordindex.rexx]{wordindex.rexx}

\subsection{WORDLENGTH}\label{wordlength}

WORDLENGTH returns the number of characters in a word from the first
argument. The second argument is the word position of that word.

\lstinputlisting[language=rexx,label=wordlength.rexx,caption=wordlength.rexx]{wordlength.rexx}

\subsection{WORDPOS}\label{wordpos}

WORDPOS finds the leftmost occurrence in the second argument of the
sequence of words in the first argument. The result is `0' or the word
position in the second argument of the first word of the matched
sequence. Third argument is a word position for the start of the search.

\lstinputlisting[language=rexx,label=wordpos.rexx,caption=wordpos.rexx]{wordpos.rexx}

\subsection{WORDS}\label{words}

WORDS counts the number of words in its argument.

\lstinputlisting[language=rexx,label=words.rexx,caption=words.rexx]{words.rexx}

\subsection{XRANGE}\label{xrange}

XRANGE returns an ordered string of all valid character encodings in the
specified range.

\lstinputlisting[language=rexx,label=xrange.rexx,caption=xrange.rexx]{xrange.rexx}

\section{Arithmetic built-in
functions}\label{arithmetic-built-in-functions}

These functions perform arithmetic at the numeric settings current at
the invocation of the built-in function. Note that CheckArgs formats any
`NUM' (numeric) argument.

\subsection{ABS}\label{abs}

\lstinputlisting[language=rexx,label=abs.rexx,caption=abs.rexx]{abs.rexx}

\subsection{FORMAT}\label{format}

FORMAT formats its first argument. The second argument specifies the
number of characters to be used for the integer part and the third
specifies the number of characters for the decimal part. The fourth
argument specifies the number of characters for the exponent and the
fifth determines when exponeniial notation is used.

\lstinputlisting[language=rexx,label=format.rexx,caption=format.rexx]{format.rexx}

\subsection{MAX}\label{max}

MAX returns the largest of its arguments.

\lstinputlisting[language=rexx,label=max.rexx,caption=max.rexx]{max.rexx}

\subsection{MIN}\label{min}

MIN returns the smallest of its arguments.

\lstinputlisting[language=rexx,label=min.rexx,caption=min.rexx]{min.rexx}

\subsection{SIGN}\label{sign}

SIGN returns `1', `0' or `-1' according to whether its argument is
greater than, equal to, or less than zero.

\lstinputlisting[language=rexx,label=sign.rexx,caption=sign.rexx]{sign.rexx}

\subsection{TRUNC}\label{trunc}

TRUNC returns the integer part of its argument, or the integer part plus
a number of digits after the decimal point, specified by the second
argument.

\lstinputlisting[language=rexx,label=trunc.rexx,caption=trunc.rexx]{trunc.rexx}

\section{State built-in functions}\label{state-built-in-functions}

These functions return values from the state of the execution.

\subsection{ADDRESS}\label{address}

ADDRESS returns the name of the environment to which commands are
currently being submitted. Optionally, under control by the argument, it
also returns information on the targets of command output and the source
of command input.

\lstinputlisting[language=rexx,label=address.rexx,caption=address.rexx]{address.rexx}

\subsection{ARG}\label{arg}

ARG returns information about the argument strings to a program or
routine, or the value of one of those strings.

\lstinputlisting[language=rexx,label=arg.rexx,caption=arg.rexx]{arg.rexx}

\subsection{CONDITION}\label{condition}

CONDITION returns information associated with the current condition.

\lstinputlisting[language=rexx,label=condition.rexx,caption=condition.rexx]{condition.rexx}

\subsection{DIGITS}\label{digits}

DIGITS returns the current setting of NUMERIC DIGITS. call CheckArgs
'\,'

\lstinputlisting[language=rexx,label=digits,caption=digits]{digits}

\subsection{ERRORTEXT}\label{errortext}

ERRORTEXT returns the unexpanded text of the message which is identified
by the first argument. A second argument of `S' selects the standard
English text, otherwise the text may be translated to another national
language. This translation is not shown in the code below.

\lstinputlisting[language=rexx,label=errortext.rexx,caption=errortext.rexx]{errortext.rexx}

\subsection{FORM}\label{form}

FORM returns the current setting of NUMERIC FORM.

\lstinputlisting[language=rexx,label=form.rexx,caption=form.rexx]{form.rexx}

\subsection{FUZZ}\label{fuzz}

FUZZ returns the current setting of NUMERIC FUZZ.

\lstinputlisting[language=rexx,label=fuzz.rexx,caption=fuzz.rexx]{fuzz.rexx}

\subsection{SOURCELINE}\label{sourceline}

If there is no argument, SOURCELINE returns the number of lines in the
program, or `0' if the source program is not being shown on this
execution. If there is an argument it specifies the number of the line
of the source program to be returned.

\lstinputlisting[language=rexx,label=sourceline.rexx,caption=sourceline.rexx]{sourceline.rexx}

\subsection{TRACE}\label{trace}

TRACE returns the trace setting currently in effect, and optionally
alters the setting.

\lstinputlisting[language=rexx,label=trace.rexx,caption=trace.rexx]{trace.rexx}

10.4 Conversion built-in functions

Conversions between Binary form, Decimal form, and heXadecimal form do
not depend on the encoding (see nnn) of the character data.

Conversion to Coded form gives a result which depends on the encoding.
Depending on the encoding, the result may be a string that does not
represent any sequence of characters.

\subsection{B2X}\label{b2x}

B2X performs binary to hexadecimal conversion.

\lstinputlisting[language=rexx,label=b2x.rexx,caption=b2x.rexx]{b2x.rexx}

\subsection{BITAND}\label{bitand}

The functions BITAND, BITOR and BITXOR operate on encoded character
data. Each binary digit from the encoding of the first argument is
processed in conjunction with the corresponding bit from the second
argument.

\lstinputlisting[language=rexx,label=bitand.rexx,caption=bitand.rexx]{bitand.rexx}

\subsection{BITOR}\label{bitor}

See nnn

\subsection{BITXOR}\label{bitxor}

See nnn

\subsection{C2D}\label{c2d}

C2D performs coded to decimal conversion.

\lstinputlisting[language=rexx,label=c2d.rexx,caption=c2d.rexx]{c2d.rexx}

10.4.6 C2X C2X performs coded to hexadecimal conversion.

\lstinputlisting[language=rexx,label=c2x.rexx,caption=c2x.rexx]{c2x.rexx}

\subsection{D2C}\label{d2c}

D2C performs decimal to coded conversion.

\lstinputlisting[language=rexx,label=d2c.rexx,caption=d2c.rexx]{d2c.rexx}

\subsection{D2X}\label{d2x}

D2X performs decimal to hexadecimal conversion.

\lstinputlisting[language=rexx,label=d2x.rexx,caption=d2x.rexx]{d2x.rexx}

\subsection{X2B}\label{x2b}

X2B performs hexadecimal to binary conversion.

\lstinputlisting[language=rexx,label=x2b.rexx,caption=x2b.rexx]{x2b.rexx}

\subsection{X2C}\label{x2c}

X2C performs hexadecimal to coded character conversion.

\lstinputlisting[language=rexx,label=x2c.rexx,caption=x2c.rexx]{x2c.rexx}

\subsection{X2D}\label{x2d}

X2D performs hexadecimal to decimal conversion.

\lstinputlisting[language=rexx,label=x2d.rexx,caption=x2d.rexx]{x2d.rexx}

\section{Input/Output built-in
functions}\label{inputoutput-built-in-functions}

The configuration shall provide the ability to access streams. Streams
are identified by character string identifiers and provide for the
reading and writing of data. They shall support the concepts of
characters, lines, and positioning. The input/output built-in functions
interact with one another, and they make use of Config\_ functions, see
nnn. When the operations are successful the following characteristics
shall be exhibited:

\begin{itemize}
\item
  The CHARIN/CHAROUT functions are insensitive to the lengths of the
  arguments. The data written to a stream by CHAROUT can be read by a
  different number of CHARINs.
\item
  The CHARIN/CHAROUT functions are reflective, that is, the
  concatenation of the data read from a persistent stream by CHARIN
  (after positioning to 1, while CHARS(Stream)\textgreater0), will be
  the same as the concatenation of the data put by CHAROUT.
\item
  All characters can be used as CHARIN/CHAROUT data.
\item
  The CHARS(Stream, `N') function will return zero only when a
  subsequent read (without positioning) is guaranteed to raise the
  NOTREADY condition.
\item
  The LINEIN/LINEOUT functions are sensitive to the length of the
  arguments, that is, the length of a line written by LINEOUT is the
  same as the length of the string returned by successful LINEIN of the
  line.
\item
  Some characters, call them line-banned characters, cannot reliably be
  used as data for LINEIN/LINEOUT. If these are not used, LINEIN/LINEOUT
  is reflective. If they are used, the result is not defined. The set of
  characters which are line-barred is a property of the configuration.
\item
  The LINES(Stream, `N') function will return zero only when a
  subsequent LINEIN (without positioning) is guaranteed to raise the
  NOTREADY condition.
\item
  When a persistent stream is repositioned and written to with CHAROUT,
  the previously written data is not lost, except for the data
  overwritten by this latest CHAROUT.
\item
  When a persistent stream is repositioned and written to with LINEOUT,
  the previously written data is not lost, except for the data
  overwritten by this latest LINEOUT, which may leave lines partially
  overwritten.
\end{itemize}

\subsection{CHARIN}\label{charin}

CHARIN returns a string read from the stream named by the first
argument.

\lstinputlisting[language=rexx,label=charin.rexx,caption=charin.rexx]{charin.rexx}

\subsection{CHAROUT}\label{charout}

CHAROUT returns the count of characters remaining after attempting to
write the second argument to the stream named by the first argument.

\lstinputlisting[language=rexx,label=charout.rexx,caption=charout.rexx]{charout.rexx}

\subsection{CHARS}\label{chars}

CHARS indicates whether there are characters remaining in the named
stream. Optionally, it returns a

count of the characters remaining and immediately available.

\lstinputlisting[language=rexx,label=chars.rexx,caption=chars.rexx]{chars.rexx}

\subsection{LINEIN}\label{linein}

LINEIN reads a line from the stream named by the first argument, unless
the third argument is Zero.

\lstinputlisting[language=rexx,label=linein.rexx,caption=linein.rexx]{linein.rexx}

\subsection{LINEOUT}\label{lineout}

LINEOUT returns `1' or `0', indicating whether the second argument has
been successfully written to the stream named by the first argument. A
result of `1' means an unsuccessful write.

\lstinputlisting[language=rexx,label=lineout.rexx,caption=lineout.rexx]{lineout.rexx}

\subsection{LINES}\label{lines}

`EOL'

LINES returns the number of lines remaining in the named stream.

\lstinputlisting[language=rexx,label=lines.rexx,caption=lines.rexx]{lines.rexx}

\subsection{QUALIFY}\label{qualify}

\lstinputlisting[language=rexx,label=qualify.rexx,caption=qualify.rexx]{qualify.rexx}

\subsection{STREAM}\label{stream}

STREAM returns a description of the state of, or the result of an
operation upon, the stream named by the first argument.

\lstinputlisting[language=rexx,label=stream.rexx,caption=stream.rexx]{stream.rexx}

\section{Other built-in functions}\label{other-built-in-functions}

\subsection{DATE}\label{date}

DATE with fewer than two arguments returns the local date. Otherwise it
converts the second argument (which has a format given by the third
argument) to the format specified by the first argument. If there are
fourth or fifth arguments, they describe the treatment of separators
between fields of the date.

\lstinputlisting[language=rexx,label=date.rexx,caption=date.rexx]{date.rexx}

\subsection{QUEUED}\label{queued}

QUEUED returns the number of lines remaining in the external data queue.

\lstinputlisting[language=rexx,label=queued.rexx,caption=queued.rexx]{queued.rexx}

\subsection{RANDOM}\label{random}

RANDOM returns a quasi-random number.

\lstinputlisting[language=rexx,label=random.rexx,caption=random.rexx]{random.rexx}

\subsection{SYMBOL}\label{symbol}

The function SYMBOL takes one argument, which is evaluated. Let String
be the value of that argument. If Config\_Length(String) returns an
indicator `E' then the SYNTAX condition 23.1 shall be raised. Otherwise,
if the syntactic recognition described in section nnn would not
recognize String as a symbol then the result of the function SYMBOL is
'BAD''.

If String would be recognized as a symbol the result of the function
SYMBOL depends on the outcome of accessing the value of that symbol, see
nnn. If the final use of Var\_Value leaves the indicator with value `D'
then the result of the function SYMBOL is `LIT', otherwise `VAR'.

\subsection{TIME}\label{time}

TIME with less than two arguments returns the local time within the day,
or an elapsed time. Otherwise it converts the second argument (which has
a format given by the third argument) to the format specified by the
first argument.

\lstinputlisting[language=rexx,label=time.rexx,caption=time.rexx]{time.rexx}

10.6.1 VALUE VALUE returns the value of the symbol named by the first
argument, and optionally assigns it a new value.

\lstinputlisting[language=rexx,label=value.rexx,caption=value.rexx]{value.rexx}

\subsection{QUEUED}\label{queued-1}

QUEUED returns the number of lines remaining in the external data queue.

\lstinputlisting[language=rexx,label=queued.rexx,caption=queued.rexx]{queued.rexx}

\subsection{RANDOM}\label{random-1}

RANDOM returns a quasi-random number.

\lstinputlisting[language=rexx,label=random.rexx,caption=random.rexx]{random.rexx}

\subsection{SYMBOL}\label{symbol-1}

The function SYMBOL takes one argument, which is evaluated. Let String
be the value of that argument. If Config\_Length(String) returns an
indicator `E' then the SYNTAX condition 23.1 shall be raised. Otherwise,
if the syntactic recognition described in section nnn would not
recognize String as a symbol then the result of the function SYMBOL is
'BAD''.

If String would be recognized as a symbol the result of the function
SYMBOL depends on the outcome of accessing the value of that symbol, see
nnn. If the final use of Var\_Value leaves the indicator with value `D'
then the result of the function SYMBOL is `LIT', otherwise `VAR'.

\subsection{TIME}\label{time-1}

TIME with less than two arguments returns the local time within the day,
or an elapsed time. Otherwise it converts the second argument (which has
a format given by the third argument) to the format specified by the
first argument.

\lstinputlisting[language=rexx,label=time.rexx,caption=time.rexx]{time.rexx}

10.6.5 VALUE

VALUE returns the value of the symbol named by the first argument, and
optionally assigns it a new value.

\lstinputlisting[language=rexx,label=value.rexx,caption=value.rexx]{value.rexx}
