%preprocessed texin
\chapter{Built-in classes}\label{built-in-classes}

\section{Notation}\label{notation}

The built-in classes are defined mainly through code. The code refers to
state variables. This is solely a notation used in this standard.

\section{Object, class and method}\label{object-class-and-method}

These objects provide the basis for class structure.

\subsection{The object class}\label{the-object-class}

\lstinputlisting[language=rexx,label=object.rexx,caption=object.rexx]{object.rexx}

Returns a new instance of the receiver class.

\lstinputlisting[language=rexx,label=config.rexx,caption=config.rexx]{config.rexx}

\emph{\texttt{\textquotesingle{}==\textquotesingle{}} with no argument
gives a hash value in OOI.}

\lstinputlisting[language=rexx,label=doubleequals.rexx,caption=doubleequals.rexx]{doubleequals.rexx}

Returns a copy of the receiver object. The copied object has the same
methods as the receiver object and an equivalent set of object
variables, with the same values.

\lstinputlisting[language=rexx,label=objectcopy.rexx,caption=objectcopy.rexx]{objectcopy.rexx}

\emph{Since we have \texttt{var\_empty} we could save a primitive by
rendering `new' as `copy' plus `empty'.}

\lstinputlisting[language=rexx,label=defaultname.rexx,caption=defaultname.rexx]{defaultname.rexx}

Returns a short human-readable string representation for the object.

\lstinputlisting[language=rexx,label=receiver.rexx,caption=receiver.rexx]{receiver.rexx}

\emph{This field would have been filled in at `NEW' time.}

\lstinputlisting[language=rexx,label=objectnames.rexx,caption=objectnames.rexx]{objectnames.rexx}

Sets the receiver object's name to the specified string.

\lstinputlisting[language=rexx,label=varset.rexx,caption=varset.rexx]{varset.rexx}

\emph{Initialized to \#Human? Or ObjectName does forwarding until
assigned to?}

\lstinputlisting[language=rexx,label=objectname,caption=objectname]{objectname}

Returns the receiver object's name (which the \texttt{OBJECTNAME=}
method sets).

\lstinputlisting[language=rexx,label=receiverObjectName.rexx,caption=receiverObjectName.rexx]{receiverObjectName.rexx}

Returns a human-readable string representation for the object.

\lstinputlisting[language=rexx,label=methodclass.rexx,caption=methodclass.rexx]{methodclass.rexx}

Returns the class object that received the message that created the
object.

\lstinputlisting[language=rexx,label=setmethod.rexx,caption=setmethod.rexx]{setmethod.rexx}

Adds a method to the receiver object's collection of object methods.

\emph{Is `object methods' what is intended; you add to a class without
adding to its instance methods? Yes.}

\lstinputlisting[language=rexx,label=hasmethod.rexx,caption=hasmethod.rexx]{hasmethod.rexx}

Returns 1 (true) if the receiver object has a method with the specified
name (translated to uppercase); otherwise, returns 0 (false).

\emph{This presumably means inherited as well as \texttt{SETMETHOD}
ones. What about ones set to \texttt{.NIL}?}

\emph{Need to use the same search as for sending.}

\lstinputlisting[language=rexx,label=unsetmethod.rexx,caption=unsetmethod.rexx]{unsetmethod.rexx}

Removes a method from the receiver object's collection of object
methods.

\emph{Use \texttt{var\_drop}}

\emph{Private means Receiver = Self check.}

\lstinputlisting[language=rexx,label=requestmethod.rexx,caption=requestmethod.rexx]{requestmethod.rexx}

Returns an object of the specified class, or the \texttt{NIL} object if
the request cannot be satisfied.

\lstinputlisting[language=rexx,label=runmethod.rexx,caption=runmethod.rexx]{runmethod.rexx}

Runs the specified method. The method has access to the object variables
of the receiver object, just as if the receiver object had defined the
method by using \texttt{SETMETHOD}.

\lstinputlisting[language=rexx,label=startat.rexx,caption=startat.rexx]{startat.rexx}

Returns a message object and sends it a \texttt{START} message to start
concurrent processing.

\lstinputlisting[language=rexx,label=init.rexx,caption=init.rexx]{init.rexx}

Performs any required object initialization.

\subsection{The class class}\label{the-class-class}

\lstinputlisting[language=rexx,label=classclass.rexx,caption=classclass.rexx]{classclass.rexx}

\emph{Lots of these methods are both class and instance. I don't know
whether to list them twice.}

\lstinputlisting[language=rexx,label=newclass.rexx,caption=newclass.rexx]{newclass.rexx}

Returns a new instance of the receiver class, whose object methods are
the instance methods of the class. This method initializes a new
instance by running its \texttt{INIT} methods.

\lstinputlisting[language=rexx,label=subclassclass.rexx,caption=subclassclass.rexx]{subclassclass.rexx}

Returns a new subclass of the receiver class.

\lstinputlisting[language=rexx,label=immediatesubclass,caption=immediatesubclass]{immediatesubclass}

Returns the immediate subclasses of the receiver class in the form of a
single-index array of the required size.

\lstinputlisting[language=rexx,label=defineclass.rexx,caption=defineclass.rexx]{defineclass.rexx}

Incorporates the method object in the receiver class's collection of
instance methods. The method name is translated to upper case.

\lstinputlisting[language=rexx,label=deleteclass.rexx,caption=deleteclass.rexx]{deleteclass.rexx}

Removes the receiver class's definition for the method name specified.

\emph{Builtin classes cannot be altered.}

\lstinputlisting[language=rexx,label=methodmethod.rexx,caption=methodmethod.rexx]{methodmethod.rexx}

Returns the method object for the receiver class's definition for the
method name given.

\emph{Do we have to keep saying ``method object'' as opposed to
``method'' because ``method name'' exists?}

\lstinputlisting[language=rexx,label=querymixin.rexx,caption=querymixin.rexx]{querymixin.rexx}

Returns 1 (true) if the class is a mixin class or 0 (false) otherwise.

\lstinputlisting[language=rexx,label=newmixinclass.rexx,caption=newmixinclass.rexx]{newmixinclass.rexx}

Returns a new mixin subclass of the receiver class.

\lstinputlisting[language=rexx,label=inheritclass.rexx,caption=inheritclass.rexx]{inheritclass.rexx}

Causes the receiver class to inherit the instance and class methods of
the class object specified. The optional class is a class object that
specifies the position of the new superclass in the list of
superclasses.

\lstinputlisting[language=rexx,label=uninheritclass.rexx,caption=uninheritclass.rexx]{uninheritclass.rexx}

Nullifies the effect of any previous \texttt{INHERIT} message sent to
the receiver for the class specified.

\lstinputlisting[language=rexx,label=enhancedclass.rexx,caption=enhancedclass.rexx]{enhancedclass.rexx}

Returns an enhanced new instance of the receiver class, with object
methods that are the instance methods of the class enhanced by the
methods in the specified collection of methods.

\lstinputlisting[language=rexx,label=baseclassclass.rexx,caption=baseclassclass.rexx]{baseclassclass.rexx}

Returns the base class associated with the class. If the class is a
mixin class, the base class is the first superclass that is not also a
mixin class. If the class is not a mixin class, then the base class is
the class receiving the \texttt{BASECLASS} message.

\lstinputlisting[language=rexx,label=superclassclass.rexx,caption=superclassclass.rexx]{superclassclass.rexx}

Returns the immediate superclasses of the receiver class in the form of
a single-index array of the required size.

\lstinputlisting[language=rexx,label=methodid.rexx,caption=methodid.rexx]{methodid.rexx}

Returns a string that is the class identity (instance \texttt{SUBCLASS}
and \texttt{MIXINCLASS} methods.)

\lstinputlisting[language=rexx,label=metaclassclass.rexx,caption=metaclassclass.rexx]{metaclassclass.rexx}

Returns the receiver class's default metaclass.

\lstinputlisting[language=rexx,label=methodsclass.rexx,caption=methodsclass.rexx]{methodsclass.rexx}

Returns a supplier object for all the instance methods of the receiver
class and its superclasses, if no argument is specified.

\subsection{The method class}\label{the-method-class}

\lstinputlisting[language=rexx,label=methodclass.rexx,caption=methodclass.rexx]{methodclass.rexx}

Returns a new instance of method class, which is an executable
representation of the code contained in the source.

\lstinputlisting[language=rexx,label=setprivatemethod.rexx,caption=setprivatemethod.rexx]{setprivatemethod.rexx}

Specifies that a method is a private method.

\lstinputlisting[language=rexx,label=setvarious.rexx,caption=setvarious.rexx]{setvarious.rexx}

Reverses any previous \texttt{SETUNGUARDED} messages, restoring the
receiver to the default guarded status.

\lstinputlisting[language=rexx,label=setunguarded.rexx,caption=setunguarded.rexx]{setunguarded.rexx}

Lets an object run a method even when another method is active on the
same object. If a method object does not receive a \texttt{SETUNGUARDED}
message, it requires exclusive use of its object variable pool.

\lstinputlisting[language=rexx,label=methodsource.rexx,caption=methodsource.rexx]{methodsource.rexx}

Returns the method source code as a single index array of source lines.

\lstinputlisting[language=rexx,label=methodinterface,caption=methodinterface]{methodinterface}

\subsection{The string class}\label{the-string-class}

The string class provides conventional strings and numbers.

\emph{Some differences from REXX class of NetRexx.}

\lstinputlisting[language=rexx,label=stringclass.rexx,caption=stringclass.rexx]{stringclass.rexx}

\emph{We can do all the operators by appeal to classic section 7.}

\lstinputlisting[language=rexx,label=stringmethods,caption=stringmethods]{stringmethods}

\emph{General problem of making the error message come right.}

\lstinputlisting[language=rexx,label=stringoperatorsandmethods,caption=stringoperatorsandmethods]{stringoperatorsandmethods}

\subsection{The array class}\label{the-array-class}

The main features of a single dimension array are provided by the
configuration. This section defines further methods and
multi-dimensional arrays.

\emph{To be done. Dimensionality set at first use. Count commas, not
classic \texttt{arg()}.}

\lstinputlisting[language=rexx,label=arrayclass.rexx,caption=arrayclass.rexx]{arrayclass.rexx}

Returns a new empty array.

\lstinputlisting[language=rexx,label=returnsingleindexarray.rexx,caption=returnsingleindexarray.rexx]{returnsingleindexarray.rexx}

Returns a newly created single-index array containing the specified
value objects.

\lstinputlisting[language=rexx,label=makearraymemberitem.rexx,caption=makearraymemberitem.rexx]{makearraymemberitem.rexx}

Makes the object value a member item of the array and associates it with
the specified index or indexes.

\lstinputlisting[language=rexx,label=operatorputmethodarray.rexx,caption=operatorputmethodarray.rexx]{operatorputmethodarray.rexx}

This method is the same as the \texttt{PUT} method.

\lstinputlisting[language=rexx,label=arraymethodat.rexx,caption=arraymethodat.rexx]{arraymethodat.rexx}

Returns the item associated with the specified index or indexes.

\lstinputlisting[language=rexx,label=arraymethodatoperator.rexx,caption=arraymethodatoperator.rexx]{arraymethodatoperator.rexx}

Returns the same value as the AT method.

\lstinputlisting[language=rexx,label=arraymethodremove.rexx,caption=arraymethodremove.rexx]{arraymethodremove.rexx}

Returns and removes the member item with the specified index or indexes
from the array.

\lstinputlisting[language=rexx,label=arraymethodhasindex.rexx,caption=arraymethodhasindex.rexx]{arraymethodhasindex.rexx}

Returns 1 (true) if the array contains an item associated with the
specified index or indexes. Returns 0 (false) otherwise.

\lstinputlisting[language=rexx,label=arraymethoditems.rexx,caption=arraymethoditems.rexx]{arraymethoditems.rexx}

Returns the number of items in the collection.

\lstinputlisting[language=rexx,label=arraymethoddimension.rexx,caption=arraymethoddimension.rexx]{arraymethoddimension.rexx}

Returns the current size (upper bound) of dimension specified (a
positive whole number). If you omit the argument this method returns the
dimensionality (number of dimensions) of the array.

\lstinputlisting[language=rexx,label=arraymethodsize.rexx,caption=arraymethodsize.rexx]{arraymethodsize.rexx}

Returns the number of items that can be placed in the array before it
needs to be extended.

\lstinputlisting[language=rexx,label=arraymethodfirst.rexx,caption=arraymethodfirst.rexx]{arraymethodfirst.rexx}

Returns the index of the first item in the array, or the \texttt{NIL}
object if the array is empty.

\lstinputlisting[language=rexx,label=arraymethodlast,caption=arraymethodlast]{arraymethodlast}

Returns the index of the last item in the array, or the \texttt{NIL}
object if the array is empty.

\lstinputlisting[language=rexx,label=arraymethodnext.rexx,caption=arraymethodnext.rexx]{arraymethodnext.rexx}

Returns the index of the item that follows the array item having the
specified index or returns the \texttt{NIL} object if the item having
that index is last in the array.

\lstinputlisting[language=rexx,label=arraymethodprevious.rexx,caption=arraymethodprevious.rexx]{arraymethodprevious.rexx}

Returns the index of the item that precedes the array item having index
index or the \texttt{NIL} object if the item having that index is first
in the array.

\lstinputlisting[language=rexx,label=arraymethodmakearray.rexx,caption=arraymethodmakearray.rexx]{arraymethodmakearray.rexx}

Returns a single-index array with the same number of items as the
receiver object. Any index with no associated item is omitted from the
new array.

Returns a new array (of the same class as the receiver) containing
selected items from the receiver array. The first item in the new array
is the item corresponding to index start (the first argument) in the
receiver array.

\lstinputlisting[language=rexx,label=arraymethodsupplier,caption=arraymethodsupplier]{arraymethodsupplier}

Returns a supplier object for the collection.

\lstinputlisting[language=rexx,label=arraymethodsection,caption=arraymethodsection]{arraymethodsection}

\subsection{The supplier class}\label{the-supplier-class}

A supplier object enumerates the items a collection contained at the
time of the supplier's creation.

\lstinputlisting[language=rexx,label=supplierclass.rexx,caption=supplierclass.rexx]{supplierclass.rexx}

Returns a new supplier object.

\lstinputlisting[language=rexx,label=suppliermethodindex.rexx,caption=suppliermethodindex.rexx]{suppliermethodindex.rexx}

Returns the index of the current item in the collection.

\lstinputlisting[language=rexx,label=suppliermethodnext.rexx,caption=suppliermethodnext.rexx]{suppliermethodnext.rexx}

Moves to the next item in the collection.

\lstinputlisting[language=rexx,label=suppliermethoditem.rexx,caption=suppliermethoditem.rexx]{suppliermethoditem.rexx}

Returns the current item in the collection.

\lstinputlisting[language=rexx,label=supplierindexavailable.rexx,caption=supplierindexavailable.rexx]{supplierindexavailable.rexx}

Returns 1 (true) if an item is available from the supplier (that is, if
the \texttt{ITEM} method would return a value). Returns 0 (false)
otherwise.

\subsection{The message class}\label{the-message-class}

\lstinputlisting[language=rexx,label=messageclass.rexx,caption=messageclass.rexx]{messageclass.rexx}

Initializes the message object for sending\ldots\ldots{}

\lstinputlisting[language=rexx,label=messagecompletedmethod.rexx,caption=messagecompletedmethod.rexx]{messagecompletedmethod.rexx}

Returns 1 if the message object has completed its message; returns 0
otherwise.

\lstinputlisting[language=rexx,label=messagebotifymethod.rexx,caption=messagebotifymethod.rexx]{messagebotifymethod.rexx}

Requests notification about the completion of processing for the message
\texttt{SEND} or \texttt{START} sends.

\lstinputlisting[language=rexx,label=messagestartmethod.rexx,caption=messagestartmethod.rexx]{messagestartmethod.rexx}

Sends the message for processing concurrently with continued processing
of the sender.

\lstinputlisting[language=rexx,label=messagesendmethod.rexx,caption=messagesendmethod.rexx]{messagesendmethod.rexx}

Returns the result (if any) of sending the message.

\lstinputlisting[language=rexx,label=methodresult.rexx,caption=methodresult.rexx]{methodresult.rexx}

Returns the result of the message \texttt{SEND} or \texttt{START} sends.
