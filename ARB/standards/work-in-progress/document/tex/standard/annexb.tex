%preprocessed texin
\hypertarget{annex-b}{%
\chapter{Annex B}\label{annex-b}}

(informative)

Method of definition

This annex describes the methods chosen to describe Rexx for this
standard.

Definitions

Definitions are given for some terms which are both used in this
standard and also may be used elsewhere. This does not include names of
syntax constructions; for example, group, which are distinguished in
this standard by the use of italic font.

Conformance

Note that irrespective of how this standard is written, the obligation
on a conforming processor is only to achieve the defined results, not to
follow the algorithms in this standard.

Notation

The notation used to describe functions provided by the configuration is
like a Rexx function call but it is not defined as a Rexx function call
since a Rexx function call is described in terms of one of these
configuration functions.

Note that the mechanism of a returned string with a distinguishing first
character is part of the notation used in this standard to explain the
functions; implementations may use a different mechanism. Notation for
completion response and conditions

The testing of `X' and `S' indicators is made implicit, for brevity.
Even when written as a subroutine call, each use of a configuration
routine implies the testing. Thus:

call Config Time

implies

\#Response = Config Time()

if left (\#Response,1) == `X' then call \#Raise `SYNTAX', 5.1, substr
(\#Response, 2) if left (\#Response,1) == `S' then call \#Raise
`SYNTAX', 48.1, substr(\#Response, 2)

Source programs and character sets

The characters required by Rexx are identified by name, with a glyph
associated so that they can be printed in this standard. Alternative
names are shown as a convenience for the reader.

Notation

Note that nnn is not specifying the syntax of a program; it is
specifying the notation used in this standard for describing syntax.

Lexical level

Productions nnn and nnn contain a recursion of comment. Apart from this
recursion, the lexical level is a finite state automaton.

Syntax level

This syntax shows a null\_clause list, which is minimally a semicolon,
being required in places where programmers do not normally write
semicolons, for example after `THEN'. This is because the `THEN' implies
a semicolon. This approach to the syntax was taken to allow the rule
`semicolons separate clauses' to define `clauses'.

The precedence rules for the operators are built into this grammar

Data Model

The following explanation of data in terms of Classic Rexx may be
helpful. References to clauses of the existing standard have 274 as a
prefix.

We start with the data model from the first Standard - a number of
variable pools. Two mechanisms, the external access of section 274.5.13
(API\_Drop etc) and the internal of 274.7.1 (Var\_Drop etc). Pools are
numbered, with pool 0 reserved for reserved names (.MN etc) and pool N-1
being related to pool N as the caller's pool. The symbols which index
the pools are distinquished as tailed or non-tailed. The items in the
pool have attributes `exposed'', `dropped', and `implicit'. The values
in the pools are string values.

An extra scope is used for `state variables' used in the definition of
the standard. These follow the same lookup rules in a conceptual and
separate pool.

The first change necessary is to define the values in the pools as
references. For string values this is just a change in definition style,
since a reference always followed to a string value is semantically
identical with the notion of having the value in the pool. However,
references open the possibility of referencing

non-strings, which can behave in a changed way while still being refered
to by the same reference. (Mutable objects)

It is reasonable that the definition should have the pools reference one
another rather than use numbered pools. It is difficult to have a notion
of numbering the pools when any object can have a set of variables
associated with it.

Assignment is defined as assignment of references. The language could
have been designed differently, for example to make assignment behave
like the COPY method, but assignment of references is the natural,
powerful, choice.

If pools are not numbered, the notation of the first standard, where
some state variables use the \#Level number as part of their names, will
not suffice. An appropriate solution is to say that each variable pool
can have state variables and user program variables in it. Placing the
state variables that are per-procedure-level in the variable pool for
their level avoids the need to specify \#Level in their tails. There are
pre-existing objects such as all possible values that can be written as
literals and the objects accessed by .SYSTEM etc. Further objects are
created by the NEW method.

Editorial note: It looks nice to unify: an object \emph{is} a variable
pool and a variable pool \emph{is} an object. There is some awkwardness
describing the classic API\_ function as applying to an object. There
don't seem to be difficulties in defining any object behaviour we want
in terms of state variables that refer from one object to another.

Evaluation (Definitions written as code) There is no single definitional
mechanism for describing semantics that is predominantly used in
standards describing programming languages, except for the use of prose.
The committee has chosen to define some parts of this standard using
algorithms written in Rexx. This has the advantages of being rigorous
and familiar to many of the intended readers of this standard. It has
the potential disadvantage of circularity - a definition based on an
assumption that the reader already understands what is being defined.
Circularity has been avoided by: - specifying the language
incrementally, so that the algorithms for more complex details are
specified in code that uses only more simple Rexx. For example, the
notion that an expression evaluates to a result can be understood by the
reader even without a complete specification of all operators and
built-in functions that might be used in the expression; - specifying
the valid syntax of Rexx programs without using Rexx coding. The method
used, Backus Normal Form, can adequately be introduced by prose.
Ultimately, some understanding of programming languages is assumed in
the reader (just as the ability to read prose is assumed) but any
remaining circularity in this standard is harmless. The comparison of
two single characters is an example of such a circularity;
Config\_Compare can compare two characters but the outcome can only be
tested by comparing characters. It has to be assumed that the reader
understands such a comparison. Some of the definition using code is
repeating earlier definition in prose. This duplication is to make the
document easier to understand when read from front to back. Note that
the layout of the code, in the choices of instructions-per-line,
indentations etc., is not significant. (The layout style used follows
the examples in the base reference and it is deliberate that the DO and
END of a group are not at the same alignment.) The code is not intended
as an example of good programming practice or style. The variables in
this code cannot be directly referenced by any program, even if the
spelling of some VAR\_SYMBOL coincides. These variables, referred to as
state variables, are referenced throughout this document; they are not
affected by any execution activity involving scopes. Some of more
significant variables and routines are written with \# as their first
character. The following list of them is intended as an aid to
understanding the code. The index of this standard shows the main usage,
but not all usage, of these names. The following are constants set by
the configuration, by Config\_Constants: \#Configuration is used for
PARSE SOURCE. \#Version is used for PARSE VERSION. \#Bif\_Digits.
represents numeric digits settings, tails are built-in function names.
\#Limit\_Digits is the maximum significant digits.
\#Limit\_EnvironmentName is a maximum length.

\#Limit\_ExponentDigits is the maximum digits in an exponent.
\#Limit\_Literal is a maximum length.

\#Limit\_Messagelnsert is a maximum length.

\#Limit\_Name is a maximum length.

\#Limit\_String is a maximum length.

\#Limit\_TraceData is a maximum length.

These are named outputs of configuration routines:

\#Response is used to hold the result from a configuration routine.
\#Indicator is used to hold the leftmost character of Response.
\#Outcome is the main outcome of a configuration routine.

\#RC is set by Contig\_Command.

\#NoSource is set by Config\_NoSource.

\#Time is set by Config\_Time

\#Adjust\textless Index ``\#Adjust'' \#``\,'' \textgreater{} is set by
Config\_Time

These variables are set up with output from configuration routines:

\#Howlnvoked records from API\_Start, for use by PARSE SOURCE.

\#Source records from API\_ Start for use by PARSE SOURCE.

\#AIIBlanks\textless Index ``\#AllBlanks'' \# ``\,'' \textgreater{} is a
string including Blank and equivalents. \#ErrorText.MsgNumber is the
text as altered by limits.

\#SourceLine. is a record of the source, retained unless NoSource is
set. \#SourceLine.0 is a count of lines.

\#Pool is a reference to the current variable pool.

These are variables not initialized from the configuration:

\#Level is a count of invocation depth, starting at one.

\#NewLevel equals \#Level plus one.

\#Pool1 is a reference to the variable pool current when the first
instruction was executed.

\#Upper is a reference to the variable pool which will be current when
the current PROCEDURE ends. \#Loop is a count of loop nesting.

\#LineNumber is the line number of the current clause.

\#Symbol is a symbol after tails replacement.

\#API\_Enabled determines when the application programming interface for
variable pools is available. \#Test is the Greater/Lesser/Equal result.

\#InhibitPauses is a numeric trace control.

\#InhibitTrace is a numeric trace control.

\#AtPause is on when executing interactive input.

\#AllowProcedure provides a check for the label needed before a
procedure.

\#DatatypeResult is a by-product of DATATYPE().

\#Condition is a condition, eg `SYNTAX'.

\#Trace\_QueryPrior detects an external request for tracing.

\#Tracelnstruction detects TRACE as interactive input.

These are variables that are per-Level, that is, have \#Level as a tail
component:

\#lsFunction. indicates a function call.

\#lsProcedure. indicates indicates the routine is a procedure.

\#Condition. indicates whether the routine is handling a condition.

\#ArgExists.\#Level.ArgNumber indicates whether an argument exists.
(Initialized from API\_Start for Level=1)

\#Arg.\#Level.ArgNumber provides the value of an argument. (Initialized
from API\_Start for Level=1) When ArgNumber=0 this gives a count of the
arguments.

\#Tracing. is the trace setting letter.

\#Interactive. indicates when tracing is interactive. (`?' trace
setting) \#ClauseLocal. ensures that DATE/TIME are consistent across a
clause. \#ClauseTime. is the TIME/DATE frozen for the clause.

\#StartTime. is for `Elapsed' time calculations.

\#Digits. is the current numeric digits.

\#Form. is the current numeric form.

\#Fuzz. is the current numeric fuzz.

These are qualified by \#Condition as well as \#Level:

\#Enabling. is `ON', `OFF' or 'DELAYED''.

\#Instruction. is `CALL' or `SIGNAL'

\#TrapName. is the label.

\#ConditionDescription. is for CONDITION(`D')

\#ConditionExtra. is for CONDITION(`E')

\#ConditionInstruction. is for CONDITION(`T')

\#PendingNow. indicates a DELAYED condition. \#PendingDescription. is
the description of a DELAYED condition. \#PendingExtra. is the extra
description fora DELAYED condition. \#EventLevel. is the \#Level at
which an event was DELAYED.

These are qualified by ACTIVE, ALTERNATE, or TRANSIENT as well as
\#Level:

\#Env\_Name. is the environment name.

\#Env\_Type. is the type of a resource, and is additionally qualified by
input/output/error distinction. \#Env\_Resource. is the name of a
resource, and is additionally qualified by input/output/error
distinction. \#Env\_Position. is INPUT or APPEND or REPLACE, and is
additionally qualified by input/output/error distinction.

These are variables that are per-loop:

\#ldentity. is the control variable.

\#Repeat. is the repetition count.

\#By. is the increment.

\#To. is the limit.

\#For. is that count.

\#lterate. holds a position in code describing DO instruction semantics.
\#Once. holds a position in code describing DO instruction semantics.

\#Leave. holds a position in code describing DO instruction semantics.

These are variables that are per-stream:

\#Charin\_Position.

\#Charout\_Position.

\#Linein\_Position.

\#Lineout\_Position.

\#StreamState. records ERROR state for return by STREAM built-in
function.

These are commonly used prefixes: Config\_ is used for a function
provided by the configuration. API\_is used for an application
programming interface.

Trap\_ is used for a routine called from the processor, not provided by
it. Var\_ is used for the routines operating on the variable pools.

These are notation routines, only available to code in this standard:

\#Contains checks whether some construct is in the source. \#Instance
returns the content of some construct in the source. \#Evaluate returns
the value of some construct in the source. \#Execute causes execution of
some construct in the source. \#Parses checks whether a string matches
some construct. \#Clause notes some position in the code.

\#Goto continues execution at some noted position.

\#Retry causes execution to continue at a previous clause.

These are frequently used routines: \#Raise is a routine for condition
raising. \#Trace is a routine for trace output.

\#TraceSource is a routine to trace the source program. \#CheckArgs
processes the arguments to a built-in function.
