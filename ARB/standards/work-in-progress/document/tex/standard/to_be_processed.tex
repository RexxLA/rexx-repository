%preprocessed texin
\chapter{To be processed:}\label{to-be-processed}

The following decisons are abstracted from minutes. We need to ensure
they are covered in the main standard and their rationale appropriately
reworded for this annex.

Aliasing. Assignment is viewed as making the target reference the same
object as the source. Hence the object (and changes to it) may be
accessed through more than one name. For `immutable' objects a changed
version of an object can only be produced by creating a new object. For
compatibility with classic Rexx, strings are immutable objects.
Non-strings may or may not be immutable. Note that there is an
alternative model in which distinction is made between assignments which
copy values and assignments which copy references. This alternative was
not chosen; the committee prefered the model in which all data names are
naming references (which may be implicitly followed to values).

Arguments `by-reference'. The introduction of aliasing makes this
natural although the detail has simple-versus-general contentions. (Is
it necessary for simple strings to be passed by reference.
Encapsulation. An object may `own' some variables and access to those
may be limited (so that re-implementation of the object could use
different variables without upsetting the usage of the object).
Classess. There will be `factory' objects capable of creating multiple
new objects which have common characteristics about how they can be
used.

Inheritance and hierachy. The semantics of a clas may be specified by
adding to the semantics of another class. This relation is used to form
a tree. We prefer a singly rooted tree, rooted in the class `Object'
which is built-in to the language. Other classes will also be built-in.
Experience with OOM and other languages is that unrestricted inheritance
by one class from multiple classes does not work in the way the coder
intended (the implementations of the classes do not combine
successfully). If multiple inheritance is added to Rexx at all, it will
be in the cautious `MIXIN' flavor of OOI.

Messaging: Executing some labelled code which is associated with objects
of a given class is a form of invocation that is sufficiently different
from classic Rexx to justify a new syntax construct. The new syntax is
Receiver \textasciitilde{} MethodName(Arguments) and implies both a
different search for the method to be invoked and a special role for the
receiver as opposed to the other arguments of the invocation.

Packaging: In principle a `program builder' could be used in developing
Rexx programs with many classes and methods, and that builder could hide
from the coder the details of how the configuration held the methods.
However, rather than define a program builder we are choosing to define
a simple method of holding multiple classes \& mthods (with
specification of their hierarchy) within a single text file. The
non-executable dividers in such a file are known as directives. The
files are known as packages and a package may specify (by directive)
that it requires another package in order to function correctly. There
are questions about when initialization of required packages occurs; we
intend to find a solution that does not require the complete graph of
requirements to be initialized before other code is executed.

A note on the syntax of directives. When no special token (eg ::) is
used to introduce directives the directives are be recognizable by the
spelling of the keyword. (CLASS REQUIRES etc.) The purpose of the
special token is emphasis of directives rather than implementation ease
in ``pre-processing'' the directives.

Packages in non-Rexx. It is necessary to exploit packages that are not
written in Rexx. To invoke their methods it is necessary that the
package makes known to the Rexx method search the names of the classes
and their methods. To do more than invoke the methods (eg to subclass
the the external classes) requires complicated mechanisms and may not be
a requirement.

External procedures. To allow Classic internal procedures to be
separated into different files with undue change of semantics, the
PROCEDURE statement will be permiteed as the first statement of a
routine which is in a separate file.

Concurrency will be added, that is multiple execution cursors
progressing through one program. The mechanism for creating multiple
cursors will be the ``early reply'' where one cursor becomes two; one of
two progresses by ``falling through'' the early reply and the other
starts its progress after the site of the current invocation. Multiple
cursors carry the risk of execution interleaving in a way which negates
the coder's intentions in writing so that clauses would execute
sequentially. The language definition will be tightened to ensure
atomicity of string assignment etc. Additionally, a set of rules about
allowing two cursors on the same method at the same time will provide a
reduction of the risk. Since in many cases the data which have to be
maintained consistent will reside in a single object the rules are
object-based. In general a cursor on a method executing against a
particular object will delay any other cursor from executing methods
against that object.

This rule provides sensible synchronization without much effort from the
programmer but other controls may be provided:

\begin{enumerate}
\def\labelenumi{\alph{enumi})}
\item
  Stronger control, eg only one cursor within the methods of a set of
  objects.
\item
  More detailed control, eg division of a method into sections which
  allow/disallow other cursors into the section.
\end{enumerate}

Extended Variable Pools. The API for variable pools will need to be
extended to reflect the model in which the named content in a pool is
always a reference (and the reference is followed when the value of a
string is required.) We note that OO! adopts a convention that names
starting with `!' (shriek) name objects that are not intended for access
by the coder. These objects will not be standardized. Additionally some
objects without shriek names are not candidates for standardising, eg
SYSTEM, .KERNEL.

A model is needed for whether changes made to methods are seen by
objects created before the changes. Changes that are seen are preferable
where a long-lived object is being brought up-to-date. Changes that only
apply to future objects are preferable if avoiding failure of what
``used to work'' is the priority. In view of OOM experience the standard
should allow both, on a method by method choice. (eg perhaps a bug fix
applied retrospectively but not an enhancement.)

Multiple inheritance. Study of the `method search' algorithm, see later,
shows that this is an ``add-on'' that could readily be retained or
omitted. That argues in principle for retention, since the non-user of
multiple inheritance would not suffer from it. On the other hand it adds
complexity and can be misused even in the conservative form that OOI has
it.

Signature-based method search. This is not in OOI but is in languages
such as Java.

Subclassing of imported classes. It is our intention to say that
imported classes can be used in all the same ways as builtin classes.
Because this may be impractical to implement with some external classes,
a conforming language processor will have a list (which may be empty) of
external classes it supports. (And hence nothing of the current SOM
interface will be part of the standard.)

Persistent objects. It is our belief that support for very-long-running
programs is required. It is a moot point whether the ENVIRONMENT
directory is enough.

If persistent objects are to converted to a form which is platform
independent, (``pickling''), there are difficulties in deciding what
pointers should be followed and further objects included, as opposed to
objects being assumed available on all platforms. This topic is defered.

Locking across a set of objects. In OOI this can only be done by locking
the events serially, which has more risk of deadlock than locking them
simultaneously. The decision was made not to add simultaneous locking.

Critical sections. The GUARD mechanism can be used in a `critical
section' style. Nothing will be added to the definition.

Old objects seeing new changed methods. When bugs in long running
programs are fixed, there can be a benefit if old objects see the
corrected methods. It seems practical to offer a variation of DEFINE for
this - see method lookup discussion.

The committee does not find the current OOI approach to merging
`classic' stems with OO stems satisfactory. It invalidates some existing
programs. (A warning about this was put in A8.3.3 of X3.274.) It
produces surprises for OO programmers, eg a==b after
a=.stem\textasciitilde new; b=.stem\textasciitilde new. The proposed
alternative is to make the presence/absence of a dot at the end of the
name determine whether coercion to string is done. The `classic' meaning
of A.=B. would be restored but AA=BB, AA==BB etc. would have their OO
meanings. The meaning of USE ARG with a dotted name would be defined to
allow `by reference' passing of a stem. Square brackets could be used
with both dotted and undotted names. A further proposal is to note that
this leaves few differences between the DIRECTORY class and the
non-dotted STEM class so that it might be a further improvement if the
DIRECTORY class was extended to the extent that the STEM class was
unnecessary.

There is a potential problem which the committee has not fully analysed
in the OO! treatment of SAY and streams. OOI has made features (of the
STREAM bif) that were configuration determined in X3-274 into OO
language methods, and has made SAY a method (undocumented?). Full
analysis may show that more of I/O could (\& should?) be made standard
or may show that some OO! I/O language should not be standardized.

The committee discussed what parts of the OOI implementation were
suitable to be defined in a standard. Potentially, all the builtin
classes and objects (which are reachable from .ENVIRONMENT) might be
standardized. However, names which start with an exclamation mark denote
unsuitable things. The committee also thought the following unsuitable:

\begin{itemize}
\tightlist
\item
  Anything specific to SOM. - RX\_QUEUE - Stream\_Supplier - Parts of
  LOCAL other than direct reference to the default streams. There is a
  naming problem with this. The names in OOI are STDIN, STDOUT and
  STDERR. We would prefer INPUT, OUTPUT, and ERROR to be\_ consistent
  with the keywords. OOI has used those names for something else. We
  will work on the proposal that we use the prefered names and the
  MONITOR class is dropped. (Users who want the monitor function can get
  it with a few lines of directive.)
\end{itemize}

The committee feels that OOI over-specifies the index of an item in a
LIST. In OO it is a count giving the sequence over time of the
insertions in the list. The risk in using numbers is that they may be
(wrongly) used as positions, and arithmetic done on them. It is proposed
that the index of a list item be of class OBJECT rather than of class
STRING.

In OOI, the .ENVIRONMENT is global, not read-only, and contains builtin
objects such as .TRUE and .FALSE. The committee regards this as too
risky - suppose that .TRUE was accidentally or maliciously revalued as
0! It seems sufficient to add read-only as a characteristic of
directories. (This characteristic at the element level might be
expensive to implement.) Reserved symbols (X8-274 clause 6.2.3.1) also
provide a mechanism for preventing the override of builtin names. It
won't be possible for a standard to exactly define in a
system-independent way the scopes/lifetimes of -ENVIRONMENT and .LOCAL
but (as with OOl) the .LOCAL will relate to ``One API\_START'' and
-ENVIRONMENT will have a wider scope. (Power on to power off of some
system'). The proposed ``search order'' is:

\begin{enumerate}
\def\labelenumi{\arabic{enumi}.}
\item
  Things provided by the system which no user is expected to want to
  override. Perhaps .TRUE .FALSE NIL.
\item
  The .LOCAL read/write directory, initialized with the default streams,
  changable by the user for individual program executions. Perhaps
  METHODS here.
\item
  The read-only part of the environment, that is the builtin classes and
  objects. Also .SYSTEM perhaps.
\item
  The read/write ENVIRONMENT directory. Changable by programmers
  co-operating at the system level. Final placement of all builtins
  needs discussion, but the read-only true\&false requirement will be
  met. Note that the algorithm of method lookup does not change if ``old
  objects see newest methods'' is desired. What changes is whether the
  method tables are updated in place or copied-and-updated when they are
  changed.
\item
  There have been sugestions to allow the REQUIRES directive appear in
  more places. The committee agrees with this and proposes:
\end{enumerate}

\begin{enumerate}
\def\labelenumi{\Alph{enumi})}
\tightlist
\item
  All REQUIRES directives must appear together in the file. B) These
  directives may appear anywhere the OOI implementation currently allows
  them to appear.
\end{enumerate}

\begin{enumerate}
\def\labelenumi{\arabic{enumi}.}
\setcounter{enumi}{1}
\item
  Message numbers and prose are now allocated to messages detected by
  the syntax, additional to the messages known to the first standard.
  Most messages simply involve new minor codes sequential beyond those
  defined in the first standard.
\item
  Proposed language, eg FORWARD, METHOD, and CLASS clauses, allow for
  many options which can appear in any order. These can be written in
  the BNF (in the manner that TO BY FOR were handled in the first
  standard) but it is neater to extend the BNF metalanguage.
\item
  The OOI syntax used in the FORWARD instruction has examples of the
  `argument' construct, which is either a symbol-or-string taken as a
  constant or is an expression in parentheses. The committee will define
  `term' to be allowed in such places. This is a change to the OOI for
  valid programs only in the case where a MESSAGE option used a symbol
  intending it to be `taken as a constant'. (As opposed to taken as a
  variable with the value defaulting to its name when uninitialized.)
\item
  In a similar vein to 4 above, some other positions where the
  ``variable reference'' notation is used (or proposed) will be changed.
  It would be nice to allow ``term'' in all these places but ambiguity
  consideration means some will be ``sub-expression'', ie parenthesed
  expression, notation.
\item
  The colon used for superclass specification will allow
  symbol-or-string to follow. DATA:
\item
  The model of data used in defining the first standard needs changing
  for OO, to:
\end{enumerate}

\begin{itemize}
\item
  Variable pools are objects, objects are variable pools.
\item
  Variable pool contents are references to objects, not values of
  strings.
\item
  Pools are not numbered, they are referenced.
\item
  The state variables (those with names beginning `\#' used to define
  processing in the standard) are present in all pools, as opposed to
  being in a separate pool.
\end{itemize}

This data model gives a natural interpretation to the variable pool API
applied to local pools. (Local pools may access non-local pool items by
reason of EXPOSE.) In principle this leads to different threads of
execution (resulting from REPLY) being able to execute the API. (In
practice OOI has a restriction to executing the API only on the `main'
thread and the committee needs to know if this is due to a generally
applicable difficulty.)

The committee considered the relevance of IBM's ``Object Rexx
Programming Guide'' G25H-7597-1 to the Configuration section of the
standard. The material there in Appendix A under headings External
Function Interface, System Exit Interface, and Variable Pool Interface
was deemed material for inclusion, and the rest not. This is similar to
the first standard, although there will be an extra trap, for method
calls. The committee considered the relevance of the STREAM section of
IBM's ``Object Rexx Reference'', G25H-7598-0. That stream class brings
into the language more I/O than the original Rexx, eg an explicit CLOSE.
The new standard will partially follow this trend also.

PEEK on queue unnecessary - same as AT{[}1{]}?

Also need to resolve the issues on Monitor class and on run time
inspection.
