\hypertarget{conformance}{%
\chapter{Conformance}\label{conformance}}

\hypertarget{conformance-1}{%
\section{Conformance}\label{conformance-1}}

A conforming language processor shall not implement any variation of
this standard except where this standard permits. Such permitted
variations shall be implemented in the manner prescribed by this
standard and noted in the documentation accompanying the processor. A
conforming processor shall include in its accompanying documentation

\begin{itemize}
\item
  alist of all definitions or values for the features in this standard
  which are specified to be dependent on the configuration.
\item
  a statement of conformity, giving the complete reference of this
  standard (ANSI X3.274-1996) with which conformity is claimed.
\end{itemize}

\hypertarget{limits}{%
\section{Limits}\label{limits}}

Aside from the items listed here (and the assumed limitation in
resources of the configuration), a conforming language processor shall
not put numerical limits on the content of a program. Where a limit
expresses the limit on a number of digits, it shall be a multiple of
three. Other limits shall be one of the numbers one, five or twenty
five, or any of these multiplied by some power of ten. Limitations that
conforming language processors may impose are:

\begin{itemize}
\item
  NUMERIC DIGITS values shall be supported up to a value of at least
  nine hundred and ninety nine.
\item
  Exponents shall be supported. The limit of the absolute value of an
  exponent shall be at least as large as the largest number that can be
  expressed without an exponent in nine digits.
\item
  String lengths shall be supported. The limit on the length shall be at
  least as large as the largest number that can be expressed without an
  exponent in nine digits.
\item
  String literal length shall be supported up to at least two hundred
  and fifty.
\item
  Symbol length shall be supported up to at least two hundred and fifty.
\end{itemize}
