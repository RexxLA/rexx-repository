\hypertarget{evaluation}{%
\chapter{Evaluation}\label{evaluation}}

The syntax section describes how expressions and the components of
expressions are written in a program. It also describes how operators
can be associated with the strings, symbols and function results which
are their operands.

This evaluation section describes what values these components have in
execution, or how they have no value because a condition is raised.

This section refers to the DATATYPE built-in function when checking
operands, see nnn. Except for considerations of limits on the values of
exponents, the test:

datatype (Subject) == `NUM' is equivalent to testing whether the subject
matches the syntax: num := {[}blank+{]} {[}`+' \textbar{} `-'{]}
{[}blank+{]} number {[}blank+{]}

For the syntax of number see nnn.

When the matching subject does not include a `-' the value is the value
of the number in the match, otherwise the value is the value of the
expression (0 - number).

The test:

datatype (Subject , `W')

is a test that the Subject matches that syntax and also has a value that
is ``whole'', that is has no non-zero fractional part.

When these two tests are made and the Subject matches the constraints
but has an exponent that is not

in the correct range of values then a condition is raised: call \#Raise
`SYNTAX', 41.7, Subject

This possibility is implied by the uses of DATATYPE and not shown
explicitly in the rest of this section nnn.

\hypertarget{variables}{%
\section{Variables}\label{variables}}

The values of variables are held in variable pools. The capabilities of
variable pools are listed here, together with the way each function will
be referenced in this definition.

The notation used here is the same as that defined in sections nnn and
nnn, including the fact that the Var\_ routines may return an indicator
of `N', `S' or `X'.

Each possible name in a variable pool is qualified as tailed or
non-tailed name; names with different qualification and the same
spelling are different items in the pool. For those Var\_ functions with
a third argument this argument indicates the qualification; it is `1'
when addressing tailed names or `0' when addressing non-tailed names.

Each item in a variable pool is associated with three attributes and a
value. The attributes are `dropped' or `not-dropped', `exposed' or
`not-exposed' and `implicit' or `not-implicit'.

A variable pool is associated with a reference denoted by the first
argument, with name Pool. The value of Pool may alter during execution.
The same name, in conjunction with different values of Pool, can
correspond to different values.

\hypertarget{var_empty}{%
\subsection{Var\_Empty}\label{var_empty}}

Var\_Empty (Pool)

The function sets the variable pool associated with the specified
reference to the state where every name is associated with attributes
`dropped', `implicit' and `not-exposed'.

\hypertarget{var-set}{%
\subsection{Var Set}\label{var-set}}

Var \_Set(Pool, Name, `0', Value)

The function operates on the variable pool with the specified reference.
The name is a non-tailed name. If the specified name has the `exposed'
attribute then Var\_Set operates on the variable pool referenced by
\#Upper in this pool and this rule is applied to that pool. When the
pool with attribute `not-exposed' for this name is determined the
specified value is associated with the specified name. It also
associates the attributes `not-dropped' and `not-implicit'. If that
attribute was previously `not-dropped' then the indicator returned is
`R'. The name is a stem if it contains just one period, as its rightmost
character. When the name is a stem Var\_Set(Pool, TailedName, `1',Value)
is executed for all possible valid tailed names which

have Name as their stem, and then those tailed-names are given the
attribute `implicit'. Var \_Set(Pool, Name, `1', Value)

The function operates on the variable pool with the specified reference.
The name is a tailed name. The left part of the name, up to and
including the first period, is the stem. The stem is a non-tailed name.
If the specified stem has the `exposed' attribute then Var\_Set operates
on the variable pool referenced by \#Upper in this pool and this rule is
applied to that pool. When the pool with attribute `not-exposed' for the
stem is determined the name is considered in that pool. If the name has
the `exposed' attribute then the

variable pool referenced by \#Upper in the pool is considered and this
rule applied to that pool. When the pool with attribute `not-exposed' is
determined the specified value is associated with the specified name. It
also associates the attributes `not-dropped' and `not-implicit' . If
that attribute was previously `not-dropped' then the indicator returned
is `R'.

\hypertarget{var_value}{%
\subsection{Var\_Value}\label{var_value}}

Var \_Value(Pool, Name, `0')

The function operates on the variable pool with the specified reference.
The name is a non-tailed name. If the specified name has the `exposed'
attribute then Var\_Value operates on the variable pool referenced by
\#Upper in this pool and this rule is applied to that pool. When the
pool with attribute `not-exposed' for this name is determined the
indicator returned is `D' if the name has `dropped' associated, `N'
otherwise. In the former case \#Outcome is set equal to Name, in the
latter case \#Outcome is set to the value most

recently associated with the name by Var\_Set. Var \_Value(Pool, Name,
`1')

The function operates on the variable pool with the specified reference.
The name is a tailed name. The left part of the name, up to and
including the first period, is the stem. The stem is a non-tailed name.
If the specified stem has the `exposed' attribute then Var\_Value
operates on the variable pool referenced by \#Upper in this pool and
this rule is applied to that pool. When the pool with attribute
`not-exposed' for the stem is determined the name is considered in that
pool. If the name has the `exposed' attribute then the variable pool
referenced by \#Upper in the pool is considered and this rule applied to
that pool. When the pool with attribute `not-exposed' is determined the
indicator returned is `D' if the name has `dropped' associated, `N'
otherwise. In the former case \#Outcome is set equal to Name, in the
latter case \#Outcome is set to the value most recently associated with
the name by Var\_Set.

\hypertarget{var_drop}{%
\subsection{Var\_Drop}\label{var_drop}}

Var \_Drop(Pool, Name, `0')

The function operates on the variable pool with the specified reference.
The name is a non-tailed name. If the specified name has the `exposed'
attribute then Var\_Drop operates on the variable pool referenced by
\#Upper in this pool and this rule is applied to that pool. When the
pool with attribute `not-exposed' for this name is determined the
attribute `dropped' is associated with the specified name. Also, when
the name is a stem, Var\_Drop(Pool,TailedName,`1') is executed for all
possible valid tailed names which have Name as astem.

Var \_Drop(Pool, Name, `1')

The function operates on the variable pool with the specified reference.
The name is a tailed name. The left part of the name, up to and
including the first period, is the stem. The stem is a non-tailed name.
If the specified stem has the `exposed' attribute then Var\_Drop
operates on the variable pool referenced by \#Upper in this pool and
this rule is applied to that pool. When the pool with attribute
`not-exposed' for the stem is determined the name is considered in that
pool. If the name has the `exposed' attribute then the variable pool
referenced by \#Upper in the pool is considered and this rule applied to
that pool. When the pool with attribute `not-exposed' is determined the
attribute `dropped' is associated with the specified name.

\hypertarget{var_expose}{%
\subsection{Var\_Expose}\label{var_expose}}

Var\_Expose (Pool, Name, `0')

The function operates on the variable pool with the specified reference.
The name is a non-tailed name. The attribute `exposed' is associated
with the specified name. Also, when the name is a stem,
Var\_Expose(Pool, TailedName,`1') is executed for all possible valid
tailed names which have Name as a stem.

Var\_Expose (Pool, Name, `1')

The function operates on the variable pool with the specified reference.
The name is a tailed name. The attribute `exposed' is associated with
the specified name.

\hypertarget{var_reset}{%
\subsection{Var\_Reset}\label{var_reset}}

Var\_ Reset (Pool)

The function operates on the variable pool with the specified reference.
It establishes the effect of subsequent API\_Next and API\_NextVariable
functions (see sections nnn and nnn). A Var\_Reset is implied by any
API\_ operation other than API\_Next and API\_NextVariable.

\hypertarget{symbols}{%
\section{Symbols}\label{symbols}}

For the syntax of a symbol see nnn.

The value of a symbol which is a NUMBER or a CONST\_SYMBOL which is not
a reserved symbol is the content of the appropriate token.

The value of a VAR\_SYMBOL which is ``taken as a constant'' is the
VAR\_SYMBOL itself, otherwise the VAR\_SYMBOL identifies a variable and
its value may vary during execution.

Accessing the value of a symbol which is not ``taken as a constant''
shall result in trace output, see nnn: if \#Tracing.\#Level == `I' then
call \#Trace Tag

where Tag is `\textgreater L\textgreater{}' unless the symbol is a
VAR\_SYMBOL which, when used as an argument to Var\_Value, does not
yield an indicator `D'. In that case, the Tag is
'\textgreater V\textgreater``.

\hypertarget{value-of-a-variable}{%
\section{Value of a variable}\label{value-of-a-variable}}

If VAR\_SYMBOL does not contain a period, or contains only one period as
its last character, the value of

the variable is the value associated with VAR\_SYMBOL in the variable
pool, that is \#Outcome after Var\_ Value (Pool, VAR SYMBOL, `0')

If the indicator is `D', indicating the variable has the `dropped'
attribute, the NOVALUE condition is raised; see nnn and nnn for
exceptions to this. \#Response = Var Value(Pool, VAR SYMBOL, `0') if
left(\#Response,1) == `D' then call \#Raise `NOVALUE', VAR\_SYMBOL, '\,'
If VAR\_SYMBOL contains a period which is not its last character, the
value of the variable is the value associated with the derived name.
7.3.1 Derived names A derived name is derived from a VAR\_SYMBOL as
follows:

VAR SYMBOL := Stem Tail

Stem := PlainSymbol `.'!

Tail i= (PlainSymbol \textbar{} `.' {[}PlainSymbol{]}) {[}`.'
{[}PlainSymbol1{]}{]}+ PlainSymbol := (general letter \textbar{} digit)+

The derived name is the concatenation of: - the Stem, without further
evaluation; - the Tail, with the PlainSymbols replaced by the values of
the symbols. The value of a PlainSymbol which does not start with a
digit is \#Outcome after Var\_ Value (Pool, PlainSymbol,`0') These
values are obtained without raising the NOVALUE condition.

If the indicator from the Var\_Value was not `D' then: if
\#Tracing.\#Level == `I' then call \#Trace
`\textgreater C\textgreater{}'

The value associated with a derived name is obtained from the variable
pool, that is \#Outcome after: Var\_Value(Pool,Derived Name,`1')

If the indicator is `D', indicating the variable has the `dropped'
attribute, the NOVALUE condition is raised; see nnn for an exception.

\hypertarget{value-of-a-reserved-symbol}{%
\subsection{Value of a reserved
symbol}\label{value-of-a-reserved-symbol}}

The value of a reserved symbol is the value of a variable with the
corresponding name in the reserved pool, see nnn.

\hypertarget{expressions-and-operators}{%
\section{Expressions and operators}\label{expressions-and-operators}}

Add a load of string coercions. Equality can operate on non-strings.
What if one operand non-string?

\hypertarget{the-value-of-a-term}{%
\subsection{The value of a term}\label{the-value-of-a-term}}

See nnn for the syntax of a ferm.

The value of a STRING is the content of the token; see nnn.

The value of a function is the value it returns, see nnn.

If a termis a symbol or STRING then the value of the term is the value
of that symbol or STRING.

If a term contains an expr\_alias the value of the term is the value of
the expr\_alias, see nnn.

\hypertarget{the-value-of-a-prefix_expression}{%
\subsection{The value of a
prefix\_expression}\label{the-value-of-a-prefix_expression}}

If the prefix\_expression is a term then the value of the
prefix\_expression is the value of the ferm, otherwise let rhs be the
value of the prefix\_expression within it\_\_ see nnn

If the prefix\_expression has the form `+' prefix\_expression then a
check is made: if datatype(rhs)==`NUM' then call \#Raise `SYNTAX',41.3,
rhs, `+'

and the value is the value of (0 + rhs).

If the prefix\_expression has the form `-' prefix\_expression then a
check is made: if datatype(rhs)==`NUM' then call \#Raise
`SYNTAX',41.3,rhs, `-'

and the value is the value of (0 - rhs).

If a prefix\_expression has the form not prefix\_expression then if rhs
== `0' then if rhs ==`1' then call \#Raise `SYNTAX', 34.6, not, rhs

See nnn for the value of the third argument to that \#Raise. If the
value of rhs is `0' then the value of the prefix\_expression value is
`1', otherwise it is `0'.

If the prefix\_expression is not a term then: if \#Tracing.\#Level ==
`I' then call \#Trace `\textgreater P\textgreater{}'

\hypertarget{the-value-of-a-power_expression}{%
\subsection{The value of a
power\_expression}\label{the-value-of-a-power_expression}}

See nnn for the syntax of a power\_expression.

If the power\_expression is a prefix\_expression then the value of the
power\_expression is the value of the prefix\_expression.

Otherwise, let Ihs be the value of power\_expression within it, and rhs
be the value of prefix\_expression within it.

\begin{verbatim}
if datatype(lhs)\=='NUM' then call #Raise 'SYNTAX',41.1,lhs,'**'

if \datatype(rhs,'W') then call #Raise 'SYNTAX',26.1,rhs,'**'
\end{verbatim}

The value of the power\_expression is

ArithOp(lhs,'**',rhs)

If the power\_expression is not a prefix\_expression then: if
\#Tracing.\#Level == `I' then call \#Trace
`\textgreater0O\textgreater{}'

\hypertarget{the-value-of-a-multiplication}{%
\subsection{The value of a
multiplication}\label{the-value-of-a-multiplication}}

See nnn for the syntax of a multiplication. If the multiplication is a
power\_expression then the value of the multiplication is the value of
the power\_expression. Otherwise, let Ihs be the value of multiplication
within it, and rns be the value of power\_expression within it. if
datatype(lhs)==`NUM' then call \#Raise `SYNTAX',41.1,lhs,multiplicative
operation if datatype(rhs)==`NUM' then call \#Raise
`SYNTAX',41.2,rhs,multiplicative operation

The value of the multiplication is ArithOp(lhs,multiplicative operation,
rhs)

If the multiplication is not a power\_expression then:

if \#Tracing.\#Level == `I' then call \#Trace
`\textgreater0O\textgreater{}'

\hypertarget{the-value-of-an-addition}{%
\subsection{The value of an addition}\label{the-value-of-an-addition}}

See nnn for the syntax of addition.

If the addition is a multiplication then the value of the addition is
the value of the multiplication. Otherwise, let Ihs be the value of
ad¢difion within it, and rhs be the value of the multiplication within
it. Let

operation be the adaltive\_operator. if datatype(lhs)==`NUM' then

call \#Raise `SYNTAX', 41.1, lhs, operation if datatype(rhs)==`NUM' then

call \#Raise `SYNTAX', 41.2, rhs, operation

If either of rhs or Ihs is not an integer then the value of the addition
is ArithOp(lhs, operation, rhs) Otherwise if the operation is `+' and
the length of the integer Ihs+rhs is not greater than \#Digits.\#Level

then the value of addition is lhs+rhs

Otherwise if the operation is `-' and the length of the integer Ihs-rhs
is not greater than \#Digits.\#Level then

the value of addition is lhs-rhs

Otherwise the value of the addition is ArithOp(lhs, operation, rhs)

If the addition is not a multiplication then: if \#Tracing.\#Level ==
`I' then call \#Trace `\textgreater0O\textgreater{}'

\hypertarget{the-value-of-a-concatenation}{%
\subsection{The value of a
concatenation}\label{the-value-of-a-concatenation}}

See nnn for the syntax of a concatenation. If the concatenation is an
addition then the value of the concatenation is the value of the
addition. Otherwise, let Ihs be the value of concatenation within it,
and rhs be the value of the additive\_expression within it. If the
concatenation contains `\textbar\textbar{}' then the value of the
concatenation will have the following characteristics:

\begin{itemize}
\item
  Config\_Length(Value) will be equal to
  Config\_Length(Ihs)+Config\_Length(rhs).
\item
  \#Outcome will be `equal' after each of:
\item
  Config\_Compare(Config\_Subsir(Ihs,n)\},Config\_Subsitr(Value,n)) for
  values of n not less than 1 and not more than Config\_Length(Ihs);
\item
  Config\_Compare(Config\_Subsir(rhs,n),Config\_Substr(Value,Config\_Length(Ihs)+n))
  for values of n not less than 1 and not more than Config\_Length(rhs).
  Otherwise the value of the concatenation will have the following
  characteristics:
\item
  Config\_Length(Value) will be equal to
  Config\_Length(Ihs)+1+Config\_Length(rhs).
\item
  \#Outcome will be `equal' after each of:
\item
  Config\_Compare(Config\_Subsir(Ihs,n)\},Config\_Subsitr(Value,n)) for
  values of n not less than 1 and not more than Config\_Length(Ihs);
\item
  Config\_Compare(' ',Config\_Substr(Value,Config\_Length(Ihs)\}+1));
\item
  Config\_Compare(Config\_Subsitr(rhs,n),Config\_Substr(Value,Config\_Length(Ins)+1+n))
  for values of n not less than 1 and not more than Config\_Length(rhs).
\end{itemize}

If the concatenation is not an addition then: if \#Tracing.\#Level ==
`I' then call \#Trace `\textgreater0O\textgreater{}'

\hypertarget{the-value-of-a-comparison}{%
\subsection{The value of a comparison}\label{the-value-of-a-comparison}}

See nnn for the syntax of a comparison.

If the comparison is a concatenation then the value of the comparison is
the value of the concatenation. Otherwise, let Ihs be the value of the
comparison within it, and rns be the value of the concatenation within
it.

If the comparison has a comparison\_operator that is a strict\_compare
then the variable \#Test is set as follows:

\#Test is set to `E'. Let Length be the smaller of Config\_Length(Ihs)
and Config\_Length(rhs). For values of n greater than O and not greater
than Length, if any, in ascending order, \#Test is set to the uppercased
first character of \#Outcome after:

Config\_Compare(Config\_Subsir(Ihs),Contfig\_Subsir(rhs)).

If at any stage this sets \#Test to a value other than `E' then the
setting of \#Test is complete. Otherwise, if Config\_Length(Ihs) is
greater than Config\_Length(rhs) then \#Test is set to `G' or if
Config\_Length(Ihs) is less than Config\_Length(rhs) then \#Test is set
to `L'.

If the comparison has a comparison\_operator that is a normal\_compare
then the variable \#Test is set as follows:

\begin{verbatim}
if datatype(lhs)\== 'NUM' | datatype(rhs)\== 'NUM' then do
/* Non-numeric non-strict comparison */
lhs=strip(lhs, 'B', ' ') /* ExtraBlanks not stripped */
rhs=strip(rhs, 'B', ' ')

if length(lhs)>length(rhs) then rhs=left (rhs, length (lhs) )
else lhs=left (lhs, length (rhs) )
if lhs>>rhs then #Test='G'
else if lhs<<rhs then #Test='L'
else #Test='E'

end
else do /* Numeric comparison */
if left(-lhs,1) == '-' & left(+rhs,1) \== '-' then #Test='G!
else if left(-rhs,1) == '-' & left(+lhs,1) \== '-' then #Test='L'
else do
Difference=lhs - rhs /* Will never raise an arithmetic condition. */
if Difference > 0 then #Test='G'
else if Difference < 0 then #Test='L'
else #Test='E'
end
end
The value of #Test, in conjunction with the operator in the comparison, determines the value of the
comparison.
The value of the comparison is '1' if
- #Test is 'E' and the operator is one of '="", '=="", '>=', <=", ‘\>', '\<', 'p>=', '<<=', \>>', or <<)

- #Test is 'G' and the operator is one of '>', '>=", ‘\<', ‘\=', '<>', '><', Nes", ‘>>! ‘p>’, or <<")
- #Test is 'L' and the operator is one of '<', <=", \>', \=', '<>', '><', \==', '<<', *<<=', or \>>'.
In all other cases the value of the comparison is '0'.

If the comparison is not a concatenation then:
if #Tracing.#Level == 'I' then call #Trace '>0O>'
\end{verbatim}

\hypertarget{the-value-of-an-and_expression}{%
\subsection{The value of an
and\_expression}\label{the-value-of-an-and_expression}}

See nnn for the syntax of an and\_expression.

If the and\_expression is a comparison then the value of the
and\_expression is the value of the comparison.

Otherwise, let Ihs be the value of the and\_expression within it, and
rhs be the value of the comparison within it.

if lhs == `0' then if lhs == `1' then call \#Raise
`SYNTAX',34.5,lhs,`\&'

if rhs == `0' then if rhs == `1' then call \#Raise
`SYNTAX',34.6,rhs,`\&'

Value=`0'

if lhs == `1' then if rhs == `1' then Value=`1'

If the and\_expression is not a comparison then:

if \#Tracing.\#Level == `I' then call \#Trace
`\textgreater0O\textgreater{}'

\hypertarget{the-value-of-an-expression}{%
\subsection{The value of an
expression}\label{the-value-of-an-expression}}

See nnn for the syntax of an expression.

The value of an expression, or an expr, is the value of the expr\_alias
within it.

If the expr\_alias is an and\_expression then the value of the
expr\_alias is the value of the and\_expression. Otherwise, let Ihs be
the value of the expr\_alias within it, and rhs be the value of the
and\_expression

within it. if lhs == `0' then if lhs == `1' then

call \#Raise `SYNTAX',34.5,lhs,or operator if rhs == `0' then if rhs ==
`1' then

call \#Raise `SYNTAX',34.6,rhs,or operator Value=`1' if lhs == `0' then
if rhs == `0' then Value=`0' If the or\_operator is `\&\&' then if lhs
== `1' then if rhs == `1' then Value=`0' If the expr\_alias is not an
and\_expression then: if \#Tracing.\#Level == `I' then call \#Trace
`\textgreater0O\textgreater{}'

The value of an expression or expr shall be traced when
\#Tracing.\#Level is `R'. The tag is `\textgreater=\textgreater{}' when

the value is used by an assignment and
`\textgreater\textgreater\textgreater{}' when it is not. if
\#Tracing.\#Level == `R' then call \#Trace Tag

\hypertarget{arithmetic-operations}{%
\subsection{Arithmetic operations}\label{arithmetic-operations}}

The user of this standard is assumed to know the results of the binary
operators `+' and `-' applied to signed or unsigned integers.

The code of ArithOp itself is assumed to operate under a sufficiently
high setting of numeric digits to avoid exponential notation.

\begin{verbatim}
ArithoOp:

arg Numberl, Operator, Number2
/* The Operator will be applied to Numberl and Number2 under the numeric
settings #Digits.#Level, #Form.#Level, #Fuzz.#Level */

/* The result is the result of the operation, or the raising of a 'SYNTAX' or
'LOSTDIGITS' condition. */

/* Variables with digit 1 in their names refer to the first argument of the
operation. Variables with digit 2 refer to the second argument. Variables
with digit 3 refer to the result. */

/* The quotations and page numbers are from the first reference in
Annex C of this standard. */

/* The operands are prepared first. (Page 130) Function Prepare does this,
separating sign, mantissa and exponent. */

v = Prepare (Numberl, #Digits.#Level)
parse var v Signl Mantissal Exponentl

v = Prepare (Number2, #Digits.#Level)
parse var v Sign2 Mantissa2 Exponent2

/* The calculation depends on the operator. The routines set Sign3
Mantissa3 and Exponent3. */

Comparator = ''

select
when Operator == '*' then call Multiply

when Operator
when Operator

' then call DivType
*' then call Power
when Operator %' then call DivType
when Operator '//" then call DivType
otherwise call AddSubComp

end

call PostOp /* Assembles Number3 */

if Comparator \== '' then do
/* Comparison requires the result of subtraction made into a logical */
/* value. */
t = '0'
select
when left (Number3,1) == '-' then
if wordpos(Comparator,'< <= <> >< \= \>') > 0 then t = '1'
when Number3 \== '0' then
if wordpos(Comparator,'> >= <> >< \= \<') > 0 then t = '1'
otherwise
if wordpos(Comparator,'>= = =< \< \>') > 0 then t = '1!
end
Number3 = t
end
return Number3 /* From ArithOp */
/* Activity before every operation: */
Prepare: /* Returns Sign Mantissa and Exponent */
/* Preparation of operands, Page 130 */
/* ",...terms being operated upon have leading zeros removed (noting the

position of any decimal point, and leaving just one zero if all the digits in

the number are zeros) and are then truncated to DIGITS+1 significant digits
(if necessary)..." */

arg Number, Digits

/* Blanks are not significant. */

/* The exponent is separated */

parse upper value space(Number,0) with Mantissa 'E' Exponent
if Exponent == '' then Exponent = '0'

/* The sign is separated and made explicit. */

Sign = '+' /* By default */
if left(Mantissa,1) == '-' then Sign = '-'
if verify (left (Mantissa,1),'+-') = 0 then Mantissa = substr(Mantissa,2)

/* Make the decimal point implicit; remove any actual Point from the
Mantissa. */
Pp = pos('.',Mantissa)
if p > 0 then Mantissa = delstr(Mantissa,p,1)
else p = 1+length (Mantissa)

/* Drop the leading zeros */
do q = 1 to length(Mantissa) - 1

if substr(Mantissa,g,1) \== '0' then leave
p=p-l
end q

Mantissa = substr(Mantissa,q)

/* Detect if Mantissa suggests more significant digits than DIGITS
caters for. */
do j = Digits+1 to length (Mantissa)
if substr(Mantissa,j,1) \== '0' then call #Raise 'LOSTDIGITS', Number
end j

/* Combine exponent with decimal point position, Page 127 */
/* "Exponential notation means that the number includes a power of ten

following an 'E' that indicates how the decimal point will be shifted. Thus

4E9 is just a shorthand way of writing 4000000000 " */

/* Adjust the exponent so that decimal point would be at right of
the Mantissa. */
Exponent = Exponent - (length(Mantissa) - p + 1)

/* Truncate if necessary */

t = length(Mantissa) - (Digits+1)

if t > 0 then do
Exponent = Exponent + t
Mantissa = left (Mantissa,Digits+1)
end

if Mantissa == '0' then Exponent = 0

return Sign Mantissa Exponent

/* Activity after every operation. */
/* The parts of the value are composed into a single string, Number3. */
PostOp:

/* Page 130 */
/* 'traditional' rounding */
t = length(Mantissa3) - #Digits.#Level
if t > 0 then do
/* 'traditional' rounding */
Mantissa3 = left (Mantissa3,#Digits.#Level+1) + 5
if length(Mantissa3) > #Digits.#Level+1 then
/* There was 'carry' */

Exponent3 = Exponent3 + 1

Mantissa3 = left (Mantissa3,#Digits.#Level)

Exponent3 = Exponent3 + t

end
/* "A result of zero is always expressed as a single character '0' "*/
if verify (Mantissa3,'0') = 0 then Number3 = '0'
else do

if Operator == '/' | Operator == '**' then do

/* Page 130 "For division, insignificant trailing zeros are removed
after rounding." */

/* Page 133 "... insignificant trailing zeros are removed." */
do q = length(Mantissa3) by -1 to 2
if substr(Mantissa3,q,1) \== '0' then leave
Exponent3 = Exponent3 + 1
end q
Mantissa3 = substr(Mantissa3,1,q)
end
if Floating() == 'E' then do /* Exponential format */

Exponent3 = Exponent3 + (length(Mantissa3)-1)

/* Page 136 "Engineering notation causes powers of ten to be expressed as a

multiple of 3 - the integer part may therefore range from 1 through
999." */

g=l

if #Form.#Level == 'E' then do

/* Adjustment to make exponent a multiple of 3 */
g = Exponent3//3 /* Recursively using ArithOp as
an external routine. */
if g < 0 then g =g+ 3
Exponent3 = Exponent3 - g
geaqgitl
if length(Mantissa3) < g then
Mantissa3 = left (Mantissa3,g,'0')
end /* Engineering */

/* Exact check on the exponent. */
if Exponent3 > #Limit ExponentDigits then

call #Raise 'SYNTAX', 42.1, Numberl, Operator, Number2
if -#Limit ExponentDigits > Exponent3 then

call #Raise 'SYNTAX', 42.2, Numberl, Operator, Number2

/* Insert any decimal [point. */

if length(Mantissa3) \= g then Mantissa3 = insert('.',Mantissa3,g)
/* Insert the E */
if Exponent3 >= 0 then Number3
else Number3
end /* Exponent format */
else do /* 'pure number' notation */
p = length(Mantissa3) + Exponent3 /* Position of the point within
Mantissa */
/* Add extra zeros needed on the left of the point. */
if p < 1 then do
Mantissa3 = copies('0',1 - p)| |Mantissa3
p=il
end
/* Add needed zeros on the right. */
if p > length(Mantissa3) then
Mantissa3 = Mantissa3||copies('0',p-length (Mantissa3) )
/* Format with decimal point. */
Number3 = Mantissa3

Mantissa3'E+'Exponent3
Mantissa3'E'Exponent3

if p < length(Number3) then Number3 = insert('.',Mantissa3,p)
else Number3 = Mantissa3
end /* pure */
if Sign3 == '-' then Number3 = '-'Number3
end /* Non-Zero */
return
/* This tests whether exponential notation is needed. */
Floating:
/* The rule in the reference has been improved upon. */
Ct = ter
if Exponent3+length(Mantissa3) > #Digits.#Level then t = 'E'
if length(Mantissa3) + Exponent3 < -5 then t = 'E'
return t
/* Add, Subtract and Compare. */

AddSubComp: /* Page 130 */
/* This routine is used for comparisons since comparison is
defined in terms of subtraction. Page 134 */
/* "Numeric comparison is affected by subtracting the two numbers (calculating

the difference) and then comparing the result with '0'." */
NowDigits = #Digits.#Level
if Operator \=='+' & Operator \== '-' then do

Comparator = Operator

/* Page 135 "The effect of NUMERIC FUZZ is to temporarily reduce the value
of NUMERIC DIGITS by the NUMERIC FUZZ value for each numeric comparison" */
NowDigits = NowDigits - #Fuzz.#Level

end
/* Page 130 "If either number is zero then the other number ... is used as
the result (with sign adjustment as appropriate). */
if Mantissa2 == '0' then do /* Result is the lst operand */
Sign3=Signl; Mantissa3 = Mantissal; Exponent3 = Exponentl
return ''
end

if Mantissal
Sign3 = Sign

== '0' then do /* Result is the 2nd operand */
2
if Operator \

; Mantissa3 = Mantissa2; Exponent3 = Exponent2
== '+' then if Sign3 = '+' then Sign3 rat
else Sign3

bat
return ''
end

/* The numbers may need to be shifted into alignment. */

/* Change to make the exponent to reflect a decimal point on the left,

so that right truncation/extension of mantissa doesn't alter exponent. */
Exponentl = Exponentl + length (Mantissal1)
Exponent2 = Exponent2 + length (Mantissa2)

/* Deduce the implied zeros on the left to provide alignment. */

Aligni = 0

Align2 = Exponentl - Exponent2

if Align2 > 0 then do /* Arg 1 provides a more significant digit */
Align2 = min(Align2,NowDigits+1) /* No point in shifting further. */
/* Shift to give Arg2 the same exponent as Argl */

Mantissa2 = copies('0',Align2) || Mantissa2
Exponent2 = Exponentl
end

if Align2 < 0 then do /* Arg 2 provides a more significant digit */
/* Shift to give Argl the same exponent as Arg2 */

Aligni = -Align2

Alignl = min(Align1l,NowDigits+1) /* No point in shifting further. */
Align2 = 0

Mantissal = copies('0',Alignl) || Mantissal

Exponentl = Exponent2

end

/* Maximum working digits is NowDigits+1. Footnote 41. */

SigDigits
SigDigits

max (length (Mantissal) , length (Mantissaz2) )
min (SigDigits,NowDigits+1)

/* Extend a mantissa with right zeros, if necessary. */
Mantissal = left (Mantissal,SigDigits,'0')

Mantissa2 = left (Mantissa2,SigDigits,'0')

/* The exponents are adjusted so that

the working numbers are integers, ie decimal point on the right. */

Exponent3 = Exponentl-SigDigits
Exponentl = Exponent3
Exponent2 = Exponent3
if Operator = '+' then
Mantissa3 = (Signl || Mantissal) + (Sign2 || Mantissa2)

else Mantissa3 (Signl || Mantigsal) - (Sign2 || Mantissa2)

/* Separate the sign */
if Mantissa3 < 0 then do

Sign3 = '-'
Mantissa3 = substr (Mantissa3,2)
end

else Sign3 = '+'

/* "The result is then rounded to NUMERIC DIGITS digits if necessary,
taking into account any extra (carry) digit on the left after addition,
but otherwise counting from the position corresponding to the most
significant digit of the terms being added or subtracted." */

if length(Mantissa3) > SigDigits then SigDigits = SigDigits+l
d = SigDigits - NowDigits /* Digits to drop. */
if d <= 0 then return
t = length(Mantissa3) - d /* Digits to keep. */
/* Page 130. "values of 5 through 9 are rounded up, values of 0 through 4 are
rounded down." */
if t > 0 then do
/* 'traditional' rounding */
Mantissa3 = left(Mantissa3, t +1) +5
if length(Mantissa3) > t+1 then
/* There was 'carry' */
/* Keep the extra digit unless it takes us over the limit. */
if t < NowDigits then t = t+l
else Exponent3 = Exponent3+1
Mantissa3 = left (Mantissa3,t)
Exponent3 = Exponent3 + d
end /* Rounding */
else Mantissa3 = '0'
return /* From AddSubComp */

/* Multiply operation: */

Multiply: /* p 131 */

/* Note the sign of the result */

if Signl == Sign2 then Sign3 = '+'
else Sign3 = '-'
/* Note the exponent */
Exponent3 = Exponentl + Exponent2
if Mantissal == '0' then do
Mantissa3 = '0'
return
end
/* Multiply the Mantissas */
Mantissa3 = ''

do q=1 to length (Mantissa2)
Mantissa3 = Mantissa3'0'
do substr(Mantissa2,q,1)
Mantissa3 = Mantissa3 + Mantissal
end
end q
return /* From Multiply */

/* Types of Division: */

DivType: /* p 131 */
/* Check for divide-by-zero */

if Mantissa2 == '0' then call #Raise 'SYNTAX',

/* Note the exponent of the result */
Exponent3 = Exponentl - Exponent2

/* Compute (one less than) how many digits will be in the integer

part of the result. */

IntDigits = length(Mantissal) - Length(Mantissa2) + Exponent3
/* In some cases, the result is known to be zero.
if Mantigsal = 0 | (IntDigits < 0 & Operator

Mantissa3 = 0
Sign3 = '+'
Exponent3 = 0
return
end
/* In some cases, the result is known to be to be the first argument.
if IntDigits < 0 & Operator == '//' then do
Mantissa3 = Mantissal
Sign3 = Signl
Exponent3 = Exponentl
return
end
/* Note the sign of the result. */
if Signl == Sign2 then Sign3 = '+'
else Sign3 = '-'

/* Make Mantissal at least as large as Mantissa2 so Mantissa2 can be
subtracted without causing leading zero to result. Page 131 */

az 0
do while Mantissa2 > Mantissal

Mantissal = Mantissal'0'
Exponent3 = Exponent3 - 1
aztadtl
end
/* Traditional divide */
Mantissa3 = ''

/* Subtract from part of Mantissal that has length of Mantissa2 */

left (Mantissal,length(Mantissa2) )
substr (Mantissal, length (Mantissa2)+1)
o forever

x
Y
d

/* Develop a single digit in z by repeated subtraction.

ze=O0
do forever
xX = kK - Mantissa2
if left(x,1) == '-' then leave
Zeze¢t+tl
end

x = x + Mantissa2 /* Recover from over-subtraction */
/* The digit becomes part of the result */

Mantissa3 = Mantissa3 || z

if Mantissa3 == '0' then Mantissa3 = '' /* A single leading


##

*f
'%') then do

*/
zero can happen. */
/* x||y is the current residue */
if y == '' then if x = 0 then leave /* Remainder is zero */

if length(Mantissa3) > #Digits.#Level then leave /* Enough digits
in the result */

/* Check type of division */
if Operator \== '/' then do
if IntDigits = 0 then leave
IntDigits = IntDigits - 1
end
/* Prepare for next digit */
/* Digits come from y, until that is exhausted. */
/* When y is exhausted an extra zero is added to Mantissal */
if y == '' then do
y = ror
Exponent3 = Exponent3 - 1
aztadtl
end
xX = xX | | left (y,1)
y = substr(y,2)
end /* Iterate for next digit. */

Remainder = x || y
Exponent3 = Exponent3 + length(y) /* The loop may have been left early.
/* Leading zeros are taken off the Remainder. */
do while length(Remainder) > 1 & Left (Remainder,1) == '0'
Remainder = substr (Remainder, 2)
end
if Operator \== '/' then do
/* Check whether % would fail, even if operation is // */

if Floating() 'E' then do
if Operator '%' then MsgNum
else MsgNum

/* Page 133. % could fail by needing exponential notation */

26.11
26.12

call #Raise 'SYNTAX', MsgNum, Numberl , Number2, #Digits.#Level

end
end
if Operator == '//' then do
/* We need the remainder */
Sign3 = Signl

Mantissa3 = Remainder
Exponent3 = Exponentl - a
end

return /* From DivType */

/* The Power operation: */

Power: /* page 132 */
/* The second argument should be an integer */
if \WholeNumber2() then call #Raise 'SYNTAX', 26.8, Number2
/* Lhs to power zero is always 1 */
if Mantissa2 == '0' then do
Sign3 = '+'
Mantissa3
Exponent3
return
end

i
ror

/* Pages 132-133 The Power algorithm */
Rhs = left (Mantissa2,length(Mantissa2)+Exponent2,'0')/* Explicit
integer form */
L length (Rhs)
b X2B(D2X(Rhs)) /* Makes Rhs in binary notation */
/* Ignore initial zeros */
do q=l1byl
if substr(b,q,1) \== '0' then leave
end q
ael
do forever
/* Page 133 "Using a precision of DIGITS+L+1" */
if substr(b,q,1) == '1' then do
a = Recursion('*',Signl || Mantissal'E'Exponent1)


*/
if left(a,2) == 'MN' then signal PowerFailed
end

/* Check for finished */

if q = length(b) then leave

/* Square a */

a = Recursion('*',a)
if left(a,2) == 'MN' then signal PowerFailed
q=qrtil
end
/* Divide into one for negative power */
if Sign2 == '-' then do
Sign2 = '+'
a = Recursion('/')
if left(a,2) == 'MN' then signal PowerFailed
end

/* Split the value up so that PostOp can put it together with rounding */
Parse value Prepare(a,#Digits.#Level+L+1) with Sign3 Mantissa3 Exponent3
return

PowerFailed:
/* Distinquish overflow and underflow */
ReWas = substr(a,4)
if Sign2 = '-' then if ReWas == '42.1' then RcWas
else RcWas
call #Raise 'SYNTAX', RcWas, Numberl, '**', Number2
/* No return */

"42.2!
"42.1!

WholeNumber2:
numeric digits Digits
if #Form.#Level == 'S' then numeric form scientific

else numeric form engineering
return datatype (Number2, 'W')

Recursion: /* Called only from '**! */
numeric digits #Digits.#Level + L + 1
signal on syntax name Overflowed
/* Uses ArithOp again under new numeric settings. */
if arg(1) == '/' then return 1l/a
else return a * arg(2)
Over flowed:
return 'MN '.MN
\end{verbatim}

\hypertarget{functions}{%
\section{Functions}\label{functions}}

\hypertarget{invocation}{%
\subsection{Invocation}\label{invocation}}

Invocation occurs when a function or a message\_term or a callis
evaluated. Invocation of a function may result in a value, in which
case:

if \#Tracing.\#Level == `I' then call \#Trace
`\textgreater F\textgreater{}' Invocation of a message\_term may result
in a value, in which case: if \#Tracing.\#Level == `I' then call \#Trace
`\textgreater M\textgreater{}'

\hypertarget{evaluation-of-arguments}{%
\subsection{Evaluation of arguments}\label{evaluation-of-arguments}}

The argument positions are the positions in the exoression\_list where
syntactically an expression occurs or could have occurred. Let ArgNumber
be the number of an argument position, counting from 1 at the left; the
range of ArgNumber is all whole numbers greater than zero.

For each value of ArgNumber, \#ArgExists.\#NewLevel.ArgNumber is set `1'
if there is an expression present, `O' if not.

From the left, if \#ArgExists.\#NewLevel.ArgNumber is `1' then
\#Arg.\#NewLevel.ArgNumber is set to the value of the corresponding
expression. If \#ArgExists.\#NewLevel.ArgNumber is `0' then
\#Arg.\#NewLevel.ArgNumber is set to the null string.

\#ArgExists.\#NewLevel.0 is set to the largest ArgNumber for which
\#ArgExists.\#NewLevel.ArgNumber is `1', or to zero if there is no such
value of ArgNumber.

\hypertarget{the-value-of-a-label}{%
\subsection{The value of a label}\label{the-value-of-a-label}}

The value of a LABEL, or of the taken\_constant in the function or
call\_instruction, is taken as a constant, see nnn. If the
taken\_constant is not a string\_literal it is a reference to the first
LABEL in the program which has the same value. The comparison is made
with the `==' operator.

If there is such a matching label and the label is trace-only (see nnn)
then a condition is raised: call \#Raise `SYNTAX', 16.3, taken constant

If there is such a matching label, and the label is not trace-only,
execution continues at the label with routine initialization (see nnn).
This is execution of an internal routine.

If there is no such matching label, or if the taken\_constant is a
string\_literal, further comparisons are made.

If the value of the taken\_constant matches the name of some built-in
function then that built-in function is invoked. The names of the
built-in functions are defined in section nnn and are in uppercase.

If the value does not match any built-in function name,
Config\_ExternalRoutine is used to invoke an external routine.

Whenever a matching label is found, the variables SIGL and .SIGL are
assigned the value of the line number of the clause which caused the
search for the label. In the case of an invocation resulting from a

condition occurring that shall be the clause in which the condition
occurred. Var \_ Set(\#Pool, `SIGL', `0', \#LineNumber) var Set(0 ,
`.SIGL', `0', \#LineNumber)

The name used in the invocation is held in \#Name.\#Level for possible
use in an error message from the RETURN clause, see nnn

\hypertarget{the-value-of-a-function}{%
\subsection{The value of a function}\label{the-value-of-a-function}}

A built-in function completes when it returns from the activity defined
in section nnn. The value of a built-in function is defined in section
nnn.

An internal routine completes when \#Level returns to the value it had
when the routine was invoked. The value of the internal function is the
value of the expression on the return which completed the routine. The
value of an external function is determined by Config\_ExternalRoutine.

\hypertarget{the-value-of-a-method}{%
\subsection{The value of a method}\label{the-value-of-a-method}}

A built-in method completes when it returns from the activity defined in
section n.~The value of a built-in method is defined in section n.

An internal method completes when \#Level returns to the value it had
when the routine was invoked. The value of the internal method is the
value of the expression on the return which completed the method. The
value of an external method is determined by Config\_ExternalMethod.

\hypertarget{the-value-of-a-message-term}{%
\subsection{The value of a message
term}\label{the-value-of-a-message-term}}

See nnn for the syntax of a message\_term. The value of the ferm within
a message\_term is called the receiver.

The receiver and any arguments of the term are evaluated, in left to
right order. r= \#evaluate(message term, term) If the message term
contains `\textasciitilde\textasciitilde{}' the value of the message
term is the receiver. Any effect on .Result? Otherwise the value of a
message\_term is the value of the method it invokes. The method invoked
is determined by the receiver and the taken\_constant and symbol. t =
\#Instance(message term, taken constant) If there is a symbol, it is
subject to a constraints. if \#contains (message term, symbol) then do
if r \textless\textgreater{} \#Self then

call \#Raise `SYNTAX', nn.n

/* OOI: ``Message search overrides can only be used from methods of the
target object.'' */

The search will progress from the object to its class and superclasses.
/* This is going to be circular because it describes the message lookup
algorithm and also uses messages. However for the messages in this code
the message names are chosen to be unique to a method so there is no
need to use this algorithm in deciding which method is intended. */

/* message term ::= receiver `\textasciitilde{}' taken constant `:'
VAR\_SYMBOL arguments */

/* This code reflects OOI - the arguments on the message don't affect
the method choice. */

/* This code selects a method based on its arguments, receiver,
taken\_constant, and symbol. */

/* This code is used in a context where \#Self is the receiver of the
method invocation which the subject message term is running under. */

\begin{verbatim}
SelectMethod:

/* If symbol given, receiver must be self. */
if arg(3,'E') then if arg(1)\==#Self then signal error /* syntax number? */

t arg(2) /* Will have been uppercased, unless a literal. */

x arg(1) /* Cursor through places to look for the method. */

Mixing 1 /* Off for potential mixins ignored because symbol given. */
Mixins -array~new /* to note any Mixins involved. */

/* Look in the method table of the object, if no 'symbol' given. */
if arg(3,'E') then do
Mixing = 0

end
else do
m = x~#MethodTable[t]
if m \== .nil then return m
end

do until x==.object
/* Follow the class hierarchy. */
x = x-class
/* Note any mixins for later reference. */
Mix = x~Inherited /* An array, ordered as the directive left-to-right. */

if Mix \== .nil then /* Append to the record. */
do j=1 to Mix~dimension (1)
Mixins [Mixins~dimension(1)+1] = Mix[j]
end

if Mixing do
/* Consider mixins only for superclasses of 'symbol'. */
do j=1 to Mixins~dimension (1)
/* Look at the baseclass of each. */
/* That is closest superclass not a mixin. */
s = Mixins[j]~class
do while s~Mixin /* Assert stop at .object if not before. */
s=s~class
end
if s==x then do
m=Mixins [j]~#MethodTable[t]
if m \== .nil then return m
end
end j
end /* Mixing */
if arg(3,'E') then if arg(3)==x then do
Mixing=1
end
if Mixing do
/* Consider non-Mixins */
m= x-#InstanceMethodTable[t]

if m \== .nil then return m
end

x=x~superclass

end

/* Try for UNKNOWN instead */
if t == 'UNKNOWN' then return .nil
if \arg(3,'E') then return SelectMethod arg(1),'UNKNOWN'
else return SelectMethod arg(1),'UNKNOWN',arg(3)
\end{verbatim}

\hypertarget{use-of-config_externalroutine}{%
\subsection{Use of
Config\_ExternalRoutine}\label{use-of-config_externalroutine}}

The values of the arguments to the use of Config\_ExternalRoutine, in
order, are:

The argument How is `SUBROUTINE' if the invocation is from a call,
```FUNCTION' if the invocation is from a function.

The argument NameType is `1' if the taken\_constant is a
string\_literal, `0' otherwise.

The argument Name is the value of the faken\_constant.

The argument Environment is the value of this argument on the API\_
Start which started this execution. The argument Arguments is the \#Arg.
and \#ArgExists. data.

The argument Streams is the value of this argument on the API\_Start
which started this execution.

The argument Traps is the value of this argument on the API\_Start which
started this execution. Var\_Reset is invoked and \#API\_Enabled set to
`1' before use of Config\_ExternalRoutine. \#API\_Enabled is set to `O'
after.

The response from Config\_ExternalRoutine is processed. If no conditions
are (implicitly) raised, \#Outcome is the value of the function.
