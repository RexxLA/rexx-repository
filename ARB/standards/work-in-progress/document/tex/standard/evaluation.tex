%preprocessed texin
\chapter{Evaluation}\label{evaluation}

The syntax section describes how expressions and the components of
expressions are written in a program. It also describes how operators
can be associated with the strings, symbols and function results which
are their operands.

This evaluation section describes what values these components have in
execution, or how they have no value because a condition is raised.

This section refers to the \texttt{DATATYPE} built-in function when
checking operands, see nnn. Except for considerations of limits on the
values of exponents, the test:

\lstinputlisting[language=rexx,label=datatypenum.rexx,caption=datatypenum.rexx]{datatypenum.rexx}

is equivalent to testing whether the subject matches the syntax:

\lstinputlisting[language=rexx,label=ebfdatatype.ebnf,caption=ebfdatatype.ebnf]{ebfdatatype.ebnf}

For the syntax of \emph{number} see nnn.

When the matching subject does not include a
\texttt{\textquotesingle{}-\textquotesingle{}} the value is the value of
the number in the match, otherwise the value is the value of the
expression \texttt{(0\ -\ number)}.

The test:

\lstinputlisting[language=rexx,label=datatypesubject.rexx,caption=datatypesubject.rexx]{datatypesubject.rexx}

is a test that the \texttt{Subject} matches that syntax and also has a
value that is ``whole'', that is has no non-zero fractional part.

When these two tests are made and the \texttt{Subject} matches the
constraints but has an exponent that is not in the correct range of
values then a condition is raised:

\lstinputlisting[language=rexx,label=callraisesyntax.rexx,caption=callraisesyntax.rexx]{callraisesyntax.rexx}

This possibility is implied by the uses of \texttt{DATATYPE} and not
shown explicitly in the rest of this section nnn.

\section{Variables}\label{variables}

The values of variables are held in variable pools. The capabilities of
variable pools are listed here, together with the way each function will
be referenced in this definition.

The notation used here is the same as that defined in sections nnn and
nnn, including the fact that the \texttt{Var\_} routines may return an
indicator of \texttt{\textquotesingle{}N\textquotesingle{}},
\texttt{\textquotesingle{}S\textquotesingle{}} or
\texttt{\textquotesingle{}X\textquotesingle{}}.

Each possible name in a variable pool is qualified as tailed or
non-tailed name; names with different qualification and the same
spelling are different items in the pool. For those \texttt{Var\_}
functions with a third argument this argument indicates the
qualification; it is \texttt{\textquotesingle{}1\textquotesingle{}} when
addressing tailed names or
\texttt{\textquotesingle{}0\textquotesingle{}} when addressing
non-tailed names.

Each item in a variable pool is associated with three attributes and a
value. The attributes are `dropped' or `not-dropped', `exposed' or
`not-exposed' and `implicit' or `not-implicit'.

A variable pool is associated with a reference denoted by the first
argument, with name \texttt{Pool}. The value of \texttt{Pool} may alter
during execution. The same name, in conjunction with different values of
\texttt{Pool}, can correspond to different values.

\subsection{Var\_Empty}\label{var_empty}

\lstinputlisting[language=rexx,label=varpoolempty.rexx,caption=varpoolempty.rexx]{varpoolempty.rexx}

The function sets the variable pool associated with the specified
reference to the state where every name is associated with attributes
`dropped', `implicit' and `not-exposed'.

\subsection{Var Set}\label{var-set}

\lstinputlisting[language=rexx,label=varsetzero.rexx,caption=varsetzero.rexx]{varsetzero.rexx}

The function operates on the variable pool with the specified reference.
The name is a non-tailed name. If the specified name has the `exposed'
attribute then \texttt{Var\_Set} operates on the variable pool
referenced by \texttt{\#Upper} in this pool and this rule is applied to
that pool. When the pool with attribute `not-exposed' for this name is
determined the specified value is associated with the specified name. It
also associates the attributes `not-dropped' and `not-implicit'. If that
attribute was previously `not-dropped' then the indicator returned is
\texttt{\textquotesingle{}R\textquotesingle{}}. The name is a stem if it
contains just one period, as its rightmost character. When the name is a
stem
\texttt{Var\_Set(Pool,\ TailedName,\ \textquotesingle{}1\textquotesingle{},Value)}
is executed for all possible valid tailed names which have Name as their
stem, and then those tailed-names are given the attribute `implicit'.

\lstinputlisting[language=rexx,label=varsetone.rexx,caption=varsetone.rexx]{varsetone.rexx}

The function operates on the variable pool with the specified reference.
The name is a tailed name. The left part of the name, up to and
including the first period, is the stem. The stem is a non-tailed name.
If the specified stem has the `exposed' attribute then Var\_Set operates
on the variable pool referenced by \#Upper in this pool and this rule is
applied to that pool. When the pool with attribute `not-exposed' for the
stem is determined the name is considered in that pool. If the name has
the `exposed' attribute then the variable pool referenced by \#Upper in
the pool is considered and this rule applied to that pool. When the pool
with attribute `not-exposed' is determined the specified value is
associated with the specified name. It also associates the attributes
`not-dropped' and `not-implicit' . If that attribute was previously
`not-dropped' then the indicator returned is
\texttt{\textquotesingle{}R\textquotesingle{}}.

\subsection{Var\_Value}\label{var_value}

\lstinputlisting[language=rexx,label=varvaluezero.rexx,caption=varvaluezero.rexx]{varvaluezero.rexx}

The function operates on the variable pool with the specified reference.
The name is a non-tailed name. If the specified name has the `exposed'
attribute then Var\_Value operates on the variable pool referenced by
\texttt{\#Upper} in this pool and this rule is applied to that pool.
When the pool with attribute `not-exposed' for this name is determined
the indicator returned is `D' if the name has `dropped' associated,
\texttt{\textquotesingle{}N\textquotesingle{}} otherwise.

In the former case \texttt{\#Outcome} is set equal to \texttt{Name}, in
the latter case \texttt{\#Outcome} is set to the value most recently
associated with the name by \texttt{Var\_Set}.

\lstinputlisting[language=rexx,label=varvalueone.rexx,caption=varvalueone.rexx]{varvalueone.rexx}

The function operates on the variable pool with the specified reference.
The name is a tailed name. The left part of the name, up to and
including the first period, is the stem. The stem is a non-tailed name.
If the specified stem has the `exposed' attribute then
\texttt{Var\_Value} operates on the variable pool referenced by
\texttt{\#Upper} in this pool and this rule is applied to that pool.
When the pool with attribute `not-exposed' for the stem is determined
the name is considered in that pool. If the name has the `exposed'
attribute then the variable pool referenced by \#Upper in the pool is
considered and this rule applied to that pool. When the pool with
attribute `not-exposed' is determined the indicator returned is
\texttt{\textquotesingle{}D\textquotesingle{}} if the name has `dropped'
associated, \texttt{\textquotesingle{}N\textquotesingle{}} otherwise. In
the former case \texttt{\#Outcome} is set equal to \texttt{Name}, in the
latter case \texttt{\#Outcome} is set to the value most recently
associated with the name by \texttt{Var\_Set}.

\subsection{Var\_Drop}\label{var_drop}

\lstinputlisting[language=rexx,label=vardropzero.rexx,caption=vardropzero.rexx]{vardropzero.rexx}

The function operates on the variable pool with the specified reference.
The name is a non-tailed name. If the specified name has the `exposed'
attribute then \texttt{Var\_Drop} operates on the variable pool
referenced by \texttt{\#Upper} in this pool and this rule is applied to
that pool. When the pool with attribute `not-exposed' for this name is
determined the attribute `dropped' is associated with the specified
name. Also, when the name is a stem,
\texttt{Var\_Drop(Pool,TailedName,\textquotesingle{}1\textquotesingle{})}
is executed for all possible valid tailed names which have \texttt{Name}
as a stem.

\lstinputlisting[language=rexx,label=vardropone.rexx,caption=vardropone.rexx]{vardropone.rexx}

The function operates on the variable pool with the specified reference.
The name is a tailed name. The left part of the name, up to and
including the first period, is the stem. The stem is a non-tailed name.
If the specified stem has the `exposed' attribute then Var\_Drop
operates on the variable pool referenced by \#Upper in this pool and
this rule is applied to that pool. When the pool with attribute
`not-exposed' for the stem is determined the name is considered in that
pool. If the name has the `exposed' attribute then the variable pool
referenced by \texttt{\#Upper} in the pool is considered and this rule
applied to that pool. When the pool with attribute `not-exposed' is
determined the attribute `dropped' is associated with the specified
name.

\subsection{Var\_Expose}\label{var_expose}

\lstinputlisting[language=rexx,label=varexposezero.rexx,caption=varexposezero.rexx]{varexposezero.rexx}

The function operates on the variable pool with the specified reference.
The name is a non-tailed name. The attribute `exposed' is associated
with the specified name. Also, when the name is a stem,
\texttt{Var\_Expose(Pool,\ TailedName,\textquotesingle{}1\textquotesingle{})}
is executed for all possible valid tailed names which have Name as a
stem.

\lstinputlisting[language=rexx,label=varexpose1.rexx,caption=varexpose1.rexx]{varexpose1.rexx}

The function operates on the variable pool with the specified reference.
The name is a tailed name. The attribute `exposed' is associated with
the specified name.

\subsection{Var\_Reset}\label{var_reset}

\lstinputlisting[language=rexx,label=varresetpool.rexx,caption=varresetpool.rexx]{varresetpool.rexx}

The function operates on the variable pool with the specified reference.
It establishes the effect of subsequent \texttt{API\_Next} and
\texttt{API\_NextVariable} functions (see sections nnn and nnn). A
\texttt{Var\_Reset} is implied by any \texttt{API\_} operation other
than \texttt{API\_Next} and \texttt{API\_NextVariable}.

\section{Symbols}\label{symbols}

For the syntax of a symbol see nnn.

The value of a symbol which is a \emph{NUMBER} or a \emph{CONST\_SYMBOL}
which is not a reserved symbol is the content of the appropriate token.

The value of a \emph{VAR\_SYMBOL} which is ``taken as a constant'' is
the \emph{VAR\_SYMBOL} itself, otherwise the \_VAR\_SYMBOL: identifies a
variable and its value may vary during execution.

Accessing the value of a symbol which is not ``taken as a constant''
shall result in trace output, see nnn:

\lstinputlisting[language=rexx,label=tracinglevel.rexx,caption=tracinglevel.rexx]{tracinglevel.rexx}

where \texttt{Tag} is
\texttt{\textquotesingle{}\textgreater{}L\textgreater{}\textquotesingle{}}
unless the symbol is a \emph{VAR\_SYMBOL} which, when used as an
argument to \texttt{Var\_Value}, does not yield an indicator
\texttt{\textquotesingle{}D\textquotesingle{}}. In that case, the Tag is
\texttt{\textquotesingle{}\textgreater{}V\textgreater{}\textquotesingle{}}.

\section{Value of a variable}\label{value-of-a-variable}

If \emph{VAR\_SYMBOL} does not contain a period, or contains only one
period as its last character, the value of the variable is the value
associated with \emph{VAR\_SYMBOL} in the variable pool, that is
\texttt{\#Outcome} after
\texttt{Var\_Value(Pool,\ VAR\_SYMBOL,\ \textquotesingle{}0\textquotesingle{})}.

If the indicator is \texttt{\textquotesingle{}D\textquotesingle{}},
indicating the variable has the `dropped' attribute, the
\texttt{NOVALUE} condition is raised; see nnn and nnn for exceptions to
this.

\lstinputlisting[language=rexx,label=varpoolsymbol.rexx,caption=varpoolsymbol.rexx]{varpoolsymbol.rexx}

If \emph{VAR\_SYMBOL} contains a period which is not its last character,
the value of the variable is the value associated with the derived name.

\subsection{Derived names}\label{derived-names}

A derived name is derived from a \emph{VAR\_SYMBOL} as follows:

\lstinputlisting[language=rexx,label=derivednames.rexx,caption=derivednames.rexx]{derivednames.rexx}

The derived name is the concatenation of:

\begin{itemize}
\tightlist
\item
  the \emph{Stem}, without further evaluation;
\item
  the \emph{Tail}, with the \emph{PlainSymbols} replaced by the values
  of the symbols. The value of a \emph{PlainSymbol} which does not start
  with a digit is \texttt{\#Outcome} after
\end{itemize}

\lstinputlisting[language=rexx,label=varvalueplainsymbol.rexx,caption=varvalueplainsymbol.rexx]{varvalueplainsymbol.rexx}

These values are obtained without raising the \texttt{NOVALUE}
condition.

If the indicator from the Var\_Value was not `D' then: if
\#Tracing.\#Level == `I' then call \#Trace
`\textgreater C\textgreater{}'

The value associated with a derived name is obtained from the variable
pool, that is \#Outcome after: Var\_Value(Pool,Derived Name,`1')

If the indicator is `D', indicating the variable has the `dropped'
attribute, the NOVALUE condition is raised; see nnn for an exception.

\subsection{Value of a reserved
symbol}\label{value-of-a-reserved-symbol}

The value of a reserved symbol is the value of a variable with the
corresponding name in the reserved pool, see nnn.

\section{Expressions and operators}\label{expressions-and-operators}

Add a load of string coercions. Equality can operate on non-strings.
What if one operand non-string?

\subsection{The value of a term}\label{the-value-of-a-term}

See nnn for the syntax of a term.

The value of a STRING is the content of the token; see nnn.

The value of a function is the value it returns, see nnn.

If a termis a symbol or STRING then the value of the term is the value
of that symbol or STRING.

If a term contains an expr\_alias the value of the term is the value of
the expr\_alias, see nnn.

\subsection{The value of a
prefix\_expression}\label{the-value-of-a-prefix_expression}

If the prefix\_expression is a term then the value of the
prefix\_expression is the value of the ferm, otherwise let rhs be the
value of the prefix\_expression within it\_\_ see nnn

If the prefix\_expression has the form `+' prefix\_expression then a
check is made: if datatype(rhs)==`NUM' then call \#Raise `SYNTAX',41.3,
rhs, `+'

and the value is the value of (0 + rhs).

If the prefix\_expression has the form `-' prefix\_expression then a
check is made: if datatype(rhs)==`NUM' then call \#Raise
`SYNTAX',41.3,rhs, `-'

and the value is the value of (0 - rhs).

If a prefix\_expression has the form not prefix\_expression then if rhs
== `0' then if rhs ==`1' then call \#Raise `SYNTAX', 34.6, not, rhs

See nnn for the value of the third argument to that \#Raise. If the
value of rhs is `0' then the value of the prefix\_expression value is
`1', otherwise it is `0'.

If the prefix\_expression is not a term then: if \#Tracing.\#Level ==
`I' then call \#Trace `\textgreater P\textgreater{}'

\subsection{The value of a
power\_expression}\label{the-value-of-a-power_expression}

See nnn for the syntax of a power\_expression.

If the power\_expression is a prefix\_expression then the value of the
power\_expression is the value of the prefix\_expression.

Otherwise, let Ihs be the value of power\_expression within it, and rhs
be the value of prefix\_expression within it.

\lstinputlisting[language=rexx,label=evaluation-datatype.rexx,caption=evaluation-datatype.rexx]{evaluation-datatype.rexx}
power\_expression is

ArithOp(lhs,'**',rhs)

If the power\_expression is not a prefix\_expression then: if
\#Tracing.\#Level == `I' then call \#Trace
`\textgreater0O\textgreater{}'

\subsection{The value of a
multiplication}\label{the-value-of-a-multiplication}

See nnn for the syntax of a multiplication. If the multiplication is a
power\_expression then the value of the multiplication is the value of
the power\_expression. Otherwise, let Ihs be the value of multiplication
within it, and rns be the value of power\_expression within it. if
datatype(lhs)==`NUM' then call \#Raise `SYNTAX',41.1,lhs,multiplicative
operation if datatype(rhs)==`NUM' then call \#Raise
`SYNTAX',41.2,rhs,multiplicative operation

The value of the multiplication is ArithOp(lhs,multiplicative operation,
rhs)

If the multiplication is not a power\_expression then:

if \#Tracing.\#Level == `I' then call \#Trace
`\textgreater0O\textgreater{}'

\subsection{The value of an addition}\label{the-value-of-an-addition}

See nnn for the syntax of addition.

If the addition is a multiplication then the value of the addition is
the value of the multiplication. Otherwise, let Ihs be the value of
ad¢difion within it, and rhs be the value of the multiplication within
it. Let

operation be the adaltive\_operator. if datatype(lhs)==`NUM' then

call \#Raise `SYNTAX', 41.1, lhs, operation if datatype(rhs)==`NUM' then

call \#Raise `SYNTAX', 41.2, rhs, operation

If either of rhs or Ihs is not an integer then the value of the addition
is ArithOp(lhs, operation, rhs) Otherwise if the operation is `+' and
the length of the integer Ihs+rhs is not greater than \#Digits.\#Level

then the value of addition is lhs+rhs

Otherwise if the operation is `-' and the length of the integer Ihs-rhs
is not greater than \#Digits.\#Level then

the value of addition is lhs-rhs

Otherwise the value of the addition is ArithOp(lhs, operation, rhs)

If the addition is not a multiplication then: if \#Tracing.\#Level ==
`I' then call \#Trace `\textgreater0O\textgreater{}'

\subsection{The value of a
concatenation}\label{the-value-of-a-concatenation}

See nnn for the syntax of a concatenation. If the concatenation is an
addition then the value of the concatenation is the value of the
addition. Otherwise, let Ihs be the value of concatenation within it,
and rhs be the value of the additive\_expression within it. If the
concatenation contains `\textbar\textbar{}' then the value of the
concatenation will have the following characteristics:

\begin{itemize}
\item
  Config\_Length(Value) will be equal to
  Config\_Length(Ihs)+Config\_Length(rhs).
\item
  \#Outcome will be `equal' after each of:
\item
  Config\_Compare(Config\_Subsir(Ihs,n)\},Config\_Subsitr(Value,n)) for
  values of n not less than 1 and not more than Config\_Length(Ihs);
\item
  Config\_Compare(Config\_Subsir(rhs,n),Config\_Substr(Value,Config\_Length(Ihs)+n))
  for values of n not less than 1 and not more than Config\_Length(rhs).
  Otherwise the value of the concatenation will have the following
  characteristics:
\item
  Config\_Length(Value) will be equal to
  Config\_Length(Ihs)+1+Config\_Length(rhs).
\item
  \#Outcome will be `equal' after each of:
\item
  Config\_Compare(Config\_Subsir(Ihs,n)\},Config\_Subsitr(Value,n)) for
  values of n not less than 1 and not more than Config\_Length(Ihs);
\item
  Config\_Compare(' ',Config\_Substr(Value,Config\_Length(Ihs)\}+1));
\item
  Config\_Compare(Config\_Subsitr(rhs,n),Config\_Substr(Value,Config\_Length(Ins)+1+n))
  for values of n not less than 1 and not more than Config\_Length(rhs).
\end{itemize}

If the concatenation is not an addition then: if \#Tracing.\#Level ==
`I' then call \#Trace `\textgreater0O\textgreater{}'

\subsection{The value of a comparison}\label{the-value-of-a-comparison}

See nnn for the syntax of a comparison.

If the comparison is a concatenation then the value of the comparison is
the value of the concatenation. Otherwise, let Ihs be the value of the
comparison within it, and rns be the value of the concatenation within
it.

If the comparison has a comparison\_operator that is a strict\_compare
then the variable \#Test is set as follows:

\#Test is set to `E'. Let Length be the smaller of Config\_Length(Ihs)
and Config\_Length(rhs). For values of n greater than O and not greater
than Length, if any, in ascending order, \#Test is set to the uppercased
first character of \#Outcome after:

Config\_Compare(Config\_Subsir(Ihs),Contfig\_Subsir(rhs)).

If at any stage this sets \#Test to a value other than `E' then the
setting of \#Test is complete. Otherwise, if Config\_Length(Ihs) is
greater than Config\_Length(rhs) then \#Test is set to `G' or if
Config\_Length(Ihs) is less than Config\_Length(rhs) then \#Test is set
to `L'.

If the comparison has a comparison\_operator that is a normal\_compare
then the variable \#Test is set as follows:

\lstinputlisting[language=rexx,label=evaluation-comparison.rexx,caption=evaluation-comparison.rexx]{evaluation-comparison.rexx}

The value of \#Test, in conjunction with the operator in the comparison,
determines the value of the comparison. The value of the comparison is
`1' if - \#Test is `E' and the operator is one of `=``\,``,'==``\,``,
`\textgreater=', \textless='', `\textgreater{}', `\textless{}',
`p\textgreater=', `\textless\textless=', \textgreater\textgreater', or
\textless\textless)

\begin{itemize}
\tightlist
\item
  \#Test is `G' and the operator is one of `\textgreater{}',
  `\textgreater=``, `\textless{}', `=', `\textless\textgreater{}',
  `\textgreater\textless{}', Nes'','\textgreater\textgreater!
  `p\textgreater{}', or \textless\textless``)
\item
  \#Test is `L' and the operator is one of `\textless{}', \textless=``,
  \textgreater{}`, =', `\textless\textgreater{}',
  `\textgreater\textless{}', ==`,'\textless\textless',
  *\textless\textless=`, or \textgreater\textgreater{}'. In all other
  cases the value of the comparison is `0'.
\end{itemize}

If the comparison is not a concatenation then: if \#Tracing.\#Level ==
`I' then call \#Trace `\textgreater0O\textgreater{}'

\subsection{The value of an
and\_expression}\label{the-value-of-an-and_expression}

See nnn for the syntax of an and\_expression.

If the and\_expression is a comparison then the value of the
and\_expression is the value of the comparison.

Otherwise, let Ihs be the value of the and\_expression within it, and
rhs be the value of the comparison within it.

if lhs == `0' then if lhs == `1' then call \#Raise
`SYNTAX',34.5,lhs,`\&'

if rhs == `0' then if rhs == `1' then call \#Raise
`SYNTAX',34.6,rhs,`\&'

Value=`0'

if lhs == `1' then if rhs == `1' then Value=`1'

If the and\_expression is not a comparison then:

if \#Tracing.\#Level == `I' then call \#Trace
`\textgreater0O\textgreater{}'

\subsection{The value of an
expression}\label{the-value-of-an-expression}

See nnn for the syntax of an expression.

The value of an expression, or an expr, is the value of the expr\_alias
within it.

If the expr\_alias is an and\_expression then the value of the
expr\_alias is the value of the and\_expression. Otherwise, let Ihs be
the value of the expr\_alias within it, and rhs be the value of the
and\_expression

within it. if lhs == `0' then if lhs == `1' then

call \#Raise `SYNTAX',34.5,lhs,or operator if rhs == `0' then if rhs ==
`1' then

call \#Raise `SYNTAX',34.6,rhs,or operator Value=`1' if lhs == `0' then
if rhs == `0' then Value=`0' If the or\_operator is `\&\&' then if lhs
== `1' then if rhs == `1' then Value=`0' If the expr\_alias is not an
and\_expression then: if \#Tracing.\#Level == `I' then call \#Trace
`\textgreater0O\textgreater{}'

The value of an expression or expr shall be traced when
\#Tracing.\#Level is `R'. The tag is `\textgreater=\textgreater{}' when

the value is used by an assignment and
`\textgreater\textgreater\textgreater{}' when it is not. if
\#Tracing.\#Level == `R' then call \#Trace Tag

\subsection{Arithmetic operations}\label{arithmetic-operations}

The user of this standard is assumed to know the results of the binary
operators `+' and `-' applied to signed or unsigned integers.

The code of \texttt{ArithOpp} itself is assumed to operate under a
sufficiently high setting of numeric digits to avoid exponential
notation.

\lstinputlisting[language=rexx,label=evaluation-arithmetic.rexx,caption=evaluation-arithmetic.rexx]{evaluation-arithmetic.rexx}

\section{Functions}\label{functions}

\subsection{Invocation}\label{invocation}

Invocation occurs when a \emph{function} or a \emph{message\_term} or a
\emph{call} is evaluated. Invocation of a function may result in a
value, in which case:

\lstinputlisting[language=rexx,label=tracinglevelF.rexx,caption=tracinglevelF.rexx]{tracinglevelF.rexx}

Invocation of a \emph{message\_term} may result in a value, in which
case:

\lstinputlisting[language=rexx,label=tracinglevelM.rexx,caption=tracinglevelM.rexx]{tracinglevelM.rexx}

\subsection{Evaluation of arguments}\label{evaluation-of-arguments}

The argument positions are the positions in the \emph{expression\_list}
where syntactically an \emph{expression} occurs or could have occurred.
Let \texttt{ArgNumber} be the number of an argument position, counting
from \texttt{1} at the left; the range of \texttt{ArgNumber} is all
whole numbers greater than zero.

For each value of \texttt{ArgNumber},
\texttt{\#ArgExists.\#NewLevel.ArgNumber} is set
\texttt{\textquotesingle{}1\textquotesingle{}} if there is an expression
present, \texttt{\textquotesingle{}O\textquotesingle{}} if not.

From the left, if \texttt{\#ArgExists.\#NewLevel.ArgNumber} is
\texttt{\textquotesingle{}1\textquotesingle{}} then
\texttt{\#Arg.\#NewLevel.ArgNumber} is set to the value of the
corresponding expression. If \texttt{\#ArgExists.\#NewLevel.ArgNumber}
is \texttt{\textquotesingle{}0\textquotesingle{}} then
\texttt{\#Arg.\#NewLevel.ArgNumber} is set to the null string.

\texttt{\#ArgExists.\#NewLevel.0} is set to the largest
\texttt{ArgNumber} for which \texttt{\#ArgExists.\#NewLevel.ArgNumber}
is \texttt{\textquotesingle{}1\textquotesingle{}}, or to zero if there
is no such value of \texttt{ArgNumber}.

\subsection{The value of a label}\label{the-value-of-a-label}

The value of a \emph{LABEL}, or of the \emph{taken\_constant} in the
function or \emph{call\_instruction}, is taken as a constant, see nnn.
If the \emph{taken\_constant} is not a \emph{string\_literal} it is a
reference to the first \emph{LABEL} in the program which has the same
value. The comparison is made with the `==' operator.

If there is such a matching label and the label is trace-only (see nnn)
then a condition is raised:

\lstinputlisting[language=rexx,label=callraisesyntax.rexx,caption=callraisesyntax.rexx]{callraisesyntax.rexx}

If there is such a matching label, and the label is not trace-only,
execution continues at the label with routine initialization (see nnn).
This is execution of an internal routine.

If there is no such matching label, or if the \emph{taken\_constant} is
a \emph{string\_literal}, further comparisons are made.

If the value of the \emph{taken\_constant} matches the name of some
built-in function then that built-in function is invoked. The names of
the built-in functions are defined in section nnn and are in uppercase.

If the value does not match any built-in function name,
\texttt{Config\_ExternalRoutine} is used to invoke an external routine.

Whenever a matching label is found, the variables \texttt{SIGL} and
\texttt{.SIGL} are assigned the value of the line number of the clause
which caused the search for the label. In the case of an invocation
resulting from a condition occurring that shall be the clause in which
the condition occurred.

\lstinputlisting[language=rexx,label=varset.rexx,caption=varset.rexx]{varset.rexx}

The name used in the invocation is held in \texttt{\#Name.\#Level} for
possible use in an error message from the \texttt{RETURN} clause, see
nnn

\subsection{The value of a function}\label{the-value-of-a-function}

A built-in function completes when it returns from the activity defined
in section nnn. The value of a built-in function is defined in section
nnn.

An internal routine completes when \texttt{\#Level} returns to the value
it had when the routine was invoked. The value of the internal function
is the value of the \emph{expression} on the \emph{return} which
completed the routine. The value of an external function is determined
by \texttt{Config\_ExternalRoutine}.

\subsection{The value of a method}\label{the-value-of-a-method}

A built-in method completes when it returns from the activity defined in
section n.~The value of a built-in method is defined in section n.

An internal method completes when \texttt{\#Level} returns to the value
it had when the routine was invoked. The value of the internal method is
the value of the \emph{expression} on the \emph{return} which completed
the method. The value of an external method is determined by
\texttt{Config\_ExternalMethod}.

\subsection{The value of a message
term}\label{the-value-of-a-message-term}

See nnn for the syntax of a \emph{message\_term}. The value of the
\emph{term} within a \emph{message\_term} is called the receiver.

The receiver and any arguments of the term are evaluated, in left to
\lstinputlisting[language=rexx,label=evaluatemessageterm.rexx,caption=evaluatemessageterm.rexx]{evaluatemessageterm.rexx}

If the message term contains
\texttt{\textquotesingle{}\textasciitilde{}\textasciitilde{}\textquotesingle{}}
the value of the message term is the receiver.

\emph{Any effect on .Result?}

Otherwise the value of a \emph{message\_term} is the value of the method
it invokes. The method invoked is determined by the receiver and the
\emph{taken\_constant} and \emph{symbol}.

\lstinputlisting[language=rexx,label=evaluatemessagetermconstant.rexx,caption=evaluatemessagetermconstant.rexx]{evaluatemessagetermconstant.rexx}

If there is a \emph{symbol}, it is subject to a constraints.

\lstinputlisting[language=rexx,label=evaluatemessagtermsymbol.rexx,caption=evaluatemessagtermsymbol.rexx]{evaluatemessagtermsymbol.rexx}

The search will progress from the object to its class and superclasses.

\lstinputlisting[language=rexx,label=messagemethodsearch.rexx,caption=messagemethodsearch.rexx]{messagemethodsearch.rexx}

\subsection{Use of
Config\_ExternalRoutine}\label{use-of-config_externalroutine}

The values of the arguments to the use of
\texttt{Config\_ExternalRoutine}, in order, are:

The argument \texttt{How} is
\texttt{\textquotesingle{}SUBROUTINE\textquotesingle{}} if the
invocation is from a \emph{call},
\texttt{\textquotesingle{}FUNCTION\textquotesingle{}} if the invocation
is from a \emph{function}.

The argument \texttt{NameType} is
\texttt{\textquotesingle{}1\textquotesingle{}} if the
\emph{taken\_constant} is a \emph{string\_literal},
\texttt{\textquotesingle{}0\textquotesingle{}} otherwise.

The argument \texttt{Name} is the value of the \emph{taken\_constant}.

The argument \texttt{Environment} is the value of this argument on the
\texttt{API\_Start} which started this execution.

The argument \texttt{Arguments} is the \texttt{\#Arg.} and
\texttt{\#ArgExists.} data.

The argument \texttt{Streams} is the value of this argument on the
\texttt{API\_Start} which started this execution.

The argument \texttt{Traps} is the value of this argument on the
\texttt{API\_Start} which started this execution.

\texttt{Var\_Reset} is invoked and \texttt{\#API\_Enabled} set to
\texttt{\textquotesingle{}1\textquotesingle{}} before use of
\texttt{Config\_ExternalRoutine}. \texttt{\#API\_Enabled} is set to
\texttt{\textquotesingle{}O\textquotesingle{}} after.

The response from \texttt{Config\_ExternalRoutine} is processed. If no
conditions are (implicitly) raised, \texttt{\#Outcome} is the value of
the function.
