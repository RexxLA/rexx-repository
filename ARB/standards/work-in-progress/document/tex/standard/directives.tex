%preprocessed texin
\hypertarget{directives}{%
\chapter{Directives}\label{directives}}

The syntax constructs which are introduced by the optional `::' token
are known as directives.

\hypertarget{notation}{%
\section{Notation}\label{notation}}

Notation functions are functions which are not directly accessible as
functions in a program but are used in this standard as a notation for
defining semantics.

Some notation functions allow reference to syntax constructs defined in
nnn. Which instance of the syntax construct in the program is being
referred to is implied; it is the one for which the semantics are being
specified.

The BNF\_primary referenced may be directly in the production or in some
component referenced in the

production, recursively. The components are considered in left to right
order. \#Contains (Identifier, BNF primary)

where:

Identifier is an identifier in a production (see nnn) defined in nnn.

BNF\_primary is a bnf\_primary (see nnn) in a production defined in nnn.
Return `1' if the production identitied by identifier contained a
bnf\_primary identified by BNF\_primary, otherwise return `0'.

\#Instance (Identifier, BNF primary) where: Identifier is an identifier
in a production defined in nnn. BNF\_primary is a Onf\_primary in a
production defined in nnn. Returns the content of the particular
instance of the BNF\_primary. If the BNF\_primary is a VAR\_SYMBOL this
is referred to as the symbol ``taken as a constant.''

\#Evaluate (Identifier, BNF primary) where: Identifier is an identifier
in a production defined in nnn. BNF\_primary is a Onf\_primary in a
production defined in nnn. Return the value of the BNF\_primary in the
production identified by Identifier.

\#Execute (Identifier, BNF primary) where: Identifier is an identifier
in a production defined in nnn. BNF\_primary is a Onf\_primary in a
production defined in nnn. Perform the instructions identified by the
BNF\_primary in the production identified by Identifier.

\#Parses (Value, BNF primary) where: Value is a string BNF\_primary is a
Onf\_primary in a production defined in nnn. Return `1' if Value matches
the definition of the BNF\_primary, by the rules of clause 6, `0'
otherwise.

\#Clause (Label) where: Label is a label in code used by this standard
to describe processing. Return an identification of that label. The
value of this identification is used only by the \#Goto notation
function.

\#Goto (Value) where:

Value identifies a label in code used by this standard to describe
processing. The description of processing continues at the identified
label.

\#Retry ()

This notation is used in the description of interactive tracing to
specify re-execution of the clause just previously executed. It has the
effect of transferring execution to the beginning of that clause, with
state variable \#Loop set to the value it had when that clause was
previously executed.

\hypertarget{initializing}{%
\section{Initializing}\label{initializing}}

Some of the initializing, now grouped in classic section 8.2.1 will have
to come here so that we have picked up anything from the START\_API that
needs to be passed on to the execution of REQUIRES subject.

We will be using some operations that are forward reference to what was
section nnn.

\hypertarget{program-initialization-and-message-texts}{%
\subsection{Program initialization and message
texts}\label{program-initialization-and-message-texts}}

Processing of a program begins when API\_Start is executed. A pool
becomes current for the reserved

variables. call Config ObjectNew \#ReservedPool = \#Outcome \#Pool =
\#ReservedPool Is it correct to make the reserved variables and the
builtin objects in the same pool?

Some of the values which affect processing of the program are parameters
of API\_ Start:

\#Howlnvoked is set to `COMMAND', `FUNCTION' or `SUBROUTINE' according
to the first parameter of APL Start.

\#Source is set to the value of the second parameter of API\_ Start.

The third parameter of API\_Start is used to determine the initial
active environment.

The fourth parameter of API Start is used to determine the arguments.
For each argument position \#ArgExists.1.ArgNumber is set `1' if there
is an argument present, `0' if not. ArgNumber is the number of the
argument position, counting from 1. If \#ArgExists.1.ArgNumber is `1'
then \#Arg.1.ArgNumber is set to the value of the corresponding
argument. If \#ArgExists.1.ArgNumber is `0' then \#Arg.1.Arg is set to
the null string. \#ArgExists.1.0 is set to the largest n for which
\#ArgExists.1.n is `1', or to zero if there is no such value of n.

Some of the values which affect processing of the program are provided
by the configuration: call Config OtherBlankCharacters

\#A11Blanks\textless Index ``\#Al1Blanks'' \# ``\,'' \textgreater{} = '
'\#Outcome /* ``Real'' blank concatenated with others */

\#Bif Digits. = 9

call Config Constants

-true = `1'

-false = `0O'

Objects in our model are only distinquished by the values within their
pool so we can construct the builtin classes incomplete and then
complete them with directives.

Can we initialize the methods of .nil by directives?

call Config ObjectNew

-List = \#Outcome

call var\_set .List, \#IsClass, `0', `1'

call var\_set .List, \#ID, `0', `List'

Some of the state variables set by this call are limits, and appear in
the text of error messages. The relation between message numbers and
message text is defined by the following list, where the message

number appears immediately before an `=' and the message text follows in
quotes.

\#ErrorText. stl

\#ErrorText.0.1 = `Error running , line :'

\#ErrorText.0.2 = `Error in interactive trace:'

\#ErrorText.0.3 = `Interactive trace. ``Trace Off'' to end debug.',
``ENTER to continue.'

\#ErrorText.2 = `Failure during finalization'

\#ErrorText.2.1 = `Failure during finalization: '

\#ErrorText.3 \#ErrorText.3.1

`Failure during initialization' `Failure during initialization: '

\#ErrorText.4 \#ErrorText.4.1

`Program interrupted' `Program interrupted with HALT condition: '

\#ErrorText.5 \#ErrorText.5.1

`System resources exhausted' `System resources exhausted: '

\#ErrorText.6 \#ErrorText.6.1

'Unmatched ``/\emph{'' or quote! 'Unmatched comment delimiter (''/}'')!

\#ErrorText.6.2 \#ErrorText.6.3

\#ErrorText.7 = \#ErrorText.7.1 =

\#ErrorText.7.2 =

\#ErrorText.7.3 =

\#ErrorText.8 = \#ErrorText.8.1 \#ErrorText.8.2

\#ErrorText.9 = \#ErrorText.9.1 \#ErrorText.9.2

\#ErrorText.10 \#ErrorText.10.1= \#ErrorText.10.2=

\#ErrorText.10.3=

\#ErrorText.10.4= \#ErrorText.10.5= \#ErrorText.10.6=

\#ErrorText.13 = \#ErrorText.13.1=

\#ErrorText.14 = \#ErrorText.14.1= \#ErrorText.14.2= \#ErrorText.14.3=
\#ErrorText.14.4=

\#ErrorText.15 = \#ErrorText.15.1=

\#ErrorText.15.2= \#ErrorText.15.3= \#ErrorText.15.4= \#ErrorText.16 =
\#ErrorText.16.1= \#ErrorText.16.2= \#ErrorText.16.3=

\#ErrorText.17 = \#ErrorText.17.1=

\#ErrorText.17.2=

\#ErrorText.18 =

\#ErrorText.18.1=

\#ErrorText.18.2=

``Unmatched single quote (`)'' `Unmatched double quote (``)'

`WHEN or OTHERWISE expected' ``SELECT on line `found ``''\,'

``SELECT on line `or END; found ``''\,'!

requires WHEN;',

`All WHEN expressions of SELECT on line are',

`false; OTHERWISE expected'

``Unexpected THEN or ELSE'

`THEN has no corresponding IF or WHEN clause' `ELSE has no corresponding
THEN clause'

``Unexpected WHEN or OTHERWISE' `WHEN has no corresponding SELECT'
`OTHERWISE has no corresponding SELECT'

`Unexpected or unmatched END'

`END has no corresponding DO or SELECT'

`END corresponding to DO on line ', `must have a symbol following that
matches', `the control variable (or no symbol);',

`found ``''\,'!

`END corresponding to DO on line ', `must not have a symbol following it
because', `there is no control variable;',

`found ``''\,'!

`END corresponding to SELECT on line ', `must not have a symbol
following;',

`found ``''\,'!

`END must not immediately follow THEN'

`END must not immediately follow ELSE'

`Invalid character in program' `Invalid character ``('`X)'' in program'

``Incomplete DO/SELECT/IF'

`DO instruction requires a matching END' `SELECT instruction requires a
matching END' `THEN requires a following instruction' `ELSE requires a
following instruction'

`Invalid hexadecimal or binary string'

`Invalid location of blank in position', ` in hexadecimal string'

`Invalid location of blank in position', ` in binary string'

`Only 0-9, a-f, A-F, and blank are valid in a', ``hexadecimal string;
found''``\,'

`Only 0, 1, and blank are valid in a',

`binary string; found ``''\,'

`Label not found'

`Label ``'' not found'

`Cannot SIGNAL to label ``'' because it is', `inside an IF, SELECT or DO
group'

`Cannot invoke label ``'' because it is', `inside an IF, SELECT or DO
group'

`Unexpected PROCEDURE'

``PROCEDURE is valid only when it is the first', `instruction executed
after an internal CALL', `or function invocation'

`The EXPOSE `instruction executed after a method invocation!' `THEN
expected'

`IF keyword on line requires', `matching THEN clause; found ``''\,'

`WHEN keyword on line requires',

requires WHEN, OTHERWISE,',

instruction is valid only when it is the first', \#ErrorText.19 =
\#ErrorText.19.1=

\#ErrorText.19.2= \#ErrorText.19.3= \#ErrorText.19.4= \#ErrorText.19.6=
\#ErrorText.19.7= \#ErrorText.19.8=

\#ErrorText.19.9=

\#ErrorText.19.11='String or symbol

\#ErrorText.19.12='String or symbol

\#ErrorText.19.13=

\#ErrorText.19.15='String or symbol

\#ErrorText.19.16='String or symbol

\#ErrorText.19.17=

Unsound now we are using ``term'?

65

\#ErrorText.20 = \#ErrorText.20.1= \#ErrorText.20.2= \#ErrorText.20.3=

\#ErrorText.21 \#ErrorText.21.1

\#ErrorText.22 = \#ErrorText.22.1=

\#ErrorText.23 = \#ErrorText.23.1=

\#ErrorText.24 = \#ErrorText.24.1= \#ErrorText.25 = \#ErrorText.25.1=
\#ErrorText.25.2= \#ErrorText.25.3= \#ErrorText.25.4= \#ErrorText.25.5=
\#ErrorText.25.6= \#ErrorText.25.7=

\#ErrorText.25.8=

`matching THEN clause; found ``''\,' `String or symbol expected' `String
or symbol expected after ADDRESS keyword;', ``found''``! `String or
symbol expected after CALL keyword;',''found ``''! `String or symbol
expected after NAME keyword;', ``found''``! `String or symbol expected
after SIGNAL keyword;',''found ``''! `String or symbol expected after
TRACE keyword;', ``found''``! `Symbol expected in parsing
pattern;',''found ``''! `String or symbol expected after REQUIRES;',
``found''``! `String or symbol expected after METHOD;',''found ``''!

expected after ROUTINE;', ``found''``!

expected after CLASS;', ``found''``! `String or symbol expected after
INHERIT;',''found ``''!

expected after METACLASS;', ``found''``!

expected after MIXINCLASS;`, ``found''``! `String or symbol expected
after SUBCLASS;',''found ``''! 'Name expected' `Name required; found
``''\,'

`Found ``'' where only a name is valid'

'Found ``'' where only a name or

`Invalid data on end of clause' `The clause ended at an unexpected
token;',

`found ``''\,'

'Invalid ``Invalid

'Invalid ``Invalid

'Invalid

character string' character string

data string' data string

TRACE request'

is valid'

''xX''

`'! xX''

`TRACE request letter must be one of',

' ``ACEFILNOR'';

found

``evalue\textgreater{}'''

`Invalid sub-keyword found' `CALL ON must be followed by one of the',

'keywords ;

found

``etoken\textgreater{}''!

`CALL OFF must be followed by one of the',

'keywords ;

found

``etoken\textgreater{}''!

``SIGNAL ON must be followed by one of the',

'keywords ;

found

``etoken\textgreater{}''!

`SIGNAL OFF must be followed by one of the',

'keywords ;

found

``etoken\textgreater{}''!

``ADDRESS WITH must be followed by one of the',

'keywords ;

found

``etoken\textgreater{}''!

``INPUT must be followed by one of the',

'keywords ; 'OUTPUT must be followed by 'keywords ; ``APPEND must be
followed by 'keywords ;

found

found

found

``etoken\textgreater{}''! one of the', ``etoken\textgreater{}''! one of
the', ``etoken\textgreater{}''!

\#ErrorText.25.9=

\#ErrorText.25.11= \#ErrorText.25.12= \#ErrorText.25.13=
\#ErrorText.25.14= \#ErrorText.25.15= \#ErrorText.25.16=
\#ErrorText.25.17=

\#ErrorText.25.18=

\#ErrorText.26 \#ErrorText.26.1=

\#ErrorText.26.2=

\#ErrorText.26.3=

\#ErrorText.26.4= \#ErrorText.26.5=

\#ErrorText.26.6=

\#ErrorText.26.7= \#ErrorText.26.8=

\#ErrorText.26.11

\#ErrorText.26.12

\#ErrorText.27 \#ErrorText.27.1=

\#ErrorText.28 \#ErrorText.28.1= \#ErrorText.28.2= \#ErrorText.28.3=

\#ErrorText.28.4=

\#ErrorText.29 \#ErrorText.29.1=

\#ErrorText.30 \#ErrorText.30.1= \#ErrorText.30.2=

\#ErrorText.31 \#ErrorText.31.1=

\#ErrorText.31.2=

\#ErrorText.31.3=

``REPLACE must be followed by one of the', `keywords ; found
``''\,'\,''NUMERIC FORM must be followed by one of the', `keywords ;
found ``''\,'

``PARSE must be followed by one of the', `keywords ; found ``''\,'

``UPPER must be followed by one of the', `keywords ; found ``''\,'

``ERROR must be followed by one of the', `keywords ; found
``''\,'\,''NUMERIC must be followed by one of the', `keywords ; found
``''\,' `FOREVER must be followed by one of the', ``keywords or nothing;
found''``'''PROCEDURE must be followed by the keyword', ``EXPOSE or
nothing; found''``\,'

``FORWARD must be followed by one of the the keywords',

`; found ``''\,'

`Invalid whole number'

`Whole numbers must fit within current DIGITS', `setting(); found
``''\,'

`Value of repetition count expression in DO instruction',

`must be zero or a positive whole number;', `found ``''\,'

`Value of FOR expression in DO instruction', `must be zero or a positive
whole number;', `found ``''\,'

``Positional pattern of parsing template', `must be a whole number;
found ``''\,'\,''NUMERIC DIGITS value',

`must be a positive whole number; ``NUMERIC FUZZ value',

`must be zero or a positive whole number;',

`found ``''\,'

`Number used in TRACE setting',

`must be a whole number; found ``''\,'

'Operand to right of the power operator (``**``)`, 'must be a whole
number; found ``''\,'

``Result of \% operation would need',

found ``'''

`exponential notation at current NUMERIC DIGITS '

``Result of \% operation used for // `, `operation would need',

`exponential notation at current NUMERIC DIGITS '

`Invalid DO syntax' `Invalid use of keyword ``'' in DO clause'

`Invalid LEAVE or ITERATE'

`LEAVE is valid only within a repetitive DO loop' `ITERATE is valid only
within a repetitive DO loop' `Symbol following LEAVE (``'') must',

`either match control variable of a current',

`DO loop or be omitted'

`Symbol following ITERATE (``'') must', `either match control variable
of a current',

`DO loop or be omitted'

`Environment name too long' `Environment name exceeds',

\#Limit EnvironmentName `characters; found ``''\,' `Name or string too
long'

`Name exceeds' \#Limit Name `characters' ``Literal string exceeds'
\#Limit Literal `characters' `Name starts with number or ``.''\,'!

`A value cannot be assigned to a number;',

`found ``''\,'

`Variable symbol must not start with a number;', `found ``''\,'

`Variable symbol must not start with a ``.'';',

`found ``''\,' \#ErrorText.33 = \#ErrorText.33.1=

\#ErrorText.33.2=

\#ErrorText.33.3=

\#ErrorText.34 = \#ErrorText.34.1=

`Invalid expression result' `Value of NUMERIC DIGITS (``'')',

'must exceed value of NUMERIC FUZZ

``(\textless¢value\textgreater)'''!

`Value of NUMERIC DIGITS (``'')',

`must not exceed'

\#Limit Digits

``Result of expression following NUMERIC FORM',

'must start with ``E''

or ``S''; found ``'''

``Logical value not''0'' or ``1''!

'Value of expression

following IF keyword',

`must be exactly ``0'' or ``1''; found ``''\,' \#ErrorText.34.2= `Value
of expression following WHEN keyword',

`must be exactly ``0'' or ``1''; found ``''\,' \#ErrorText.34.3= `Value
of expression following WHILE keyword',

`must be exactly ``0'' or ``1''; found ``''\,' \#ErrorText.34.4= `Value
of expression following UNTIL keyword',

`must be exactly ``0'' or ``1''; found ``''\,' \#ErrorText.34.5= `Value
of expression to left',

`of logical operator ``''\,',

`must be exactly ``0'' or ``1''; found ``''\,' \#ErrorText.34.6= `Value
of expression to right',

`of logical operator ``''\,',

`must be exactly ``0'' or ``1''; found ``''\,' \#ErrorText.35 = `Invalid
expression' \#ErrorText.35.1= `Invalid expression detected at ``''\,'

\#ErrorText.36

`Unmatched ``('' in expression'

\#ErrorText.37 = \#ErrorText.37.1= \#ErrorText.37.2=

'Unexpected n a n ny nt

`Unexpected ``,''! `Unmatched ``)'' in expression'

or

\#ErrorText.38 = \#ErrorText.38.1= \#ErrorText.38.2= \#ErrorText.38.3=

`Invalid template or pattern'

`Invalid parsing template detected at ``''\,' `Invalid parsing position
detected at ``''\,' ``PARSE VALUE instruction requires WITH keyword'

``Incorrect call to routine'

`External routine ``'' failed'

`Not enough arguments in invocation of ;', `minimum expected is '

`Too many arguments in invocation of ;', `maximum expected is '

`Missing argument in invocation of ;', `argument is required'
`ebif\textgreater{} argument ',

`exponent exceeds' \#Limit ExponentDigits ``found''``\,'

` argument ',

`must be a number; found ``''\,'! ` argument ',

\#ErrorText.40 = \#ErrorText.40.1= \#ErrorText.40.3= \#ErrorText.40.4=
\#ErrorText.40.5=

\#ErrorText.40.9= `digits;',

\#ErrorText.40.11=

\#ErrorText.40.12=

`must be a whole number; found ``''\,' \#ErrorText.40.13=` argument ',

`must be zero or positive; found ``''\,' \#ErrorText.40.14=` argument ',

`must be positive; found ``''\,'

\#ErrorText.40.17=` argument 1', `must have an integer part in the range
0:90 and a', `decimal part no larger than .9; found ``''\,'
\#ErrorText.40.18=` conversion must',

``have a year in the range 0001 to 9999! \#ErrorText.40.19=` argument 2,
``'', is not in the format',

`described by argument 3, ``''\,'

\#ErrorText.40.21=` argument must not be null' \#ErrorText.40.23=`
argument ',

`must be a single character; found ``''\,' \#ErrorText.40.24=` argument
1',

`must be a binary string; found ``''\,' \#ErrorText.40.25=` argument 1',

`must be a hexadecimal string; found ``''\,' \#ErrorText.40.26=`
argument 1',

`must be a valid symbol; found ``''\,' \#ErrorText.40.27=` argument 1',

`must be a valid stream name; found ``''\,' \#ErrorText.40.28=` argument
,',

`option must start with one of ``'';',

`found ``''\,'

\#ErrorText.40.29=` conversion to format ``'' is not allowed'
\#ErrorText.40.31=` argument 1 (``'') must not exceed 100000'
\#ErrorText.40.32=` the difference between argument 1 (``'') and',

`argument 2 (``'') must not exceed 100000' \#ErrorText.40.33=` argument
1 (``'') must be less than',

`or equal to argument 2 (``'')' \#ErrorText.40.34=` argument 1 (``'')
must be less than',

`or equal to the number of lines',

`in the program (\textless sourceline()\textgreater)'
\#ErrorText.40.35=` argument 1 cannot be expressed as a whole number;',

`found ``''\,'

\#ErrorText.40.36=` argument 1', `must be the name of a variable in the
pool;', `found ``''\,'

\#ErrorText.40.37=` argument 3',

`must be the name of a pool; found ``''\,' \#ErrorText.40.38=` argument
',

`is not large enough to format ``''\,' \#ErrorText.40.39=` argument 3 is
not zero or one; found ``''\,' \#ErrorText.40.41=` argument ',

`must be within the bounds of the stream;',

`found ``''\,'

\#ErrorText.40.42=` argument 1; cannot position on this stream;', `found
``''\,' \#ErrorText.40.45=` argument must be a single',

`non-alphanumeric character or the null string;',

' ``found '''

\#ErrorText.40.46=` argument 3, ``'', is a format incompatible',

`with separator specified in argument '

\#ErrorText.41 = `Bad arithmetic conversion' \#ErrorText.41.1=
`Non-numeric value (``'')', `to left of arithmetic operation ``''\,'

\#ErrorText.41.2= `Non-numeric value (``'')', `to right of arithmetic
operation ``''\,' \#ErrorText.41.3= `Non-numeric value (``'')',

`used with prefix operator ``''\,'

\#ErrorText.41.4= `Value of TO expression in DO instruction', `must be
numeric; found ``''\,'!

\#ErrorText.41.5= `Value of BY expression in DO instruction', `must be
numeric; found ``''\,'!

\#ErrorText.41.6= `Value of control variable expression of DO
instruction', `must be numeric; found ``''\,'!

\#ErrorText.41.7= `Exponent exceeds' \#Limit ExponentDigits `digits;',
`found ``''\,'

\#ErrorText.42 = `Arithmetic overflow/underflow' \#ErrorText.42.1l=
`Arithmetic overflow detected at', `Nevalue\textgreater{} ``;',
`exponent of result requires more than', \#Limit ExponentDigits `digits'
\#ErrorText.42.2= `Arithmetic underflow detected at',
`Nevalue\textgreater{} ``;`, `exponent of result requires more than',
\#Limit ExponentDigits `digits'\,''Arithmetic overflow; divisor must not
be zero'

\#ErrorText.42.3

``Routine not found' `Could not find routine ``''\,'

\#ErrorText.43 = \#ErrorText.43.1= \#ErrorText.44 = `Function did not
return data'

\#ErrorText.44.1= `No data returned from function ``''\,'

\#ErrorText.45 = `No data specified on function RETURN'
\#ErrorText.45.1= `Data expected on RETURN instruction because',
`routine ``'' was called as a function'

\#ErrorText.46 = `Invalid variable reference' \#ErrorText.46.1= `Extra
token (``'') found in variable', `reference; ``)'' expected'

\#ErrorText.47 = `Unexpected label' \#ErrorText.47.1l= `INTERPRET data
must not contain labels;', 'found ``''!

\#ErrorText.48 = `Failure in system service' \#ErrorText.48.1= `Failure
in system service: '

\#ErrorText.49 = `Interpretation Error' \#ErrorText.49.1=
`Interpretation Error: ' \#ErrorText.50 = `Unrecognized reserved symbol'

\#ErrorText.50.1= `Unrecognized reserved symbol ``''\,'

\#ErrorText.51 = `Invalid function name'

\#ErrorText.51.1= `Unquoted function names must not end with a period;',
`found ``''\,'

\#ErrorText.52

``Result returned by'''' is longer than', \#Limit String `characters'

\#ErrorText.53 = `Invalid option' \#ErrorText.53.1l= `Variable reference
expected', `after STREAM keyword; found ``''\,' \#ErrorText.53.2=
`Variable reference expected', `after STEM keyword; found ``''\,'
\#ErrorText.53.3= `Argument to STEM must have one period,', `as its last
character; found ``''\,' \#ErrorText.54 = `Invalid STEM value'
\#ErrorText.54.1= `For this use of STEM, the value of ``'' must be a',
`count of lines; found: ``''\,'

If the activity defined by clause 6 does not produce any error message,
execution of the program continues.

call Config NoSource

If Config\_NoSource has set \#NoSource to `0' the lines of source
processed by clause 6 are copied to \#SourceLine. , with \#SourceLine.O
being a count of the lines and \#SourceLine.n for n=1 to \#SourceLine.0
being the source lines in order.

If Config\_NoSource has set \#NoSource to `1' then \#SourceLine.0 is set
to 0. The following state variables affect tracing:

\#InhibitPauses = 0

\#InhibitTrace = 0

\#AtPause = 0 /* Off until interactive input being received. */

\#Trace QueryPrior = `No' An initial variable pool is established:

call Config ObjectNew

\#Pool = \#Outcome

\#P0011 = \#Pool

call Var\_Empty \#Pool

call Var\_Reset \#Pool

\#Level = 1 /* Level of invocation */ \#NewLevel = 2
\#IsFunction.\#Level = (\#HowInvoked == `FUNCTION')

For this first level, there is no previous level from which values are
inherited. The relevant fields are initialized.

\#Digits.\#Level = 9 /* Numeric Digits \emph{/ \#Form.\#Level =
`SCIENTIFIC' /} Numeric Form \emph{/ \#Fuzz.\#Level = 0 /} Numeric Fuzz
\emph{/ \#StartTime.\#Level = '\,' /} Elapsed time boundary */
\#LineNumber = '\,'

\#Tracing.\#Level = `N'

\#Interactive.\#Level = `0'

69 An environment is provided by the API\_ Start to become the initial
active environment to which commands will be addressed. The alternate
environment is made the same:

/* Call the environments ACTIVE, ALTERNATE, TRANSIENT where these are
never-initialized state variables.

Similarly call the redirections I O and E */

call EnvAssign ALTERNATE, \#Level, ACTIVE, \#Level

Conditions are initially disabled:

\#Enabling.SYNTAX.\#Level = `OFF' \#Enabling.HALT.\#Level = `OFF'
\#Enabling.ERROR.\#Level = `OFF' \#Enabling.FAILURE.\#Level = `OFF'
\#Enabling.NOTREADY.\#Level = `OFF' \#Enabling.NOVALUE.\#Level = `OFF'
\#Enabling.LOSTDIGITS.\#Level = `OFF'

\#PendingNow.HALT.\#Level = 0 \#PendingNow.ERROR.\#Level = 0
\#PendingNow.FAILURE.\#Level = 0 \#PendingNow.NOTREADY.\#Level = 0 /*
The following field corresponds to the results from the CONDITION
built-in function. */ \#Condition.\#Level = '\,' The opportunity is
provided for a trap to initialize the pool. \#API Enabled = `1' call
Var\_Reset \#Pool call Config Initialization \#API Enabled = `0' \#\#
REQUIRES For each requires in order of appearence: A use of Start\_API
with \#instance(requires, taken\_constant). Msg40.1 or a new if
completion `E'. Add Provides to an ordered collection. Not cyclic
because .LIST can be defined without defining REQUIRES but a fairly
profound forward reference. \#\# CLASS For each class in order of
appearence: \#ClassName = \#Instance(class, taken constant) call
var\_value \#ReservedPool, `\#CLASSES.'ClassName, '1' if \#Indicator ==
`D' then do call Config ObjectNew \#Class = \#Outcome call var\_set
\#ReservedPool, `\#CLASSES.'ClassName, '1', \#Class end else call
\#Raise `SYNTAX', nn.nn, \#ClassName

New instance of CLASS class added to list. Msg ``Duplicate ::CLASS
directive instruction''(?) \#\# METHOD

For each method in order of appearence: call Config ObjectNew \#Po00ol =
\#Outcome call Config ObjectSource (\#Pool) \#MethodName =
\#Instance(method, taken constant) call var\_value \#Class,
`\#METHODS.'\#MethodName, `1' if \#Indicator == `D' then call var set
\#Class, `\#METHODS.'\#MethodName, `1', \#Pool else call \#Raise
`SYNTAX', nn.nn, \#MethodName, \#ClassName

GUARDED \& public is default. if \#contains(method, `PRIVATE') then
m\textasciitilde setprivate; if \#contains(method, 'UNGUARDED)) then
m\textasciitilde setunguarded

Why is there a keyword for GUARDED but not for PUBLIC here?

Does CLASS option mean ENHANCE with Class class methods?

\#CurrentClass \textasciitilde class(\#instance(method,
taken\_constant), m)

For ATTRIBUTE, should we actually construct source for two methods?
ATTRIBUTE case needs test of null body. OO! doesn't have source (because
it actually traps UNKNOWN?).

For EXTERNAL test for null body. Simon Nash doc says ``Accessibility to
external methods \ldots{} is implementation-defined''. Left like that it
doesn't even tell us about search order. We will need a
Config\_ExternalClass to import the public names of the class.

\hypertarget{routine}{%
\section{ROUTINE}\label{routine}}

For each routine in order of appearence:

Add name (with duplicate check) to list for this file.

Extra step needed in the invocation search order. Although this is
nominally EXTERNAL we presumably wont use the external call mechanism.
(Except perhaps when the routine was made available by a REQUIRES; in
that case the PARSE SOURCE answer has to change.)

have the builtins-defined-by-directives elsewhere; it would make sense
if they wound up about here.
